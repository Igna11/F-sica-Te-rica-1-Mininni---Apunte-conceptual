\section{Repaso de electromagnetismo clásico}
Toda la materia parte de las ecuaciones de Maxwell sin fuentes:
\begin{equation}
    \begin{matrix}
        \nabla \cdot \textbf{E} = 4\pi\rho 
        & \mbox{Ley de Gauss}\\
        &\\
        \nabla \times \textbf{E} = -\frac{1}{c}\frac{\partial \textbf{B}}{\partial t}
        & \mbox{Ley de Faraday}\\
        &\\
        \nabla \cdot \textbf{B} = 0 
        & \mbox{Inexistencia de monopolos}\\
        &\\
        \nabla \times \textbf{B} = \frac{4\pi}{c}\textbf{J} + \frac{1}{c}\frac{\partial \textbf{E}}{\partial t}
        & \mbox{Ley de Ampere generalizada}
        
    \end{matrix}
    \label{ec:EcuacionesDeMaxwell}
\end{equation}
Estas son las ecuaciones de Maxwell sin fuentes en el \textbf{sistema de unidades CGS Gaussiano, en el vacío}. Son ecuaciones para el rotor y divergencia del campo eléctrico $\textbf{E}$ y el campo magnético $\textbf{B}$. \\
\indent Algunos comentarios importantes de estas ecuaciones: Son ecuaciones que surgen de modelar resultados experimentales, no fueron derivadas teóricamente a partir de primeros principios. Además son ecuaciones desarrolladas en el siglo XIX que perduran hasta la fecha sin sufrir modificaciones y que siguen valiendo para el modelo estandar. Incluso entraron en conflicto con la mecánica Newtoniana y se terminó desarrollando la mecánica relativista.\\
\indent En la ecuación para el rotor del campo magnético (ley de Ampere), Maxwell agrega el término $\frac{1}{c}\frac{\partial \textbf{E}}{\partial t}$ (Corriente de desplazamiento) para que se cumpla la conservación de carga. Este resultado es importante porque, además de hacer que las ecuaciones satisfagan la conservación de carga, también predice la existencia de ondas electromagnéticas. ¿Y de dónde viene este término? Bueno, resulta que si se toma la ley de Gauss y se la deriva respecto al tiempo, como las derivadas espaciales conmutan con las temporales, se llega a la siguiente expresión
\begin{equation*}
    \frac{\partial \rho}{\partial t} 
    = \frac{1}{4\pi}\nabla \cdot 
    \left(
        \frac{\partial \textbf{E}}{\partial t}
    \right)
\end{equation*}
de donde se puede usar la ley de Ampere modificada para reemplazar $\frac{\partial \textbf{E}}{\partial t}$. Haciendo esto, aparece una divergencia de un rotor (se anula) y aparece la divergencia de la corriente $\textbf{J}$. En definitiva, queda esto
\begin{equation}
    \frac{\partial \rho}{\partial t} + \nabla \cdot \textbf{J} = 0
        \label{ec:EcuacionDeContinuidad}
\end{equation}
que es la ecuación de continuidad o ecuación de conservación de carga! O sea que hay dos formas de verlo: Se puede llegar a la ecuación de continuidad de carga a partir de las ecuaciones de Maxwell, solo si estas tienen el término que agrega Maxwell para que satisfaga conservación de carga; Se puede llegar a las ecuaciones de Maxwell con la ley de Ampere generalizada pidiendo que las ecuaciones satisfagan la ecuación de carga.\\
\indent Ahora, alcanza con tener definidos la divergencia y el rotor de un campo para tener la descripción completa de ese campo? La respuesta es sí y para eso voy a mostrar la \textbf{descomposición de Helmholtz}.


%%%%%%%%%%%%%%%%%%%%%%%%%%%%%%%%%%%%%%%%%%%%%%%%
%%%%%%%%%%%%%%%%%%%%%%%%%%%%%%%%%%%%%%%%%%%%%%%%


\subsection{Descomposición de Helmholtz}
Dado un campo vectorial $\textbf{W}(\textbf{x})$ cualquiera, este se puede descomponer unívocamente (a menos de un Gauge a elección) en un campo irrotacional $\textbf{V}$ más un campo soleniodal (sin divergencia) $\textbf{U}$, es decir
\begin{equation*}
    \textbf{W} = \textbf{U} + \textbf{V}
    \quad
    \quad
    \mbox{y}
    \quad
    \nabla \times \textbf{V} = 0,\ \ \nabla \cdot \textbf{U} = 0
\end{equation*}
A partir de esta, usando que $\nabla \times \textbf{V} = 0$ y que $\nabla \cdot \textbf{U} =0$, entonces vale que $\textbf{V} = \nabla \varphi$ y $\textbf{U} = \nabla \times \textbf{A}$, donde $\varphi$ es un potencial escalar y $\textbf{A}$ es un potencial vector. De esta forma la descomposición se puede reescribir en función de estos, de la siguiente forma
\begin{equation*}
    \textbf{W} = \nabla \times \textbf{A} + \nabla \varphi
\end{equation*}
Pero falta determinar los campos $\textbf{A}$ y $\varphi$. Resulta estos campos se puede determinar completamente a partir de las siguientes \footnote{Ver apéndice 1.8}
\begin{equation*}
    \varphi 
    = -\frac{1}{4\pi}\int 
    \frac{\nabla \cdot \textbf{W}}{|\textbf{r} - \textbf{r}'|}d^{3}\textbf{r}'
\end{equation*}
y 
\begin{equation*}
    \textbf{A} 
    = \frac{1}{4\pi}\int 
    \frac{\nabla \times \textbf{W}}{|\textbf{r} - \textbf{r}'|}d^{3}\textbf{r}'
\end{equation*}


%%%%%%%%%%%%%%%%%%%%%%%%%%%%%%%%%%%%%%%%%%%%%%%%
%%%%%%%%%%%%%%%%%%%%%%%%%%%%%%%%%%%%%%%%%%%%%%%%


\subsection{Electrostática}
Ley experimental de Coulomb: La fuerza que siente una carga $q_{i}$ ubicada en la posición $\textbf{r}_{i}$ debido a la presencia de otra carga $q_{2}$ ubicada en $\textbf{r}_{2}$ es
\begin{equation}
    \textbf{F}_{12} =
    k q_{1}q_{2}
    \frac{\textbf{r}_{1}-\textbf{r}_{2}}{|\textbf{r}_{1}-\textbf{r}_{2}|^{3}}
        \label{ec:LeydeCoulomb}
\end{equation}
y notar que $\textbf{F}_{12} = -\textbf{F}_{21}$ (satisface el principio de acción fuerte). Se define el campo eléctrico como la fuerza que siente una dada carga, en presencia de otra, cuando esta otra tiende a cero, es decir
\begin{equation}
    \textbf{E}(\textbf{r}) = \lim_{\delta q \to 0} \frac{\textbf{F}{(\delta q)}}{\delta q}
        \label{ec:DefiniciondeCampoElectrico}
\end{equation}
O sea, se piensa que se tiene una dada distribución de carga y una carga de prueba $\delta q$, que se la hace tender a cero. Se mide la intensidad y la dirección de la fuerza que siente la carga dada la presencia de la distribución para diferentes puntos del espacio. De estas mediciones se puede trazar un mapa con las intensidades y direcciones de la fuerza que siente la carga de prueba. Este mapa es el campo eléctrico.\\
\indent De esto se sigue que el campo eléctrico para una carga puntual será
\begin{equation*}
    \textbf{E} = q\frac{\textbf{r}-\textbf{r}'}{|\textbf{r}-\textbf{r}'|^{3}}
\end{equation*}
donde $\textbf{r}'$ es el punto fuente, son las posiciones donde se encuentra la carga, y $\textbf{r}$ es el punto campo, la posición donde se quiere calcular la magnitud del campo eléctrico generado por la carga.\\ 
\indent Para el caso general, dada una distribución de carga arbitraria $\rho(\textbf{r}')$, el campo eléctrico de la distribución será
\begin{equation}
    \textbf{E}(\textbf{r}) =
    \int \rho(\textbf{r}')
    \frac{\textbf{r}-\textbf{r}'}{|\textbf{r}-\textbf{r}'|^{3}}
    d^{3}\textbf{r}'
        \label{ec:CampoElectrico}
\end{equation}
y esto es así porque vale el principio de superposición, es decir, la contribución de cada una de las cargas se suma linealmente.

\subsubsection{Distribuciones}
En esta parte hace un par de cuentas para mostrar qué es la de delta de Dirac y además muestra a modo de ejemplo que la derivada de la función de Heaviside es la delta de Dirac\footnote{arranca más o menos por el minuto 1:15 de video de la clase 1}.


%%%%%%%%%%%%%%%%%%%%%%%%%%%%%%%%%%%%%%%%%%%%%%%%
%%%%%%%%%%%%%%%%%%%%%%%%%%%%%%%%%%%%%%%%%%%%%%%%


\subsubsection{Ley de Gauss}
Se quiere ver que la ley de Gauss se sigue de la ley de Coulomb. Dada una carga puntual $q$ ubicada en $\textbf{r}'$. Dada una superficie arbitraria $S$, que encierra la carga puntual, con diferencial $d\textbf{S} = dS\hat{\textbf{n}}$, con $\hat{\textbf{n}}$ la normal a la superficie. Se define el ángulo $\theta$ como el ángulo formado por la dirección del campo magnético y la normal a la superficie.\\
\indent Se quiere calcular el flujo de campo eléctrico que atraviesa la superficie \textbf{S}. Eso no es más que integrar el campo eléctrico en la superficie cerrada, es decir
\begin{equation*}
    \phi 
    = \oint\limits_{S} \textbf{E}(\textbf{r})\cdot d\textbf{S}
    = q\oint\limits_{S}
    \frac{\textbf{r}-\textbf{r}'}{|\textbf{r}-\textbf{r}|^{3}}\cdot d\textbf{S}
    = q\oint\limits_{S}
    \frac{1}{|\textbf{r}-\textbf{r}|^{2}}
    \frac{\textbf{r}-\textbf{r}'}{|\textbf{r}-\textbf{r}|} \cdot \hat{\textbf{n}}\,dS
    = q\oint\limits_{S}
    \frac{1}{|\textbf{r}-\textbf{r}|^{2}}
    \hat{\textbf{r}} \cdot \hat{\textbf{n}}\,dS
    = \oint \frac{q\cos{(\theta)}}{|\textbf{r}-\textbf{r}'|^{2}}dS
\end{equation*}
donde cambié $\hat{\textbf{r}}\cdot \textbf{n}$ por $\cos{(\theta)}$. Resulta que esa integral se puede escribir como la integral en ángulo sólido, ya que $\cos{(\theta)}dS = r^{2}d\Omega$, donde $r^{2} = |\textbf{r}-\textbf{r}'|^{2}$. Con lo cual, se llega a
\begin{equation*}
    \phi = q\oint\limits_{S}d\Omega
\end{equation*}
Entonces, si la carga está dentro de la superficie $S$, el ángulo sólido subtendido por la superficie es de $4\pi$ y si la carga está fuera del sólido, el ángulo sólido subtendido es 0. Es decir, en general vale que $\phi = 4\pi Q_{enc}$, con $Q_{enc}$ la carga encerrada por la superficie.\\
\indent Si considero el caso general de la carga encerrada como $Q_{enc} = \int \rho(\textbf{r}')dV'$ y vuelvo a la expresión del flujo de campo eléctrico, tengo
\begin{equation*}
    \oint \textbf{E}(\textbf{r})\,d\textbf{S}
    = 4\pi \int \rho(\textbf{r}')dV'
\end{equation*}
Si a esta se le aplica el teorema de la divergencia de Gauss, la integral de flujo se transforma en una integral en volumen de la divergencia del campo eléctrico. Como ahora tengo ambas integrales en volumen, y como tiene que valer para todo volumen $V$, necesariamente se obtiene que
\begin{equation*}
    4\pi\rho = \nabla \cdot \textbf{E}
\end{equation*}
que es la forma diferencial de la ecuación de Gauss y una de las 4 ecuaciones de Maxwell.
%fin de clase 1


%%%%%%%%%%%%%%%%%%%%%%%%%%%%%%%%%%%%%%%%%%%%%%%%
%%%%%%%%%%%%%%%%%%%%%%%%%%%%%%%%%%%%%%%%%%%%%%%%


\subsubsection{Potencial electrostático}
A partir de la expresión general del campo eléctrico \eqref{ec:CampoElectrico}, se puede obtener la expresión general de potencial electrostático. Para eso, se puede empezar observando que vale lo siguiente
\begin{equation}
    \frac{\textbf{r}-\textbf{r}'}{|\textbf{r}-\textbf{r}'|^{3}}
    = -\nabla_{\textbf{r}}
    \left(
        \frac{1}{|\textbf{r}-\textbf{r}'|}
    \right)
    = \nabla_{\textbf{r}'}
    \left(
        \frac{1}{|\textbf{r}-\textbf{r}'|}
    \right)
        \label{ec:IdentidadPotencialCampo}
\end{equation}
donde el subíndice $\textbf{r}$ o $\textbf{r}'$ indica respecto de cuál coordenada se está haciendo el gradiente. Con lo cual, se puede reemplazar en la expresión del campo eléctrico y se tiene que
\begin{equation*}
    \textbf{E}(\textbf{r}) =
    \int \rho(\textbf{r}')
    \frac{\textbf{r}-\textbf{r}'}{|\textbf{r}-\textbf{r}'|^{3}}
    d^{3}\textbf{r}'
    = 
    -\int \rho(\textbf{r}')
    \nabla_{\textbf{r}}
    \left(
        \frac{1}{|\textbf{r}-\textbf{r}'|}
    \right)
    d^{3}\textbf{r}'
\end{equation*}
como la integral es respecto a los puntos fuente ($\textbf{r}'$), puedo sacar el gradiente respecto a los puntos campo ($\textbf{r}$) fuera de la integral, entonces
\begin{equation*}
    \textbf{E}(\textbf{r}) =
    -\nabla_{\textbf{r}}
    \int \frac{\rho(\textbf{r}')}{|\textbf{r}-\textbf{r}'|}
    d^{3}\textbf{r}'
\end{equation*}
entonces se llega a que el campo eléctrico se puede escribir como el gradiente de otra cantidad. Esta otra cantidad es el potencial electrostático y, además, indica que el campo eléctrico es un campo conservativo\footnote{Todo campo que proviene de un gradiente es un campo conservativo}, entonces se define el potencial como
\begin{equation}
    \varphi(\textbf{r}) = 
    -\int \frac{\rho(\textbf{r}')}{|\textbf{r}-\textbf{r}'|}
    d^{3}\textbf{r}'
        \label{ec:PotencialElectrostatico}
\end{equation}
Usando este resultado y la Ley de Gauss ($\nabla\cdot\textbf{E} = 4\pi \rho$), se obtiene una ecuación de Poisson\footnote{La divergencia del gradiente es el Laplaciano} para el potencial $\varphi$, es decir
\begin{equation}
    \nabla^{2}\varphi = -4\pi\rho
        \label{ec:Poisson}
\end{equation}
Para el caso particular donde $\rho = 0$, se recupera una ecuación de Laplace para $\varphi$, o sea $\nabla^{2}\varphi = 0$. Esto es útil porque la ecuación de Laplace tiene soluciones, de forma que se puede determinar completamente el potencial electrostático.\\
\indent La interpretación física del (la diferencia de)potencial electrostático es que es el trabajo necesario para llevar una carga $q$ desde un dado punto $A$ en presencia de un campo eléctrico, a un punto $B$.


%%%%%%%%%%%%%%%%%%%%%%%%%%%%%%%%%%%%%%%%%%%%%%%%
%%%%%%%%%%%%%%%%%%%%%%%%%%%%%%%%%%%%%%%%%%%%%%%%


\subsubsection{Desarrollo Multipolar}
Tomo una distribución de carga totalmente arbitraria $\rho$, con un tamaño característico $L$, y tomo un punto campo $\textbf{r}$ lejano a la distribución, es decir $r = |\textbf{r}| >> L $. Quiero ver el potencial electrostático de la distribución en el punto $\textbf{r}$ lejano. Como $r >> L$, y todos los puntos campo $\textbf{r}'$ son en módulo menores a $L$\footnote{Solo si la distribución está centrada en el origen}, entonces se puede hacer Taylor alrededor de $\textbf{r}' = 0$. Entonces
\begin{equation*}
    \frac{1}{|\textbf{r}-\textbf{r}'|}
    =
    \frac{1}{r} 
    + \partial_{i}'
        \left(
            \frac{1}{|\textbf{r}-\textbf{r}|}
        \right)_{\textbf{r}' = 0}r_{i}'
    + \frac{1}{2} \partial_{i}'\partial_{j}'
        \left(
            \frac{1}{|\textbf{r}-\textbf{r}|}
        \right)_{\textbf{r}' = 0}r_{i}'r_{j}'
    +
    \cdots
\end{equation*}
haciendo el Taylor y evaluando en $\textbf{r}' = 0$ se llega\footnote{Ver apéndice 4.1} se llega a 

\begin{equation}
    \frac{1}{|\textbf{r}-\textbf{r}'|}
    \approx 
    \frac{1}{r}
    +
    \frac{r_{i}r_{i}'}{r^{3}}
    +
    \frac{1}{2}
    \frac{3r_{i}r_{j} - r^{2}\delta_{ij}}{r^{5}}
    r_{i}'r_{j}'
        \label{ec:DesarrolloMultipolarCartesianas}
\end{equation}
de esta forma, reemplazando en la expresión del potencial, se tiene que
\begin{equation*}
    \varphi(\textbf{r}) =
    \int
    \rho(\textbf{r}')
        \left\{
            \frac{1}{r}
            +
            \frac{r_{i}r_{i}'}{r^{3}}
            +
            \frac{1}{2}
            \frac{3r_{i}r_{j} - r^{2}\delta_{ij}}{r^{5}}
            r_{i}'r_{j}'
            +
            \cdots
        \right\}
    d^{3}\textbf{r}'
\end{equation*}
y es el potencial electrostático de una distribución de carga arbitraria, centrada en el origen de coordenadas, vista desde una posición $\textbf{r}$ muy lejana (la posición es mucho mayor que el tamaño característico de la distribución de carga).\\
\indent Se puede observar cada término del desarrollo por separado y ver la implicación física de cada uno
\begin{equation*}
    \varphi(\textbf{r}) = 
    \varphi^{(0)}(\textbf{r}) + \varphi^{(1)}(\textbf{r}) + \varphi^{(2)}(\textbf{r}) + \cdots
\end{equation*}
miro el término \textbf{monopolar}:
\begin{equation*}
    \varphi^{(0)}(\textbf{r}) 
    = \frac{1}{r}\int \rho(\textbf{r}')\,d^{3}\textbf{r}'
    = \frac{Q}{r}
\end{equation*}
cae como $1/r$ y lo que se ve es el potencial de una carga puntual $Q$! Lo que significa que una distribución de carga arbitraria, se ve como una carga puntual si se la mira lo suficientemente lejos, dado que los demás términos caen mucho más rápido, ya que son potencias superiores de la coordenada $r$.\\
\indent Miro ahora el término \textbf{dipolar}:
\begin{equation*}
    \varphi^{(1)}(\textbf{r})
    = \frac{r_{i}}{r^{3}}\int \rho(\textbf{r}')r_{i}'\,d^{3}\textbf{r}'
    \equiv \frac{\textbf{r}\cdot\textbf{p}}{r^{3}}
\end{equation*}
con $\textbf{p} = \int \rho(\textbf{r}')\textbf{r}'\,d^{3}\textbf{r}'$, el momento dipolar eléctrico de la carga. Entonces, si supongo el caso en el que la distribución de carga es tal que no tiene carga neta, es decir $Q = 0$ (las contribuciones de carga positiva y carga negativa se anulan exactamente), entonces sólo sobrevive el término dipolar y muy lejos de la distribución se va a ver una dirección privilegiada en el espacio, el potencial ya no decae de forma uniforme. Por otro lado, si hay carga neta $Q$, entonces muy lejos de la distribución se vuelve a observar el potencial de una carga puntual, dado que el término dipolar cae como $r^{3}$, 2 órdenes de magnitud más rápido que el término monopolar que cae como $r$.\\
\indent Otra observación para el momento dipolar $\textbf{p}$ es que este depende del origen de coordenadas, salvo que la carga neta $Q$ sea cero\footnote{Ver apéndice 4.4}.\\
\indent Ahora se plantea la siguiente pregunta: ¿Existe una \textit{carga dipolar puntual}, tal que el potencial que genere sea el potencial del dipolo? La respuesta es que, matemáticamente sí, y queremos hallarla. Para eso lo que pido es que la carga neta total sea cero: $Q = 0$, de modo que $\varphi^{(0)} = \varphi^{(2)} = 0$ y solo sobrevive $\varphi^{(1)}$. Entonces lo que se hace es tomar dos cargas puntuales $q$ y $-q$, y se hace tender la distancia $d$ que las separa a cero. Como la distancia tiende a cero, tengo que pedir que la distribución de carga $\rho$ (O sea $q$ y $-q$) tienda a infinito, de forma que $\textbf{p}$ sea finito
\begin{equation*}
    \left\{
        \begin{matrix}
            \rho \longrightarrow \infty\\
            d \longrightarrow 0\\
            Q = 0
        \end{matrix}
    \right.
    \Longrightarrow \rho d \to p\quad \mbox{finito}
\end{equation*}
entonces, tomo
\begin{equation*}
    \varphi(\textbf{r})=
    \lim_{\substack
        {
            q\to\infty\\
            \varepsilon = d\to 0\\
            qd\to p
        }
    }
    \frac{q}{|\textbf{r}-(\textbf{r}'+\varepsilon \hat{p})|}
    -
    \frac{q}{|\textbf{r}-\textbf{r}'|}
    =
    \cdots\footnote{Ver apéndice 4.4.1}
    =
    \frac{\textbf{p}\cdot(\textbf{r}-\textbf{r}')}{|\textbf{r}-\textbf{r}'|^{3}}
\end{equation*}
y ahora cuál sería una densidad de carga puntual correspondiente a un dipolo? tomo
\begin{equation*}
    \rho_{p}(\textbf{r})
    =
    \lim_{\substack
        {
            q \to \infty\\
            \varepsilon = d \to 0\\
            qd \to p
        }
    }
    q\delta
    \left(
        \textbf{r}-(\textbf{r}'+\varepsilon \hat{p})
    \right)
    -q \delta(\textbf{r}-\textbf{r}') 
    = \cdots \footnote{Ver apéndice 4.4.1}
    = -\left( \textbf{p}\cdot \nabla\right)\delta(\textbf{r}-\textbf{r}')
\end{equation*}
Esto es una expresión para la densidad de carga de un dipolo puntual.\\
\indent Si tomo el caso particular en que la densidad de polarización se escribe como $\textbf{P} = \textbf{p}\,\delta(\textbf{r}-\textbf{r}')$, es decir, no hay momento dipolar en ningún lado a excepción de donde está el dipolo $\textbf{p}$, entonces si reemplazo en la otra expresión, se llega a que la densidad de carga de polarización es
\begin{equation}
    \rho_{\textbf{p}}(\textbf{r}) = -\nabla \cdot \textbf{P}
        \label{ec:DensidadeCargadePolarizacion}
\end{equation}
Esto se obtuvo formalmente solo usando la expresión de la fuerza de Coulomb y el formalismo de distribuciones, es consistente con la fuerza de Coulomb que la densidad de carga de polarización sea la menos la divergencia de la densidad de polarización del medio.\\
\indent Ahora miro el término \textbf{cuadrupolar:}
\begin{equation*}
    \varphi^{(2)}(\textbf{r})
    = 
    \frac{1}{2}
    \left(
        \frac{3r_{i}r_{j}- \delta_{ij}r^{2}}{r^{5}}
    \right)C_{ij},
    \quad
    \quad
    \mbox{con}
    \quad
    C_{ij} = \int \rho(\textbf{r}')r_{i}'r_{j}'\,d^{3}\textbf{r}
\end{equation*}
pero esto no es el término cuadrupolar eléctrico, hay que reordenar un poco los términos, dado que en $C_{ij}$, que es un tensor, hay muchos términos que no son independientes. Pero por qué esta expresión no es el término cuadrupolar? la respuesta es que esta expansión en Taylor se hizo en el sistema de coordenadas cartesiano, que para en este caso no es el sistema de coordenadas natural del sistema. El sistema de coordenadas natural para este problema es esféricas, dada la simetría del sistema al ser visto desde muy lejos. De esta forma, al expandir en Taylor en coordenadas cartesianas, lo que pasa es que aparecen muchos términos extra que no tienen que ver con el término cuadruplar en sí mismo. Para solucionar esto hay que reordenar los términos que se tienen hasta ahora.\\
\indent Tomando $(3r_{i}r_{j} - \delta_{ij}r^{2})C_{ij}$ y reescribiéndolo\footnote{Ver apéndice 4.10} se llega a
\begin{equation*}
    (3r_{i}r_{j} - \delta_{ij}r^{2})C_{ij} 
    = r_{i}r_{j}(3 C_{ij} - \delta_{ij}C_{ll})
\end{equation*}
donde defino al tensor momento cuadrupolar como
\begin{equation*}
    Q_{ij} = 3 C_{ij} - C_{ll}\delta_{ij}
\end{equation*}
y reemplazando por lo que vale $C_{ij}$ se llega a \footnote{Ver apéndice 4.10}
\begin{equation}
    Q_{ij} = 
    \int \rho(\textbf{r}') (3r_{i}'r_{j}' - \delta_{ij}r^{2})\,d^{3}\textbf{r}
        \label{ec:TérminoCuadrupolar}
\end{equation}
finalmente, metiendo todo esto en la expresión de $\varphi^{(2)}$
\begin{equation*}
    \varphi^{(2)} = \frac{1}{2}
    \frac{r_{i}r_{j}}{r^{5}}Q_{ij}
\end{equation*}
el momento cuadrupolar eléctrico representa las deformaciones, a lo largo de los 3 ejes principales de la distribución de carga, en la forma del potencial eléctrico. De esta última se observa que para $r$ muy grande, este término cae como $r^{3}$.\\
\indent Observar también que la traza de $Q_{ij}$ es cero y además $Q_{ij}$ es simétrico. Es un tensor simétrico de traza nula.\\
\indent Por último, si se sigue la expansión del potencial con Taylor en este sistema de coordendas, siguen apareciendo más términos dependientes entre sí debido a la elección del sistema de coordenadas. Con lo cual lo mejor va a ser realizar la expansión en coordenadas esféricas más adelante, donde van a aparecer los armónicos esféricos.


%%%%%%%%%%%%%%%%%%%%%%%%%%%%%%%%%%%%%%%%%%%%%%%%
%%%%%%%%%%%%%%%%%%%%%%%%%%%%%%%%%%%%%%%%%%%%%%%%


\subsubsection{Medios materiales}
En general hay dos tipos de medios materiales: Los conductores y los dieléctricos.\\ 
\indent Los conductores son medios materiales tales que, en presencia de un campo eléctrico, sus portadores de carga pueden desplazarse distancias macroscópicas y reacomodarse en la superficie, forma una densidad de carga superficial $\sigma$, de forma tal que el campo eléctrico en el interior del conductor se anule completamente.\\
\indent Un dieléctrico, en cambio, es un medio material en el cual sus portadores de carga no están libres y no pueden desplazarse, a lo sumo reacomodar su orientación formado dipolos en el interior del material y generando así polarización neta del medio.\\
\indent Volviendo a la ecuación de Maxwell para la ley de Gauss, pero ahora en el contexto de medios materiales, se va a tener que
\begin{equation*}
    \nabla \cdot \textbf{E}
    = 4\pi (\rho_{l} + \rho_{p})
\end{equation*}
es decir, sigue valiendo que la divergencia del campo eléctrico es proporcional a la densidad de carga, pero ahora se distinguen dos tipos de densidades de carga: La densidad de carga libre $\rho_{l}$ y la densidad de carga de polarización $\rho_{p}$ (la cual se logró escribir como una divergencia en \eqref{ec:DensidadeCargadePolarizacion}). Entonces, reemplazando por \eqref{ec:DensidadeCargadePolarizacion} y despejando se llega a dos expresiones importantes, una es una expresión para la densidad de carga libre $\rho_{l}$ y la otra es la definición del vector de desplazamiento eléctrico $\textbf{D}$:
\begin{equation}
    \nabla \cdot (\textbf{E} + 4\pi \textbf{P}) = 
    \nabla \cdot \textbf{D}
    = 4\pi \rho_{l}
        \label{ec:DensidaddeCargaLibre}
\end{equation}
y 
\begin{equation}
    \textbf{D} \equiv \textbf{E} + 4\pi \textbf{P}
        \label{ec:VectorDeplazamientoElectrico}
\end{equation}
Ahora se puede ver como queda la expresión del rotor del vector desplazamiento, y es simplemente
\begin{equation*}
    \nabla \times \textbf{D} = 
    4\pi \nabla \times \textbf{P}
\end{equation*}
ya que $\nabla \times \textbf{E} = 0$.\\
\indent Con estas nuevas expresiones ahora es necesario, para resolver las ecuaciones de Maxwell, hallar las \textit{relaciones constitutivas del material}, es decir, como se polariza el medio en presencia del campo. Entonces para ver la respuesta del medio, lo que se hace es ver la relación entre $\textbf{E}$ y $\textbf{P}$, o sea, cómo se polariza el medio en presencia del campo. Escribiendo un desarrollo se tiene
\begin{equation*}
    \textbf{P}(\textbf{E}) =
    \textbf{P}(0) + 
    \left.
        \frac{\partial\textbf{P}}{\partial E_{i}}
    \right)_{\textbf{E} = 0}E_{i} +
    \left.
        \frac{\partial^{2}\textbf{P}}{\partial E_{i}\partial E_{j}}
    \right)_{\textbf{E} = 0}E_{i}E_{j} +
    \cdots
\end{equation*}
donde los coeficientes (las derivadas evaluadas en $\textbf{E}= 0$) representan características intrínsecas del medio, por ejemplo cuando $\textbf{P}(0) \neq 0$ el medio es un electrete, es decir, presenta polarización aún cuando no está en presencia de un campo eléctrico. En general, $P(0) = 0$ y el medio puede modelarse según
\begin{equation*}
    P_{i} = \chi_{ij}E_{j}
\end{equation*}
donde $\chi_{ij}$ es el \textbf{tensor de susceptibilidad eléctrica}. En el caso más simple de un medio isótropo (no hay dirección privilegiada), es la identidad por una constante, es decir, una constante. En otros casos pueden existir direcciones del espacio privilegiadas (casos de medios birrefrigentes? quizás?). Para el caso del medio isótropo entonces se termina teniendo\footnote{Ver apéndice 5.3}
\begin{equation*}
    \textbf{P} = \chi \textbf{E},
    \quad
    \quad
    \textbf{D} = (1 + 4\pi\chi)\textbf{E} = \varepsilon \textbf{E}
\end{equation*}


%%%%%%%%%%%%%%%%%%%%%%%%%%%%%%%%%%%%%%%%%%%%%%%%
%%%%%%%%%%%%%%%%%%%%%%%%%%%%%%%%%%%%%%%%%%%%%%%%


\subsubsection{Condiciones de contorno para el campo eléctrico en interfaces}
Supongo una superficie que cumple el rol de interfaz y separa dos regiones del espacio, una región 1 y una region 2. La normal a la superficie de la interfaz será $\hat{\textbf{n}}$. Supongo un pequeño cilindro imaginario de volumen $V$ y longitud $\delta$ que atraviesa la superficie, de forma que una parte del volumen del cilindro está en contacto con la región 1 y la otra con la región 2. Aplicando la ley de Gauss al cilindro, se obtienen las relaciones de las condiciones de contorno\footnote{ver apéndice 5.5 hora 1:34 del video de la clase 2}
\begin{equation}
    \left\{
        \begin{matrix}
            (\textbf{D}_{2} - \textbf{D}_{1})\cdot \hat{\textbf{n}} & = 4\pi\sigma\\
            & \\
            \hat{\textbf{n}}\times (\textbf{E}_{2}-\textbf{E}_{1}) & = 0
        \end{matrix}
    \right.
    \label{ec:CondicionesdeContornoE}
\end{equation}
Esta expresión dice que el campo eléctrico transversal a la superficie tiene un salto igual a $4\pi \sigma$, mientras que el campo eléctrico tangencial a la superficie es siempre continuo.

%%%%%%%%%%%%%%%%%%%%%%%%%%%%%%%%%%%%%%%%%%%%%%%%
%%%%%%%%%%%%%%%%%%%%%%%%%%%%%%%%%%%%%%%%%%%%%%%%


\subsection{Magnetostática}
Experimentalmente se determinó que la fuerza que siente un \textit{circuito} $1$, por el que circula una corriente $I_{1}$ debido a un \textit{circuito} $2$ por el que circula una corriente $I_{2}$ es de la forma
\begin{equation}
    \textbf{F}_{12} = \frac{1}{c^{2}}
    \oint I_{1}\,d\textbf{l}_{1}\times 
    \oint I_{2}\,d\textbf{l}_{2}\times 
    \frac{\textbf{r}_{1}-\textbf{r}_{2}}{|\textbf{r}1-\textbf{r}_{2}|^{3}}
        \label{ec:LeydeAmpere}
\end{equation}
Notar que es una expresión global, no está escrita para elemento diferencial como lo estaba ley de Coulomb ($I\,d\textbf{l}$ no es equivalente a $q$). Además, esta satisface el principio de acción y reacción débil.\\
\indent Puedo reescribir la expresión en función del campo magnético de una de las espiras (o circuitos)
\begin{equation*}
    \textbf{F}_{1} = \frac{1}{c}\oint I_{1}\,d\textbf{l}_{1}\times \textbf{B}_{2}(\textbf{r}_{1})
\end{equation*}
de esta forma
\begin{equation}
    \textbf{B}_{2}(\textbf{r}_{1})
    = \frac{1}{c}\oint I_{2}\,d\textbf{l}_{2}\times 
    \frac{\textbf{r}_{1}-\textbf{r}_{2}}{|\textbf{r}_{1}-\textbf{r}_{2}|^{3}}
        \label{ec:CampoMagnético}
\end{equation}
y en general, si se tiene una densidad volumétrica de corriente $\textbf{J}(\textbf{r})$, el campo magnético se escribe como
\begin{equation}
    \textbf{B}(\textbf{r})
    = \frac{1}{c}\int \textbf{J}(\textbf{r}')\times 
    \frac{\textbf{r}-\textbf{r}'}{|\textbf{r}-\textbf{r}'|^{3}} dV'
        \label{ec:CampoMagnéticoGeneral}
\end{equation}

%%%%%%%%%%%%%%%%%%%%%%%%%%%%%%%%%%%%%%%%%%%%%%%%
%%%%%%%%%%%%%%%%%%%%%%%%%%%%%%%%%%%%%%%%%%%%%%%%

\subsubsection{Ley de Ampere}
Se parte de tomar el rotor del campo magnético $\textbf{B}$, respecto a las coordenadas campo
\begin{equation*}
    \nabla \times \textbf{B} = 
     \frac{1}{c} \nabla \times 
     \left(
        \int \textbf{J}(\textbf{r}')
        \times \frac{\textbf{r}-\textbf{r}'}{|\textbf{r}-\textbf{r}'|^{3}}
        \,dV'
     \right)
\end{equation*}
esta se puede reescribir usando la identidad \eqref{ec:IdentidadPotencialCampo} y queda expresada en función de un \textit{potencial} $\phi$. Entonces la cuenta que hay que hacer es el rotor de la densidad de corriente $\textbf{J}$ y el rotor del gradiente del potencial $\phi$. Haciendo las cuentas \footnote{Ver apéndice 5.9} se llega a
\begin{equation*}
    \left[
        \nabla \times 
        \left(
            \textbf{J}'\times \nabla \phi
        \right)
    \right]_{i} 
    = \textbf{J}'_{i} \nabla^{2}\phi
\end{equation*}
reemplazando en la expresión del rotor del campo magnético se llega
\begin{equation*}
    \nabla \times \textbf{B} = -\frac{1}{c}
    \int \textbf{J}(\textbf{r}')\,
    \nabla^{2}
    \left(
        \frac{1}{|\textbf{r}-\textbf{r}'|}
    \right)\,
    dV'
\end{equation*}
Observación: Si parto de la expresión para la divergencia del campo eléctrico de una carga puntual, en particular si la carga es $q = 1$, se puede llegar a que\footnote{Ver apéndice 5.9} $-\nabla^{2}\left(\frac{1}{|\textbf{r}-\textbf{r}'|}\right) = 4\pi \delta(\textbf{r}-\textbf{r}')$, reemplazando e integrando se tiene que
\begin{equation*}
    \nabla \times \textbf{B} = 
    \frac{4\pi}{c} \textbf{J}(\textbf{r})
\end{equation*}

%%%%%%%%%%%%%%%%%%%%%%%%%%%%%%%%%%%%%%%%%%%%%%%%
%%%%%%%%%%%%%%%%%%%%%%%%%%%%%%%%%%%%%%%%%%%%%%%%

\subsubsection{Potencial Vector}
Parto de la expresión del campo magnético que tiene dentro de la integral la identidad \eqref{ec:IdentidadPotencialCampo}, es decir
\begin{equation*}
    \textbf{B}(\textbf{r})
    = \frac{1}{c}\int
    \textbf{J}(\textbf{r}')\times
    \left(
        -\nabla_{\textbf{r}}\frac{1}{|\textbf{r}-\textbf{r}'|}
    \right)\,dV'
\end{equation*}
como $\nabla_{\textbf{r}}$ opera sobre las variables no primadas (coordenadas de campo), lo puedo sacar fuera de la integra y, dado el producto vectorial entre la corriente y la magnitud entre paréntesis, se transforma en un rotor, es decir
\begin{equation*}
    \textbf{B}(\textbf{r}) = 
    \nabla_{\textbf{r}}\times
    \left(
        \frac{1}{c}
        \int
        \frac{\textbf{J}(\textbf{r}')}{|\textbf{r}-\textbf{r}'|}\,dV'
    \right)
\end{equation*}
y de donde se define a la magnitud entre paréntesis como el potencial vector $\textbf{A}(\textbf{r})$, es decir
\begin{equation}
    \textbf{A}(\textbf{r}) =
    \frac{1}{c}
    \int
    \frac{\textbf{J}(\textbf{r}')}{|\textbf{r}-\textbf{r}'|}\,dV'
    + \nabla \chi,
    \quad
    \quad
    \Longrightarrow
    \textbf{B} = \nabla \times \textbf{A}
        \label{ec:PotencialVector}
\end{equation}
donde se agrega el factor $\nabla \chi$ ya que el potencial vector queda definido a menos de un gradiente (al aplicar rotor, este muere y se recupera el campo magnético), es decir, tiene libertad de gauge. Tiene que seguir valiendo $\nabla \cdot \textbf{B} = 0$. Por último, usando la ley de ampere, ahora en función de \textbf{A}, y eligiendo el gauge de Coulomb, que es pedir que $\nabla \cdot \textbf{A} = 0$ y que $\chi = 0$, se tiene que\footnote{ver apéndice 1.8}
\begin{equation*}
    \nabla \times \textbf{B} 
    = \nabla \times \nabla \times \textbf{A} 
    = \cancel{\nabla (\nabla \cdot \textbf{A})} - \nabla^{2}\textbf{A}
    = \frac{4\pi}{c}\textbf{J}
\end{equation*}
y se tiene una ecuación de Poisson para el potencial vector $\textbf{A}$
\begin{equation}
    \nabla^{2}\textbf{A} = -\frac{4\pi}{c}\textbf{J}
        \label{ec:PoissonPotencialVector}
\end{equation}
con lo cual, tanto en magnetostática como electrostática, si se puede resolver el problema de Poisson, se puede resolver todo el problema.

%%%%%%%%%%%%%%%%%%%%%%%%%%%%%%%%%%%%%%%%%%%%%%%%
%%%%%%%%%%%%%%%%%%%%%%%%%%%%%%%%%%%%%%%%%%%%%%%%


\subsubsection{Desarrollo multipolar del potencial vector}
Consideramos una densidad de corriente en volumen $\textbf{J}$ de tamaño característico $L$ y queremos ver como se ve el potencial vector $\textbf{A}(\textbf{r})$ para una distancia $r >> L$, considerando a la densidad de corriente en el origen. Entonces, como $L<<r$, se puede hacer Taylor al potencial vector alrededor de $\textbf{r}' = 0$ (ya que está centrada en el origen) de la misma forma que se hizo con el potencial electrostático, usando el desarrollo \eqref{ec:DesarrolloMultipolarCartesianas} y metiéndolo en la expresión del potencial vector $\textbf{A}$ da
\begin{equation*}
    \textbf{A}(\textbf{r}) =
    \frac{1}{cr}\int \textbf{J}(\textbf{r}')d^{3}\textbf{r}'
    +
    \frac{1}{cr^{3}}\int (\textbf{r}\cdot \textbf{r}') \textbf{J}(\textbf{r}')d^{3}\textbf{r}'
    +
    \cdots
\end{equation*}
donde se puede ver que el primer término es siempre nulo\footnote{Ver apéndice 7.1} y corresponde al término monopolar magnético. Teniendo en cuenta que nunca fueron observados en la naturaleza los monopolos magnéticos, tiene sentido que este término sea siempre nulo. Viendo el primer término no nulo, escrito en notación de índices queda como
\begin{equation*}
    A_{i}^{(1)} = \frac{1}{cr^{3}}\int r_{l}r_{l}'J_{i}'\,dV'
\end{equation*}
y se puede demostrar que vale $\int r_{l}'J_{i}'\,dV' = -\int r_{i}'J_{l}'\,dV'$, lo cual significa que es una matriz antisimétrica y entonces tiene traza nula. Y esto es importante porque porque al tener en tres dimensiones una matriz antisimétrica, significa que sólo tengo 3 valores independientes (de los 9 en total) y además, con estos 3 valores independientes se puede armar un vector o pseudovector. Por qué un pseudovector? porque no transforma como vector ante reflexiones, sino que cambia el signo de sus componentes. En definitiva, un pseudovector es aquella magnitud vectorial que se obtiene a partir de las 3 componentes independientes de un tensor antisimétrico de traza nula.\\
\indent Finalmente, el primer término no nulo $\textbf{A}^{(1)}$ se puede reescribir\footnote{Ver apéndice HACER APÉNDICE, VER CUENTAS} como
\begin{equation*}
    \textbf{A}^{(1)}(\textbf{r}) = 
    \frac{\textbf{m}\times \textbf{r}}{r^{3}},
    \quad
    \quad
    \mbox{con}
    \quad
    \textbf{m} 
    = \frac{1}{2c}\int \textbf{r}'\times \textbf{J}'d^{3}\textbf{r}'
\end{equation*}
donde $\textbf{m}$ es el momento dipolar magnético. Se define el vector de magnetización $\textbf{M}(\textbf{r}) = \frac{d\textbf{m}}{dV}$.

%%%%%%%%%%%%%%%%%%%%%%%%%%%%%%%%%%%%%%%%%%%%%%%%
%%%%%%%%%%%%%%%%%%%%%%%%%%%%%%%%%%%%%%%%%%%%%%%%


\subsubsection{Campo magnético en medios materiales}
Se quiere modelar el comportamiento de los campos magnéticos en presencia de medios materiales. Análogamente al caso de electrostática, un medio material en presencia de un campo magnético sufrirá modificaciones en la distribución de corrientes en su interior. Ahora la corriente total del sistema $\textbf{J}$ se descompone en dos contribuciones: las corrientes libres $\textbf{J}_{l}$ y las corrientes de magnetización del medio $\textbf{J}_{m}$. Estas últimas se puede escribir como
\begin{equation*}
    \textbf{J}_{m} = c\nabla \times \textbf{M}
\end{equation*}
usando la ley de Ampere y escribiendo $\textbf{J} = \textbf{J}_{l} + \textbf{J}_{m}$ y despejando, se puede llegar a 
\begin{equation*}
    \nabla \times (\textbf{B} - 4\pi \textbf{M}) = \frac{4\pi}{c}\textbf{J}_{l}
\end{equation*}
que es una expresión para las corrientes libres sobre el medio el medio material. A su vez, de esta se define el vector \textbf{Intensidad Magnética} $\textbf{H}$ como
\begin{equation*}
    \textbf{H} = \textbf{B} - 4\pi \textbf{M}
\end{equation*}
de forma que
\begin{equation*}
    \nabla \times \textbf{H} = \frac{4\pi}{c}\textbf{J}_{l}
\end{equation*}
Nuevamente es necesario conocer las relaciones constitutivas del material, es decir, una relación entre la magnetización y el campo magnético: $\textbf{M} = \textbf{M}(\textbf{H})$. Cuánto se magnetiza el medio en función del campo magnético externo aplicado. Entonces, si el medio es lineal, se tiene
\begin{equation*}
    M_{j} = \chi_{ij}^{m}H_{i}
\end{equation*}
donde $\chi_{ij}^{m}$ es el tensor de permeabilidad magnética. Para el caso isótropo, el tensor no es más que una constante por la identidad, es decir, una constante y resulta $\textbf{M} = \chi_{m} \textbf{H}$. Y existen dos tipos de medios, dependiendo del valor de la constante de permeabilidad magnética: 
\begin{equation*}
    \left\{
        \begin{matrix}
            \chi_{m} > 0 \quad \mbox{Paramagnéticos}\\
            \\
            \chi_{m} < 0 \quad \mbox{Diamagnéticos}
        \end{matrix}
    \right.
\end{equation*}
Donde los medios paramagnéticos son aquellos donde los dipolos magnéticos se alinean de forma paralela al campo magnético aplicado (igual que el ferromagnético?), mientras que los medios diamagnéticos son aquellos donde los dipolos magnéticos se alinean antiparalelamente al campo magnético aplicado (como antiferromagnético?).\\
\indent Por último 
\begin{equation*}
    \textbf{B} = \textbf{H} + 4\pi \textbf{M} \equiv \mu \textbf{H}
\end{equation*}
%%%%%%%%%%%%%%%%%%%%%%%%%%%%%%%%%%%%%%%%%%%%%%%%
%%%%%%%%%%%%%%%%%%%%%%%%%%%%%%%%%%%%%%%%%%%%%%%%

\subsubsection{Condiciones de contorno para el campo magnético en interfaces}
Análogas a las de electrostática, son \footnote{Ver apéndice 7.9}
\begin{equation}
    \left\{
        \begin{matrix}
            (\textbf{B}_{2} - \textbf{B}_{1})\cdot \hat{\textbf{n}} & = 0\\
            & \\
            \hat{\textbf{n}}\times (\textbf{H}_{2}-\textbf{H}_{1}) & = \frac{4\pi}{c}\textbf{K}_{l}
        \end{matrix}
    \right.
    \label{ec:CondicionesdeContornoB}
\end{equation}
con $\textbf{K}_{l}$ la densidad de corriente en la superficie del medio. La primera corresponde a la continuidad normal del campo magnético en la interfaz, sin embargo, en la componente transversal hay un salto en el campo magnético proporcional a la corriente inducida en superficie.

\subsection{Resumen}
Tanto en electrostática como en magnetostática se llegó a las expresiones de los campos en función de potenciales que cumplen la ecuación de Poisson. Es decir que, si es posible resolver la ecuación de Poisson, entonces se pueden hallar los potenciales y con ellos los campos. El desafío es aprender a resolver estas ecuaciones.


%%%%%%%%%%%%%%%%%%%%%%%%%%%%%%%%%%%%%%%%%%%%%%%%
%%%%%%%%%%%%%%%%%%%%%%%%%%%%%%%%%%%%%%%%%%%%%%%%
%%%%%%%%%%%%%%%%%%%%%%%%%%%%%%%%%%%%%%%%%%%%%%%%

\newpage
\section{Tratamiento teórico del potencial electrostático}
Se quiere estudiar el problema electromagnético (estático) para el caso en que se tienen dominios finitos y condiciones de contorno dadas, es decir, no nos interesa resolver el problema en todo el espacio. Dadas estas condiciones, se quiere probar que existe solución única para el problema de Poisson.


%%%%%%%%%%%%%%%%%%%%%%%%%%%%%%%%%%%%%%%%%%%%%%%%
%%%%%%%%%%%%%%%%%%%%%%%%%%%%%%%%%%%%%%%%%%%%%%%%


\subsection{Teorema de Green}
Consideremos un recinto cerrado de volumen $V$, con una superficie $S$ de normal externa $\hat{\textbf{n}}$. Dado un campo vectorial arbitrario $\textbf{A}$, sabemos que vale el teorema de la divergencia
\begin{equation*}
    \int\limits_{V}\nabla\cdot\textbf{A}\,dV =
    \int\limits_{S}\textbf{A}\cdot d\textbf{S}
\end{equation*}
Lo que se va a hacer es escribir las identidades de Green y luego usarlas convenientemente con los campos para ver como se escribe un dado potencial en función de las condiciones de contorno.\\
\indent Entonces, si tomo $\textbf{A} = \phi \nabla \psi$ y reemplazo en el teorema de la divergencia se llega a la primera identidad de Green\footnote{Ver apéndice 7.11}
\begin{equation*}
    \int\limits_{V}
    \left(
        \nabla \phi \cdot \nabla \psi + \phi \nabla^{2}\psi
    \right)\,dV
    = \int\limits_{S} \phi \frac{\partial \psi}{\partial \hat{\textbf{n}}}\,dS
\end{equation*}
Se repite el proceso pero tomando $\textbf{A} = \psi \nabla \phi$ y se obtiene la identidad análoga
\begin{equation*}
    \int\limits_{V}
    \left(
        \nabla \psi \cdot \nabla \phi + \psi \nabla^{2}\phi
    \right)\,dV
    = \int\limits_{S} \psi \frac{\partial \phi}{\partial \hat{\textbf{n}}}\,dS
\end{equation*}
es idéntica a la anterior intercambiando $\psi \longleftrightarrow \phi$. Restando ambos se llega a la segunda identidad de Green, expresión importante del Teorema de Green
\begin{equation}
    \int\limits_{V}
    \left(
        \phi \nabla^{2}\psi - \psi \nabla^{2}\phi
    \right)\,dV = 
    \int\limits_{S}
    \left(
        \phi\frac{\partial \psi}{\partial\hat{\textbf{n}}}
        -
        \psi\frac{\partial \phi}{\partial\hat{\textbf{n}}}
    \right)\,dS
        \label{ec:TeoremaGreen}
\end{equation}
entonces, con esta expresión, si se considera una distribución de carga arbitraria $\rho$ totalmente contenida en un recinto cerrado de volumen $V$, tomando como a $\phi\longrightarrow \varphi$ (el potencial electrostático), $\psi = \frac{1}{|\textbf{r}-\textbf{r}'|}$, reemplazando en la expresión del teorema y despejando el potencial electrostático se llega a\footnote{Ver apéndice 7.11 segunda parte}
\begin{equation*}
    \varphi(\textbf{r}) = 
    \int\limits_{V}
    \frac{\rho(\textbf{r}')}{|\textbf{r}-\textbf{r}'|}\,d^{3}r'
    -\frac{1}{4\pi}
    \int\limits_{S}
    \varphi \frac{\textbf{r}-\textbf{r}'}{|\textbf{r}-\textbf{r}'|^{3}}\cdot\hat{\textbf{n}}\,dS'
    -
    \frac{1}{4\pi}
    \int\limits_{S}
    \frac{\partial \varphi}{\partial \hat{\textbf{n}}}
    \frac{1}{|\textbf{r}-\textbf{r}'|}\,dS'
\end{equation*}
donde hasta el primer término se tiene la información de lo que ocurre con el potencial dentro del volumen del recinto y es lo que ya se sabía de electrostática. Los dos términos restantes son \textit{nuevos} y son la información de lo que ocurre en los contornos o la superficie del recinto. O sea que conociendo cuánto vale el potencial en el borde del recinto (segundo término) y cuánto vale la derivada normal del potencial en la superficie (tercer término), se tiene toda la información de cómo es el potencial electrostático en todo el volumen. \textit{Toda la información de lo que pasa adentro y afuera está contenida en el borde y en las condiciones de contorno.}\\
\indent Resulta que las condiciones de contorno pueden ser: O fijo el potencial en el borde o fijo la distribución de carga en el borde, y hay que remarcar que se fija una u otra porque una es función de la otra. Es decir, fijado el potencial en el borde, se deberá formar la distribución de carga acorde para mantener esa condición y viceversa (fijada la distribución de carga en la superficie, de define el potencial electrostático debido a estas cargas). Con lo cual solo hace falta conocer una condición.


%%%%%%%%%%%%%%%%%%%%%%%%%%%%%%%%%%%%%%%%%%%%%%%%
%%%%%%%%%%%%%%%%%%%%%%%%%%%%%%%%%%%%%%%%%%%%%%%%


\subsection{Unicidad de solución de la ecuación de Poisson}
Consideremos la ecuación de Poisson en una región finita del espacio con condiciones de contorno
\begin{itemize}
    \item \textbf{(i)} Tipo Dirichlet: $\left.\varphi\right)_{s}$, corresponde a especificar el potencial en el contorno
    \item \textbf{(ii)} Tipo Neumann:
    $\left.\frac{\partial \phi}{\partial \hat{\textbf{n}}}\right)_{s}$, corresponde a especificar el campo eléctrico en el contorno.
\end{itemize}
y el problema a resolver es la ecuación de Poisson $\nabla^{2}\varphi = -4\pi \rho$, donde la densidad $\rho$ es dato y alguna de las condiciones de contorno también.\\
\indent Quiero ver la unicidad de la solución de la ecuación de Poisson. Para eso supongo que existen 2 soluciones a la ecuación: $\varphi_{1}$ y $\varphi_{2}$, de modo tal que $\nabla^{2}\varphi_{1} = \nabla^{2}\varphi_{2} = -4\pi \rho$. Defino la cantidad $u = \varphi_{1} - \varphi_{2}$ y reemplazo en la primera identidad de Green, tomando $\phi = u$ y $\psi = u$, se llega a 
\begin{equation*}
    \int\limits_{V} u\nabla^{2}u\,dV
    +\int\limits_{V}|\nabla u |^{2}\,dV
    = \int\limits_{S} u\frac{\partial u}{\partial \hat{\textbf{n}}}\,dS
\end{equation*}
donde $\int u\nabla^{2}u\,dV = 0$ ya que el Laplaciano de $u$ no es mas que el Laplaciano de $\varphi_{1} - \varphi_{2}$, que valen lo mismo y entonces la integral se anula. Por otro lado, $\int u \frac{\partial u}{\partial \hat{\textbf{n}}}\,dS = 0$ ya que ambas soluciones tiene que cumplir las condiciones de contorno tanto de Neumann como de Dirichlet, de modo que se cancelan dentro de la integral. Por último, queda que
\begin{equation*}
    \int\limits_{V}|\nabla^{2}u|\,dV = 0
\end{equation*}
y para que eso valga, necesariamente tiene que ser $\nabla u = 0$, lo cual implica $u = \varphi_{1}-\varphi_{2} = cte$, la diferencia entre ambas soluciones sólo puede ser una constante.\\
\indent Entonces, considerando las condiciones de contorno de Dirichlet, es decir, imponer cuánto vale el potencial en el borde, ambos potenciales son iguales: $\varphi_{1} = \varphi_{2}$. Entonces no son dos soluciones distintas, son la misma.\\
\indent Para el caso de las condiciones de contorno de Neumann, es decir, impongo cuánto vale el campo eléctrico en los contornos, si vuelvo a ver que $cte = \varphi_{1} - \varphi_{2}$, veo que la diferencia entre los potenciales de una constante. Al derivar ambos me devuelven el mismo campo. Entonces la solución es nuevamente la misma!


%%%%%%%%%%%%%%%%%%%%%%%%%%%%%%%%%%%%%%%%%%%%%%%%
%%%%%%%%%%%%%%%%%%%%%%%%%%%%%%%%%%%%%%%%%%%%%%%%


\subsection{Ecuación de Laplace}
La ecuación de Laplace es la ecuación de Poisson igualada a cero, es decir
\begin{equation*}
    \nabla^{2}\varphi = 0
\end{equation*}
Por qué es importante esta ecuación? Porque resulta que lo que se puede hacer para resolver tanto el problema electrostático como magnetostático, es separar el recinto donde se encuentra la distribución de carga (o corriente) en regiones donde hay carga y donde no la hay. En las regiones donde hay carga necesariamente hay que resolver Poisson, pero en las regiones donde no, se puede resolver Laplace, que es más sencillo. Para las regiones donde hay carga, lo que se hace es tratarla como una condición de contorno.\\
\indent Para resolver la ecuación de Laplaces se usa el método de separación de variables.


%%%%%%%%%%%%%%%%%%%%%%%%%%%%%%%%%%%%%%%%%%%%%%%%
%%%%%%%%%%%%%%%%%%%%%%%%%%%%%%%%%%%%%%%%%%%%%%%%


\subsection{Separación de variables}
La primera hipótesis que se tiene que cumplir es que la ecuación a resolver sea separable en las coordenadas $x_{1}$, $x_{2}$ y $x_{3}$. La segunda hipótesis es que el contorno del volumen del recinto $V$ debe poder expresarse como un paralelepípedo en coordenadas $x_{1}$, $x_{2}$ y $x_{3}$.\\
\indent Entonces, si se tiene la ecuación de Laplace para $\varphi$: $\nabla^{2} \varphi= 0$, se quieren buscar soluciones de la forma
\begin{equation*}
    \varphi(x_{1},x_{2},x_{3}) = 
    u(x_{1})v(x_{2})w(x_{3})
\end{equation*}
de modo que al aplicar el Laplaciano, quedan 3 ecuaciones diferenciales ordinarias para cada coordenada. A su vez va a resultar que las soluciones en cada variable van a estar relacionadas por unas constantes de separación $\lambda_{1}$ y $\lambda_{2}$.


%%%%%%%%%%%%%%%%%%%%%%%%%%%%%%%%%%%%%%%%%%%%%%%%
%%%%%%%%%%%%%%%%%%%%%%%%%%%%%%%%%%%%%%%%%%%%%%%%


\subsubsection{Separación de variables en cartesianas}
Considero un recinto paralelepípedo con vértice en el origen del sistema de coordenada y lados de longitud $a$ en dirección $\hat{x}$, $b$ en dirección $\hat{y}$ y $c$ en dirección $\hat{z}$. Se tiene que en el interior del recinto que vale la ecuación de Laplace para el potencial $\nabla^{2} \varphi = 0$ y el valor del potencial en la superficie del recinto $\left.\varphi\right)_{s}$ es dato.\\
\indent Entonces ahora propongo una solución del tipo separable, como antes y queda
\begin{equation*}
    \frac{\partial^{2} u}{\partial x_{1}^{2}}v(x_{2})w(x_{3})
    +
    u(x_{1})\frac{\partial^{2} v}{\partial x_{2}^{2}}w(x_{3})
    +
    u(x_{1})v(x_{2})\frac{\partial^{2} w}{\partial x_{3}^{2}}
    = 0
\end{equation*}
ahora esta expresión la puedo dividir por la solución completa y obtengo
\begin{equation*}
    \frac{1}{u}\frac{\partial^{2}u}{\partial x_{1}^{2}}
    +
    \frac{1}{v}\frac{\partial^{2}v}{\partial x_{2}^{2}}
    +
    \frac{1}{w}\frac{\partial^{2}w}{\partial x_{3}^{2}}
    = 0
\end{equation*}
y la ventaja de esta nueva expresión es que se separó en términos que dependen únicamente de una variable, de forma que para que se cumpla la igualdad, necesariamente cada término tiene que ser constante:
\begin{equation*}
    \frac{1}{u}\frac{\partial^{2}u}{\partial x_{1}^{2}} = \lambda_{1},
    \quad
    \quad
    \frac{1}{v}\frac{\partial^{2}v}{\partial x_{2}^{2}} =
    \lambda_{2},
    \quad
    \quad
    \frac{1}{w}\frac{\partial^{2}w}{\partial x_{3}^{2}} = 
    -(\lambda_{1} + \lambda_{2})
\end{equation*}
entonces tengo 3 ecuaciones diferenciales ordinarias, una para cada dirección en el espacio. La ecuación es separable! Entonces, por ahora, para cada valor de $\lambda_{1}$ y $\lambda_{2}$ voy a tener una solución para $u$, $v$ y $w$. Entonces, la solución más general posible es la que suma todas las contribuciones de las soluciones para todos los valores posibles de las constantes de separación, es decir
\begin{equation*}
    \varphi(x_{1},x_{2},x_{3}) = 
    \sum\limits_{\lambda_{1},\lambda_{2}}
    A_{\lambda_{1},\lambda_{2}}
    u_{\lambda_{1}}(x_{1})
    v_{\lambda_{2}}(x_{2})
    w_{\lambda_{1},\lambda_{2}}(x_{3})
\end{equation*}
una vez que se tiene la solución general, hay que imponer las condiciones de contorno para hallar los factores de proporcionalidad $A_{\lambda_{1},\lambda_{2}}$. En este caso, tenemos condiciones de contorno del tipo Dirichlet, es decir, especificar el valor del potencial en el contorno. Para resolver el problema se va a pedir que al menos en dos direcciones del espacio se tengan \textit{condiciones de contorno triviales}. Es decir, condiciones de contorno tales que sea posible, en vez de aplicárselas a toda la suma, puedan ser aplicadas a cada solución por separado.\\
\indent Por ejemplo, vamos a imponer que en $x_{1} = 0, a$ y $x_{2} = 0, b$ el potencial valga 0.
\begin{equation*}
    \varphi(x_{1} = 0) = 
    \varphi(x_{1} = a) =
    \varphi(x_{2} = 0) =
    \varphi(x_{2} = b) = 0
\end{equation*}
entonces, si pido lo arriba lo que tengo es la siguiente ecuación diferencial
\begin{equation*}
    \frac{1}{u}\frac{\partial^{2}u}{\partial x_{1}^{2}} = \lambda_{1},
    \quad
    \quad
    \mbox{con condición de contorno}
    \quad
    u(0) = u(a) = 0
    \quad 
    \forall\ \lambda_{1}
\end{equation*}
básicamente es pedirle a cada término de la suma (en una dirección de condición de contorno trivial) que en evaluado en el contorno de 0. Para cada valor de $\lambda_{1}$, se tiene una solución $u_{\lambda_{1}}$, es decir, que hay un conjunto infinito $\left\{ u_{\lambda_{1}}, \lambda_{1}\right\}$.\\
\indent Además (sin demostración), cualquier $f(x_{1})$, para $x\ \in [0,a]$ que cumpla las condiciones de contorno se puede escribir como combinación lineal de las soluciones $u_{\lambda_{1}}$
\begin{equation*}
    f(x_{1}) = \sum\limits_{\lambda_{1}}A_{\lambda_{1}}u_{\lambda_{1}}(x_{1})
\end{equation*}
es decir, las $u_{\lambda_{1}}$ forman una base completa en el intervalo dado.


%%%%%%%%%%%%%%%%%%%%%%%%%%%%%%%%%%%%%%%%%%%%%%%%
%%%%%%%%%%%%%%%%%%%%%%%%%%%%%%%%%%%%%%%%%%%%%%%%


\subsubsection{Sturm-Liouville}
Resolviendo las ecuaciones diferenciales en las direcciones $\hat{x}$ y $\hat{y}$, es decir
\begin{equation*}
    \frac{\partial^{2}u}{\partial x_{1}^{2}} = \lambda_{1}u
    \quad
    \quad
    \mbox{y}
    \quad
    \quad
    \frac{\partial^{2}v}{\partial x_{2}^{2}} = \lambda_{2}v
\end{equation*}
se llega a las siguientes soluciones\footnote{Ver apéndice 9}
\begin{equation*}
    \left\{
        \begin{matrix}
            u_{n}(x_{1}) = \sin{(k_{n}x_{1})},
            \quad
            \mbox{con}
            \quad k_{n} = \frac{n\pi}{a}\\
            \\
            v_{m}(x_{2}) = \sin{(k_{m}x_{2})},
            \quad
            \mbox{con}
            \quad k_{m} = \frac{m\pi}{b}
        \end{matrix}
    \right.
\end{equation*}
Y además 
\begin{equation*}
    \lambda_{1} = -k_{n}^{2},
    \quad
    \quad 
    \lambda_{2} = -k_{m}^{2}
\end{equation*}
conociendo estas se puede resolver la tercera ecuación diferencial, para $w$ en las coordenadas $x_{3}$
\begin{equation*}
    w_{n,m}(x_{3}) 
    = A_{n,m}e^{\gamma_{n,m}x_{3}} 
    + B_{n,m}e^{-\gamma_{n,m}x_{3}}
\end{equation*}
donde $\gamma_{n,m} = -\lambda_{1}-\lambda_{2} = (k_{n}^{2} + k_{m}^{2})$. Finalmente, condensando todas estas soluciones en la solución total, se llega (cambio $x_{1}$, $x_{2}$ y $x_{3}$ por $x$, $y$ y $z$)
\begin{equation*}
    \varphi(x,y,z) = 
    \sum\limits_{n,m}
    \sin{(k_{n}x)}\sin{(k_{m}y)}
    \left(
        A_{nm}e^{\gamma_{n,m}z} + B_{n,m}e^{-\gamma_{n,m}z}
    \right)
\end{equation*}
esta es la solución para el potencial eléctrico dentro del recinto dado, cuando se tiene potencial 0 en las caras perpendiculares a las direcciones $x$ e $y$. Lo único que le falta a esta solución es hallar los coeficientes $A_{n,m}$ y $B_{n,m}$ que van a salir de especificar el potencial en las caras de normal $\hat{z}$. \\
\indent Notar que en las direcciones donde se tienen condiciones de contorno triviales, son las direcciones en las que se puede encontrar un conjunto de soluciones que forman una base completa ortonormal en el intervalo dado.
\textcolor{red}{Acá hace un par de cuentas más, pero no sé qué tanto sentido tienen.}\\
\indent El problema de todo este desarrollo, hasta ahora, es que parecería que solo se puede resolver el problema si en dos de las tres direcciones del problema se tienen condiciones de contorno triviales. Por ejemplo, qué pasaría si se tiene un paralelepípedo en cual todas sus caras están a distintos potenciales? La respuesta es que como el problema es lineal, vale el principio de superposición y siempre se resolver una parte del problema en el que se tengan dos condiciones de contorno triviales. Una vez que se resuelve para todas las caras, se suman las soluciones y se tiene el problema completo resuelto.



%%%%%%%%%%%%%%%%%%%%%%%%%%%%%%%%%%%%%%%%%%%%%%%%
%%%%%%%%%%%%%%%%%%%%%%%%%%%%%%%%%%%%%%%%%%%%%%%%


\subsubsection{Probema de Sturm-Liouville}
Dada una ecuación diferencial de la forma
\begin{equation}
    \left[
        -\frac{d}{dx}
        \left(
            p(x)\frac{d}{dx}
        \right)
        +q(x)
    \right]u
    = \lambda w(x)u
        \label{ec:ProblemaSturmLiouville}
\end{equation}
donde $w(x)$ es una función \textit{peso} definida positiva y $p(x)$ no se anula en el intervalo abierto $(a,b)$. Además si  tiene condiciones de contorno tales que dada una solución $u$ con autovalor $\lambda_{1}$ y alguna otra solución $v$ con autovalor $\lambda_{2}$ se cumple que
\begin{equation*}
    p(x) \Big[ uv' -vu'\Big]_{a}^{b} = 0,
    \quad
    \quad
    \mbox{con}
    \quad 
    \lambda_{1} \neq \lambda_{2}
\end{equation*}
entonces se tiene un problema de Sturm-Liouville.\\
\textbf{Teorema de Sturm-Liouville} Dada una ecuación diferencial con condiciones de contorno tales que definen un problema de Sturm-Liouville, y si el intervalo $[a,b]$ es finito, entonces
\begin{itemize}
    \item Existe un conjunto discreto de autovalores reales $\left\{ \lambda_{n}, n \geq 1 \right\}$ tales que $|\lambda_{n}|\to \infty$ cuando $n\to\infty$.
    \item Autovectores correspondientes a autovalores distintos son ortogonales $L_{w}^{2}[a,b]$
    \begin{equation*}
        \int\limits_{a}^{b}
        u_{\lambda_{i}}^{*}(x)u_{\lambda_{j}}(x)w(x)\,dx = \delta_{ij}C
    \end{equation*}
    \item Las autofunciones normalizadas forman una base ortonormal de $L_{w}^{2}[a,b]$
\end{itemize}
de los resultados de este teorema se sigue el siguiente resultado: Supongo que tengo una función $f(x)$ arbitraria en el intervalo $[a,b]$. Entonces, puedo escribir a la función $f$ como combinación lineal de los elementos de la base de soluciones del problema de S-L.
\begin{equation*}
    f(x) = \sum\limits_{n}A_{n}u_{n}(x)
\end{equation*}
donde los coeficientes $A_{n}$ se hallan aplicando ortogonalidad\footnote{Ver apéndice 10.6}
\begin{equation*}
    A_{n} 
    = \int\limits_{a}^{b}u_{m}^{*}(x)f(x)w(x)dx
\end{equation*}
y reemplazando en la expresión para $f(x)$
\begin{equation*}
    f(x) = \int\limits_{a}^{b}f(x')
    \left(
        \sum\limits_{n}
        u_{n}^{*}(x')u_{n}(x')w(x')
    \right)dx'
\end{equation*}
donde, la única forma que esa ecuación se cumpla es si lo que está entre paréntesis es la delta de Dirac
\begin{equation}
    \delta(x-x')
    = \sum\limits_{n}
    u_{n}^{*}(x')u_{n}(x')w(x')
        \label{ec:DeltaDiracDiscreta}
\end{equation}
en resumen se tienen las relaciones de completitud y ortogonalidad:
\begin{equation*}
    \underbrace{%
    \int\limits_{a}^{b}
    u_{m}^{*}(x)u_{n}(x)w(x) = \delta_{nm}}_{\mbox{Ortogonalidad}},
    \quad
    \quad
    \quad
    \underbrace{%
    w(x)\sum\limits_{n}
    u_{n}^{*}(x)u_{n}(x') = \delta(x-x')}_{\mbox{Completitud}}
\end{equation*}

Observaciones finales: Notar que si el intervalo $[a,b]$ no es finito, se tiene un espectro continuo de autovalores reales y las autofunciones forman una base ortonormal de $L_{w}^{2}[a,b]$.\\
\indent Ejemplo de ecuación de Sturm-Liouville: Si se tiene $P(x) = \frac{\hbar}{2m}$, $q(x) = V(x)$ y $w(x) = 1$, al reemplazar en la ecuación de Sturm-Liouville se llega a la ecuación de Schrodinger. Schrodinger es un problema de S-L con las condiciones de contorno correctas!


%%%%%%%%%%%%%%%%%%%%%%%%%%%%%%%%%%%%%%%%%%%%%%%%
%%%%%%%%%%%%%%%%%%%%%%%%%%%%%%%%%%%%%%%%%%%%%%%%


\subsubsection{Separación de variables en coordenadas esféricas}
Supongo que tengo un recinto tal que escrito en coordenadas esféricas, sus contornos son los de un paralelepípedo\footnote{Esto no significa que es realmente un paralelepípedo, una esfera visto en cartesianas, es un paralelepípedo en coordenadas esféricas}, es decir, en la dirección radial $\hat{r}$ se tienen contornos en $r_{1}$ y $r_{2}$, en la dirección azimutal\footnote{rotando alrededor del eje $z$} $\hat{\phi}$ se tiene contornos $\phi_{1}$ y $\phi_{2}$ y por último, en la dirección polar $\theta$ se tienen contornos $\theta_{1}$ y $\theta_{2}$.\\
\indent Hay que resolver la ecuación de Laplace para los potenciales, $\nabla^{2}\varphi$, en coordenadas esféricas:
\begin{equation*}
    \nabla^{2} \varphi = 
    \LaplaceSpherics{\varphi}
    = 0
\end{equation*}
Para poder resolver el problema tienen que cumplirse las tres hipótesis:
\begin{itemize}
    \item Que el recinto forme un paralelepípedo en el sistema de coordenadas esféricas (se cumple)
    \item Que la ecuación resulte separable (hay que problarlo) 
    \item Que haya condiciones de contorno triviales en al menos dos direcciones del paralelepípedo (hay que imponerlo)
\end{itemize}
entonces tengo que probar que la ecuación es separable. Para eso propongo una solución de la forma
\begin{equation*}
    \varphi(r,\theta,\phi) = R(r)P(\theta)Q(\phi)
\end{equation*}
y reemplazo en la ecuación de Laplace y además, como antes, divido por la solución completa para simplificar, se llega \footnote{Ver apéndice 12.13}
\begin{equation*}
    \frac{1}{rR}
    \frac{d^{2}}{dr^{2}}(rR)
    +
    \frac{1}{Pr^{2}\sin{(\theta)}}
    \frac{d}{d\theta}
    \left(
        \sin{(\theta)}\frac{dP}{d\theta}
    \right)
    +
    \frac{1}{Qr^{2}\sin^{2}{(\theta)}}
    \frac{d^{2}Q}{d\theta^{2}}
    = 0
\end{equation*}
Pero todavía así no se pudo separar la ecuación. Pero viendo que el último término tiene un $r^{2}\sin^{2}{(\theta)}$ dividiendo, puedo pasarlo multiplicando y dejar este último solamente como función de $\phi$. Así, quedan los dos primeros términos como función de de $r$ y $\theta$. Lo que gané con esto es que para que valga la ecuación, necesariamente los dos primeros términos sumados y el último tienen que ser constantes, es decir
\begin{equation}
    \underbrace%
    {
        \frac{r\sin^{2}{(\theta)}}{R}
        \frac{d^{2}}{dr^{2}}(rR)
        +
        \frac{\sin{(\theta)}}{P}
        \frac{d}{d\theta}
        \left(
            \sin{(\theta)}\frac{dP}{d\theta}
        \right)
    }_{\alpha_{1}}
        +
    \underbrace%
    {%
        \frac{1}{Q}
        \frac{d^{2}Q}{d\theta^{2}}
    }_{-\alpha_{1}}
    = 0
        \label{ec:SeparacionEsfericas2}
\end{equation}
y ahora ya se puede resolver la ecuación del último término
\begin{equation*}
    -\alpha_{1}Q = \frac{d^{2}Q}{d\theta^{2}}
\end{equation*}
para el caso particular de que el recinto es una esfera (vista desde un sistema de coordenadas cartesianas), se van a tener condiciones de contorno periódicas en $\phi$, es decir, tiene que pasar que $Q(0) = Q(2\pi)$ (condiciones de contorno de Dirichlet) o $Q'(0) = Q'(2\pi)$ (condiciones de contorno de Neumann).\\
\indent Resulta que si $\alpha_{1}<0$, no existe solución que cumpla las condiciones de contorno, con lo cual miro $\alpha_{1}>0$. Para ese caso se tiene una solución de la forma
\begin{equation*}
    Q(\phi) 
    = A\cos{(k\phi)} 
    + B\sin{(k\phi)} 
    + C 
    + \cancel{D\phi}
\end{equation*}
donde también la solución lineal es solución, pero se cancela porque no cumple las condiciones de contorno. Para el caso en que $k \neq 0$, entonces $C = 0$ por condiciones de contorno.\\
\indent Asi que ahora analizo el caso en que $k = m$, con $m$ natural, $m = 1,2,3,...$, entonces se puede formar una base de soluciones ortogormales $\{ A_{m}\cos{(m\phi)}, B_{m}\sin{(m\phi)} \}$ en el intervalo $[0,2\pi)$, de modo que la solución general para $Q$ es
\begin{equation*}
    Q(\phi) = C_{0} 
    + \sum\limits_{m = 1}^{\infty}
    \left[
        A_{m}\cos{(m\phi)}
        +
        B_{m}\sin{(m\phi)}
    \right]
\end{equation*}
Ya tengo una de las soluciones, ahora hay que seguir con los dos primeros términos del Laplaciano que hay que resolver. Ahora lo que hago es \eqref{ec:SeparacionEsfericas2}, reemplazar por $\alpha_{1}$ y pasar dividiendo el seno al cuadrado. Ahora me quedaron 3 términos, uno que solo depende de la variable radial $r$ y los otros dos que solo dependen de $\theta$, nuevamente tienen que ser constantes para que la ecuación se cumpla, de forma que tengo 
\begin{equation*}
    \underbrace%
    {%
        \frac{r}{R}
        \frac{d^{2}}{dr^{2}}(rR)
    }_{\alpha_{2}}
    +
    \underbrace%
    {%
        \frac{1}{P\sin{(\theta)}}
        \frac{d}{d\theta}
        \left(
            \sin{(\theta)}\frac{dP}{d\theta}
        \right)
        -
        \frac{\alpha_{1}}{\sin^{2}{(\theta)}}
    }_{-\alpha_{2}}
        = 0
\end{equation*}
de estas dos ya se puede resolver la ecuación para $\theta$ que queda escrita como
\begin{equation*}
    \frac{1}{\sin{(\theta)}}\frac{d}{d\theta}
    \left(
        \sin{(\theta)}\frac{dP}{d\theta}
    \right)
    -\frac{\alpha_{1}P}{\sin^{2}{(\theta)}}
    + \alpha_{2}P = 0
\end{equation*}
usando que la parte en $r$ es constante y vale $\alpha_{2}$. De nuevo usando que hay periodicidad en $\phi$, resulta $\alpha_{1} = m^{2}$ \footnote{Ver apendice 12.13}, reemplazo y obtengo
\begin{equation*}
    \frac{1}{\sin{(\theta)}}\frac{d}{d\theta}
    \left(
        \sin{(\theta)}\frac{dP}{d\theta}
    \right)
    +
    \left(
        \alpha_{2} - \frac{m^{2}}{\sin^{2}{(\theta)}}
    \right)P
    = 0 
\end{equation*}
si miro el caso $m = 0$ y reemplazando $x = \cos{(\theta)}$ se llega \footnote{Ver apéndice 12.13} a la siguiente expresión
\begin{equation}
    \frac{d}{dx}
    \left[
        (1-x^{2})\frac{dP}{dx}
    \right]
    + \alpha_{2} P
    = 0
        \label{ec:SeparacionEsfericas3}
\end{equation}
es una ecuación de Sturm-Liouville! Si en \eqref{ec:ProblemaSturmLiouville} tomo $p(x) = 1-x^{2}$, $q(x) = 0$, $u = P$, $w = 1$ y $\lambda = \alpha_{2}$ recupero esta ecuación. Para poder resolver completamente el problema, necesito que las condiciones de contorno sean tales que definan un problema sea de Sturm-Liouville (y entonces asegurarme que existe un conjunto discreto de autovalores reales ordenados en forma creciente, autofunciones que forman base completa ortonormal en el intervalo dado).\\
\indent En este caso, el recinto es toda la esfera, entonces $\theta_{1} = 0$ y $\theta_{2} = \pi$, con lo cual, la nueva variable $x$ está en el intervalo $[-1,1]$, como $P = 1-x^{2}$ se anula en $-1$ y $1$ ($\theta_{1}=0$ y $\theta_{2} = \pi$), automáticamente es un problema de Sturm-Liouville, no hacen falta condiciones de contorno triviales (al menos en la esfera, si tuviese una media esfera necesitaría un contorno a tierra.\\
\indent Si se resuelve la ecuación, se llega a que los autovalores son de la forma $\alpha_{2} = l(l+1)$\footnote{Nunca voy a entender bien de donde sale eso. Ver video clase 5 minuto 39:20} y la ecuación se reescribe como
\begin{equation*}
    \frac{d}{dx}
    \left[
        (1-x^{2})\frac{dP}{dx}
    \right]
    + l(l+1) P
    = 0
\end{equation*}
es la ecuación de Legendre, cuyas soluciones son los Polinomios de Legendre, que tiene una expresión general cerrada dada por la fórmula de Rodrigues
\begin{equation*}
    P_{l}(x) =
    \frac{1}{2^{l}l!}
    \frac{d^{l}}{dx^{l}}
    \left(
        x^{2} - 1
    \right)^{l}
\end{equation*}
Los polinomios de Legendre tienen la propiedad de que se normalizan a $1$ cuando $x = 1$ (por convención). Además poseen tantos ceros como grado del polinomio, es decir, el polinomio de Legendre de orden 5, tiene 5 ceros, el de orden 3 tiene 3 ceros, y así. Por otro lado, tienen  las siguientes relaciones de Ortogonalidad y Completitud
\textbf{Ortogonalidad:}
\begin{equation*}
    \int\limits_{-1}^{1}
    P_{l}(x)P_{l'}(x)dx = \left(\frac{2}{2l +1}\right)\delta_{ll'}
\end{equation*}
\textbf{Completitud}
\begin{equation*}
    \sum\limits_{l = 0}^{\infty}
    \left(
        \frac{2l+1}{2}
    \right)P_{l}(x)P_{l}(x')
    =
    \delta(x-x')
\end{equation*}
Acá hay fórmulas de recuerrencia que no voy a copiar.\\
\indent Para el caso en que $m \neq 0$, los Polinomios de Legendre se pueden escribir como
\begin{equation*}
    P_{l}^{m} = \frac{(-1)^{m}}{2^{l}l!}(1-x^{2})^{m/2}
    \frac{d^{l+m}}{dx^{l+m}}
    \left( x^{2}-1\right)^{l}
\end{equation*}
y para $m$ fijo, $\{ P_{l}^{m}(x)\}$ forman base de funciones en el intervalo $[0, \pi]$. Las relaciones de ortogonalidad y completitud para estas son\\
\textbf{Ortogonalidad:}
\begin{equation*}
    \int\limits_{-1}^{1}
    P_{l}^{m}(x)P_{l'}^{m}(x)dx 
    =
    \frac{2}{2l +1}
    \frac{(l+m)!}{(l-m)!}
    \delta_{ll'}
\end{equation*}
\textbf{Completitud}
\begin{equation*}
    \sum\limits_{l = 0}^{\infty}
    \left(
        \frac{2l+1}{2}
    \right)
    \frac{(l-m)!}{(l+m)!
    }
    P_{l}^{m}(x)P_{l}^{m}(x')
    =
    \delta(x-x')
\end{equation*}
Entonces lo que se hizo hasta ahora fue, tomar como recinto a la esfera completa e imponer condiciones de contorno triviales tanto en $\phi$ como en $\theta$ de forma de poder resolver con Sturm-Liouville. Asi que se tiene base de soluciones en ambas direcciones. A esta base de soluciones se la llama Armónicos Esféricos y se escriben combinando ambas soluciones de forma general, según
\begin{equation}
    Y_{lm}(\theta, \phi) =
    \sqrt%
    {
        \frac{2l + 1}{4\pi}
        \frac{(l-m)!}{(l+m)!}
    }
    P_{l}^{m}(cos{(\theta)})
    e^{im\phi}
        \label{ec:ArmonicosEsfericos}
\end{equation}
donde $l = 0,1,2,...$ y $-l \leq m \leq l$.\\
\indent Los armónicos esféricos son modos de oscilación de la superficie esférica. son modos normales del problema de Laplace en la superficie de la esfera unitaria.\\
\indent Las relaciones de ortogonalidad y completitud vienen dadas por\\
\textbf{Ortogonalidad:}
\begin{equation*}
    \int Y_{lm}^{*}(\theta,\phi)
    Y_{l'm'}(\theta,\phi)
    \sin{(\theta)}\,d\theta\,d\phi = 
    \delta_{ll'}\delta_{mm'}
\end{equation*}
\textbf{Completitud:}
\begin{equation*}
    \sum\limits_{l = 0}^{\infty}
    \sum\limits_{m = -l}^{l}
    Y_{lm}^{*}(\theta',\phi')
    Y_{lm}(\theta, \phi)
    = \delta(\phi - \phi')\frac{\delta(\theta - \theta')}{\sin{(\theta)}}
\end{equation*}
Ahora resta ver las soluciones para $r$, que es la dirección donde se tienen condiciones de contorno no triviales. Se tiene
\begin{equation*}
    \frac{r}{R}\frac{d^{2}}{dr^{2}}(rR) = l(l+1)
\end{equation*}
se propone como solución a 
\begin{equation*}
    R_{l}(r) = A_{l}r^{l} + \frac{B_{l}}{r^{l+1}}
\end{equation*}
con lo cual, la solución general para el problema de la esfera queda
\begin{equation*}
    \varphi(r,\theta,\phi) = 
    \sum\limits_{l = 0}^{\infty}
    \sum\limits_{m = -l}^{l}
    \left(
        A_{lm}r^{l} + \frac{B_{lm}}{r^{l+1}}
    \right)Y_{lm}(\theta,\phi)
\end{equation*}
y hay que imponer las condiciones de contorno para $r$, de donde van a salir las expresiones para $A_{l}$ y $B_{l}$. Como son condiciones de contorno no triviales, significa que, por ejemplo, puedo tener un capacitor esférico con dos capas, una en $r_{1}$ y otra en $r_{2}$, cada una a un potencial diferente.\\
\indent Algunas casos particulares: 
\begin{itemize}
    \item Supongo que tengo una distribución de carga sobre la esfera, que no necesariamente tiene que tener simetría esférica (o sea puede no ser uniforme en la superficie o volumen de la esfera) y se quiere medir el potencial en $r_{1}$ fuera de la esfera. La condición de contorno es que si $r_{1}\to \infty$, entonces el potencial vale cero (si no hay cargas en el infinito). Para que la solución general cumpla esto, necesariamente $A_{lm} = 0\ \forall\ l,m$. En ese caso la solución se reescribe como
    \begin{equation*}
        \varphi(r,\theta,\phi) = 
        \sum\limits_{l = 0}^{\infty}
        \sum\limits_{m = -l}^{l}
        \frac{B_{lm}}{r^{l+1}}
        Y_{lm}(\theta,\phi)
    \end{equation*}
    y si expando la suma explícitamente para $l = 0$, $m = 0$, $l = 1$, $m = \{-1,0,1\}$, $l = 2$, $m = \{-2,-1,0,1,2\}$, etc
    \begin{equation*}
        \varphi(r, \theta, \phi) = 
        \frac{1}{4\pi}\frac{B_{0,0}}{r} +
        \frac{B_{1,-1}}{r^{2}}Y_{1,-1} +
        \frac{B_{1,0}}{r^{2}}Y_{1,0} +
        \frac{B_{1,1}}{r^{2}}Y_{1,1} +
        \cdots
    \end{equation*}
    donde el primer término es una constante sobre $r$, es isótropo, es el primer término de la expansión multipolar! Expandir en los armónicos esféricos devuelve la expansión multipolar del potencial.
    \item En el otro caso, quiero resolver el potencial dentro de la esfera. En este caso, como eventualmente voy a mirar $r=0$, necesito pedir que no diverja la expresión, entonces necesariamente el coeficiente $B_{ml} = 0\ \forall\ l,m$.
\end{itemize}


%%%%%%%%%%%%%%%%%%%%%%%%%%%%%%%%%%%%%%%%%%%%%%%%
%%%%%%%%%%%%%%%%%%%%%%%%%%%%%%%%%%%%%%%%%%%%%%%%


\subsubsection{Ejemplo}
En el siguiente ejemplo se tiene una carga puntual $q$ ubicada sobre el eje $z$, a una distancia $d$ del origen de coordenadas. El potencial de esta distribución es trivial de escribir con lo que ya se sabe y vale
\begin{equation*}
    \varphi(\textbf{r}) = \frac{q}{|\textbf{r}-d\hat{z}|}
\end{equation*}
Se lo quiere resolver por separación de variables. El problema que se tiene que resolver para este sistema, es el problema de Laplace, que es $\nabla^{2}\varphi = 0$, sin embargo, lo que en realidad se tiene es un problema de Poisson porque hay una carga, es decir $\nabla^{2}\varphi = 4\pi \rho(\textbf{r})$, donde la distribución será simplemente $\rho(\textbf{r}) = q\, \delta (\textbf{r}-d\hat{z})$. Entonces hay que partir el problema en regiones donde vale la ecuación de Laplace y dejar la distribución de carga en los contornos. Para eso se toma una superficie esférica de radio $d$ y para todo $r < d$ se tiene una región $I$ donde vale $\nabla^{2}\varphi = 0$ y para todo $r > d$ otra región $II$ donde vale $\nabla^{2}\varphi = 0$, dejando la carga en el contorno o interfaz de ambas superficies. Se ha transformado un problema de Poisson en dos problemas de Laplace con contornos.\\
\indent Entonces se pueden usar los resultados que ya se tienen:\\
\indent Para $r < d$ se tienen soluciones de la forma
\begin{equation*}
    \varphi_{I}(r,\theta,\phi) = 
    \sum\limits_{l=0}^{\infty}
    \sum\limits_{m = -l}^{l}
    A_{l,m}r^{l}Y_{l,m}(\theta, \phi)
\end{equation*}

y para $r>d$ se tienen soluciones de la forma
\begin{equation*}
    \varphi_{II}(r,\theta,\phi) = 
    \sum\limits_{l=0}^{\infty}
    \sum\limits_{m = -l}^{l}
    \frac{B_{l,m}}{r^{l+1}}Y_{l,m}(\theta, \phi)
\end{equation*}
y como el potencial tiene que ser continuo en $r = d$, tiene que valer que $\varphi_{I}(d,\theta,\phi) = \varphi_{II}(d,\theta,\phi)$ y de ahí se obtiene una relación entre los coeficientes $A_{lm}$ y $B_{lm}$:
\begin{equation*}
    B_{l,m} = A_{l,m}d^{2l+1}
\end{equation*}
Para hallar el coeficiente $B_{l,m}$ hay que usar las condiciones de contorno del problema, que en este caso van a ser condiciones de contorno de Neumann que salen de la ecuación de Poisson:
\begin{equation*}
    \left.
        \frac{\partial \varphi_{I}}{\partial r}
    \right)_{r = d}
    -
    \left.
        \frac{\partial \varphi_{II}}{\partial r}
    \right)_{r = d}
    = 4\pi\sigma(\theta,\phi)
\end{equation*}
donde $\sigma(\theta,\phi)$ es una densidad superficial de carga, que integrada en toda la superficie debe ser el valor de la carga puntual $q$. La densidad se puede escribir como \footnote{Ver apéndice 13.3}
\begin{equation*}
    \sigma = \frac{q}{2\pi d^{2}\sin{(\theta)}}
\end{equation*}
entonces, reemplazando en las condiciones de contorno las soluciones que se tienen hasta ahora y la densidad superficial de carga y usando ortogonalidad, se puede despejar el valor del coeficiente $A_{l,m}$, que termina siendo
\begin{equation*}
    A_{l,m} = 
    \left(
        \frac{4\pi}{2l + 1}
    \right)^{1/2}
    \frac{q}{d^{l+1}}
\end{equation*}
Una observación importante es que el coeficiente $A_{l,m}$ no depende de $\phi$, es decir, el índice $m = 0$! esto es porque el problema tiene simetría de rotación alrededor del eje $z$. Si esto lo notaba de antemano, podía plantear las soluciones únicamente como funciones de los polinomios de Legendre y no de los armónicos esféricos, es decir
\begin{equation*}
    \left\{
        \begin{matrix}
            \varphi_{I}(r,\theta,\phi) = 
            \sum\limits_{l=0}^{\infty}
            A_{l}'r^{l}P_{l}(\cos{(\theta)})\\
            \\
            \varphi_{II}(r,\theta,\phi) = 
            \sum\limits_{l=0}^{\infty}
            \frac{B_{l}'}{r^{l+1}}P_{l}(\cos{(\theta)})
        \end{matrix}
    \right.
\end{equation*}
Repitiendo el proceso para hallar los coeficientes se llega a que \footnote{Ver apéndice 13.3} $A_{l}' = q/d^{l+1}$, $B_{l}' = q\,d^{l}$ y, además, la relación entre los coeficientes para el desarrollo con armónicos esféricos y con los polinomios de Legendre únicamente es\footnote{Ver apéndice 13.3}
\begin{equation*}
    A_{l}' = A_{l,0}\sqrt{\frac{2l + 1}{4\pi }}
\end{equation*}
Finalmente, la solución para el problema será
\begin{equation*}
    \varphi(\textbf{r}) = q \sum\limits_{l=0}^{\infty}
    \frac{r_{<}^{l}}{r_{>}^{l+1}}
    P_{l}(\cos{(\theta)})
    = \frac{q}{|\textbf{r}-d\hat{z}|}
\end{equation*}
donde se definieron $r_{<} = \min\{r,d\}$ y $r_{>} = \max\{r,d\}$. Lo importante es ver que se obtuvo una igualdad que no parece para nada intuitiva. Es en realidad la expansión en la base de los polinomios de Legendre del potencial.\\
\indent Surge entonces la necesidad de generalizar esto para el caso en el que la carga puntual no está sobre el eje $z$, si no que puede estar en cualquier parte del espacio.


%%%%%%%%%%%%%%%%%%%%%%%%%%%%%%%%%%%%%%%%%%%%%%%%
%%%%%%%%%%%%%%%%%%%%%%%%%%%%%%%%%%%%%%%%%%%%%%%%


\subsubsection{Teorema de adición de armónicos esféricos}
Entonces, para resolver el potencial de una carga puntual en cualquier punto del espacio (ya no colocada convenientemente sobre el eje $z$), lo que se hace es pensar que colocamos la carga $q$ en un nuevo eje $z'$, que forma un ángulo $\theta'$ con el eje $z$ original. Como lo que se quiere ahora es calcular el potencial en cualquier punto del espacio alrededor del eje $z'$, defino un nuevo ángulo polar $\gamma$ que mide el ángulo del vector $\textbf{r}$ de punto campo desde este eje primado. De esta forma, puedo reescribir $\gamma = \theta - \theta'$, donde $\theta$ es el ángulo polar usual (medido desde el eje $z$ no primado). Lo mismo sucede con el ángulo azimutal, se tiene el ángulo azimutal, aparece un nuevo ángulo $\phi'$.\\
\indent Se quiere poder expresar a los armónicos esféricos para este nuevo eje $z'$ en términos de los armónicos esféricos del eje $z$ original. Equivalentemente a esto es querer expresar a los polinomios de Legendre respecto del ángulo $\gamma$ en función de los armónicos esféricos evaluados $\theta$ y $\phi$ (los medidos desde $z$) , es decir $P_{l}{(\cos{(\gamma)})} = f(\theta, \phi)$.\\
\indent Como los armónicos esféricos son base completa, puedo escribir a los polinomios de Legendre como
\begin{equation*}
    P_{l}{(\cos{(\gamma)})}=
    \sum\limits_{l' = 0}^{\infty}\sum\limits_{m = -l'}^{l'}
    A_{l',m}(\theta',\phi')Y_{l',m}(\theta,\phi)
\end{equation*}
donde los coeficientes $A_{l',m}(\theta',\phi')$ están evaluados en los ángulos primados porque estos coeficientes van a depender de la posición de la carga $q$ (que va a ser siempre el nuevo eje $z'$) medidos desde el eje $z$ original.\\
\indent Sin hacer las cuentas\footnote{Ver apéndice 14.6}, lo que se hace es escribir los polinomios de Legendre en función de $\gamma$ con la base de armónicos esféricos, con unos ciertos coeficientes $A_{l,m}(\theta',\phi')$, que terminan siendo los armónicos esféricos evaluados en $\theta'$ y $\phi'$ con una constante de proporcionalidad
\begin{equation*}
    A_{l,m}(\theta',\phi') = \frac{4\pi}{2l + 1}Y_{l,m}(\theta',\phi')
\end{equation*}
\indent El resultado final es que la expansión en la base de armónicos esféricos del potencial de una carga puntual, en cualquier punto del espacio $\textbf{r}'$ es
\begin{equation}
    \varphi(\textbf{r})
    = 4\pi q
    \sum\limits_{l = 0}^{\infty}
    \sum\limits_{m = -l}^{l}
    \frac{1}{2l + 1}
    \left(
        \frac{r_{<}^{l}}{r_{>}^{l+1}}
    \right)
    Y_{l,m}^{*}(\theta', \phi')
    Y_{l,m}(\theta, \phi)
    = \frac{q}{|\textbf{r}-\textbf{r}'|}
        \label{ec:DesarrolloEsfericas}
\end{equation}
donde puede tomarse como los coeficientes de la expansión a $\frac{1}{2l+1}Y_{l,m}^{*}(\theta',\phi')$.
%%%%%%%%%%%%%%%%%%%%%%%%%%%%%%%%%%%%%%%%%%%%%%%%
%%%%%%%%%%%%%%%%%%%%%%%%%%%%%%%%%%%%%%%%%%%%%%%%



\subsubsection{Separación de variables en coordenadas cilíndricas}
Se quiere resolver la ecuación de Laplace en coordenadas cilíndricas. El Laplaciano de un potencial $\varphi$ en cilíndricas se escribe como
\begin{equation*}
    \nabla^{2}\varphi(\rho,\theta, z) 
    = 
    \frac{\partial^{2}\varphi}{\partial \rho^{2}}
    +
    \frac{1}{\rho}\frac{\partial \varphi}{\partial \rho}
    +
    \frac{1}{\rho^{2}}\frac{\partial^{2}\varphi}{\partial \phi^{2}}
    +
    \frac{\partial^{2}\varphi}{\partial z^{2}}
    = 0
\end{equation*}
Lo primero que hay que hacer es proponer una solución y ver si la ecuación es separable en coordenadas cilíndricas. Lo segundo es ver si se tiene un recinto que en estas coordenadas sea un paralelepípedo (que sí se cumple por como se arma el problema) y lo último es ver en cuáles dos direcciones conviene imponer las condiciones de contorno triviales para tener base de soluciones. Por último, hallar la solución en la tercera dirección usando las bases halladas antes.\\
\indent Se propone una solución de la forma
\begin{equation*}
    \varphi(\rho,\theta,z) = R(\rho)Q(\theta)Z(z)
\end{equation*}
reemplazando en el Laplaciano, dividiéndolo por la solución (como ya se hizo antes) se llega a la siguiente
\begin{equation*}
    \underbrace%
    {%
        \frac{1}{R}
        \left(
            \frac{d^{2}}{d\rho^{2}} 
            + \frac{1}{\rho}\frac{d}{d\rho}
        \right)R
        +
        \frac{1}{\rho^{2}Q}\frac{d^{2}Q}{d\phi^{2}}
    }_{= -\lambda}
    +
    \underbrace%
    {%
        \frac{1}{Z}\frac{d^{2}Z}{dz^{2}}
    }_{= \lambda}
    = 0
\end{equation*}
entonces ya se puede resolver la ecuación para $z$.
\begin{equation}
    \frac{d^{2}Z}{dz^{2}} = \lambda Z
        \label{ec:CilindricasZ}
\end{equation}
Luego, como el término que tenía $z$ ahora es constante, si se multiplica por $\rho^{2}$ a toda la expresión, el término con $\phi$ ya no depende de ningún otro término y se puede volver a definir una constante $\beta$
\begin{equation*}
    \underbrace%
    {%
        \frac{\rho^{2}}{R}
        \left(
            \frac{d^{2}}{d\rho^{2}} 
            + \frac{1}{\rho}\frac{d}{d\rho}
        \right)R
        +
        \rho^{2}\lambda
    }_{= \beta}
    +
    \underbrace%
    {%
        \frac{1}{Q}\frac{d^{2}Q}{d\phi^{2}}
    }_{= -\beta}
    = 0
\end{equation*}
entonces
\begin{equation}
    \frac{d^{2}Q}{d\phi^{2}} = -\beta Q
        \label{ec:CilindricasPhi}
\end{equation}
y finalmente la ecuación separada queda
\begin{equation}
    \rho^{2}\frac{d^{2}R}{d\rho^{2}} + \rho\frac{dR}{d\rho}
    =
    \left(
        \beta - \lambda\rho^{2}
    \right)R
        \label{ec:CilindriciasRho}
\end{equation}
con lo cual, la Ecuación de Laplace para la solución propuesta resultó separable. Ahora hay que resolver las ecuaciones.\\
\indent Para el caso de la ecuación en $\phi$, \eqref{ec:CilindricasPhi}, se van a tener distintos tipos de soluciones dependiendo del valor de $\beta$:
\begin{equation*}
    \mbox{si}\ \left\{
        \begin{matrix}
            \beta = -\nu^{2}, \Longrightarrow 
            &Q_{\nu}(\phi) = \{e^{\nu\phi}, e^{-\nu\phi} \}\\
            &\\
            \beta = 0, \Longrightarrow 
            &Q(\phi) = \{1, \phi\}\\
            &\\
            \beta = \nu^{2}, \Longrightarrow
            &Q_{\nu}(\phi) = \{\cos{(\nu \phi)}, \sin{(\nu\phi)} \}
        \end{matrix}
    \right.
\end{equation*}
Para el caso de $z$, \eqref{ec:CilindricasZ}, se van a tener también distintos tipos de soluciones dependiendo el valor de la constante de separación $\lambda$:
\begin{equation*}
    \mbox{si}\ \left\{
        \begin{matrix}
            \lambda = k^{2}, \Longrightarrow 
            &Z_{k}(z) = \{e^{kz}, e^{-kz} \}\\
            &\\
            \lambda = 0, \Longrightarrow 
            &Z(z) = \{1, z\}\\
            &\\
            \lambda = -k^{2}, \Longrightarrow
            &Z_{k}(Z) = \{\cos{(kz)}, \sin{(kz)} \}
        \end{matrix}
    \right.
\end{equation*}
¿Cómo se determina qué valores toman las constantes $\beta$ y $\lambda$? Bueno, depende de las condiciones de contorno que imponga el problema. Si en la coordenada $\phi$ se imponen condiciones de contorno de Sturm Liouville, entonces en esa dirección habrá una base de soluciones, es decir, los senos y cosenos y entonces $\beta = \nu^{2}$ o a lo sumo $\beta =0$. En el caso que en esa dirección no hayan condiciones de contorno de S-L, se usan las exponenciales. Idem para el caso en $z$, si en esa dirección hay condiciones de contorno de S-L, entonces $\lambda = -k^{2}$ y tengo una base de senos y cosenos. De lo contrario, las soluciones son las exponenciales.\\
\indent ¿Y por qué solo las exponenciales complejas (senos y cosenos) pueden formar base y las exponenciales reales (senos y cosenos hiperbólicos) no? Porque los cosenos hiperbólicos no tienen ceros, entonces no es posible escribir cualquier función como combinación lineal de senos y cosenos hiperbólicos, en cambio con los senos y cosenos sí.\\
\indent Nótese que si se tienen condiciones de contorno en dirección $\phi$ y dirección $z$ tales que se tiene un problema de S-L, necesariamente en la dirección $\rho$ no habrá problema de S-L y las soluciones no formarán base. Por otro lado, si tengo un problema de S-L solo en $\phi$ o solo en $z$, entonces en $\rho$ debería tenerse un problema de S-L para poder resolver el problema. Siempre tienen que haber dos problemas de S-L para poder resolver el problema.\\
\indent Entonces ahora hay que resolver la última ecuación, en $\rho$ \eqref{ec:CilindriciasRho}, para las diferentes posibilidades:
\begin{itemize}
    \item \textbf{(1)} S-L en $\phi$ y $\rho$
    \item \textbf{(2)} S-L en $\phi$ y $z$
    \item \textbf{(3)} S-L en $z$ y $\rho$
\end{itemize}
pero siempre voy a considerar que tengo S-L en $\phi$, y analizaré los casos con S-L en $\rho$ y S-L en $z$.
\begin{enumerate}
    \item Primero el caso de $\lambda = k^{2}$, que da como soluciones en la dirección $z$ a las exponenciales reales (no forman base), con lo cual, en $\rho$ y en $\phi$ va a haber base de soluciones, caso \textbf{(1)}, porque se define un problema de S-L en cada una.\\
    \indent Dividiendo la expresión \eqref{ec:CilindriciasRho} por $\rho^{2}$ y tomando el cambio de variables $x = k\rho$, se obtiene la siguiente expresión, que es la ecuación de Bessel en cilíndricas:
    \begin{equation*}
        \frac{d^{2}R}{dx^{2}} + \frac{1}{x}\frac{dR}{dx} + 
        \left(
            1 - \frac{\nu^{2}}{x^{2}}
        \right)R = 0
    \end{equation*}
    cuyas soluciones son las funciones de Bessel
    \begin{equation*}
        J_{\nu}(x) = 
        \left(
            \frac{x}{2}
        \right)^{\nu}
        \sum\limits_{j = 0}^{\infty}
        \frac{(-1)^{j}}{j!\Gamma(j+ \nu + 1)}
        \left(
            \frac{x}{2}
        \right)^{2j}
    \end{equation*}
    resulta que si $\nu$ no es entero, entonces $J_{\nu}$ y $J_{-\nu}$ son soluciones independientes\footnote{qué significa esto?}, pero si $\nu$ sí entero, entonces $J_{-\nu} = (-1)^{\nu}J_{\nu}$. Además se definen las funciones de Neumann como
    \begin{equation*}
        N_{\nu}(x) = 
        \frac{J_{\nu}(x)\cos{(\nu\pi)}-J_{-\nu}(x)}{\sin{(\nu\pi)}}
    \end{equation*}
    que también son soluciones a la ecuación de Bessel.\\
    \textbf{Obs:} Las funciones de Bessel divergen en el infinito y son finitas en el origen: $J_{\nu}\to\infty$ si $x \to \infty$\\
    \textbf{Obs:} Las funciones de Neumann divergen en el origen y son finitas en el infinito: $N_{\nu}\to \infty$ si $x \to 0$\\
    \indent Además, tanto $J_{\nu}$ como $N_{\nu}$ tienen infinitas raíces, para un dado $\nu$, se tiene $x_{\nu_{n}}$ tal que $J_{\nu}(x_{\nu_{n}}) = 0$.\\
    \indent En resumen, para el caso en que se tiene S-L en $\rho$ y en $\phi$, la base de soluciones son las funciones trigonométricas y las funciones de Bessel y de Neumann respectivamente. En la dirección $z$ se tendrán las exponenciales reales o senos y cosenos hiperbólicos.
    \item Ahora el caso con $\lambda = 0$, y tomo $\beta = \nu^{2}$ (es decir, pido S-L en $\phi$), la ecuación para $\rho$ queda
    \begin{equation*}
        \rho^{2}\frac{d^{2}R}{d\rho^{2}} + \rho\frac{dR}{d\rho} = \nu^{2}R
    \end{equation*}
    cuyas soluciones dependen de si $\nu = 0$ o $\nu \neq 0$.
        \subitem Si $\nu = 0$, entonces $R(\rho) = \{1, \ln{(\rho)} \}$
        \subitem si $\nu \neq 0 $, entonces $R(\rho) = \{\rho^{\nu}, \rho^{-\nu}\}$\\
    Y ninguna de estas últimas forman base. Entonces, con esta elección de parámetros, se tiene que hay S-L en $\phi$, con lo cual hay base de soluciones en esa dirección, que van a ser exponenciales complejas o senos y cosenos. En $z$, como se eligió $\lambda = 0$, las soluciones son soluciones lineales o constantes y en $\rho$ se obtuvieron soluciones que tampoco forman base, \textcolor{red}{entonces qué pasó acá? No se puede resolver el problema? qué quería mostrar?}
    \item Último caso, si $\lambda = -k^{2}$, entonces en $z$ se tiene base de soluciones de senos y cosenos (o exponenciales complejas), y si $\beta = \nu^{2}$, entonces habrá base de soluciones en $\phi$ también. Se tiene S-L en $z$ y $\phi$ y es el caso \textbf{(2)}. No habrá base de soluciones en $\rho$. Para resolver $\rho$ entonces, tomo el cambio de variable $x = ik\rho$ y la ecuación se reescribe como
    \begin{equation*}
        \frac{d^{2}R}{d\rho^{2}} 
        + 
        \frac{1}{\rho}\frac{dR}{d\rho}
        +
        \left(
            \frac{k^{2} -\nu^{2}}{\rho^{2}} 
        \right)R = 0
    \end{equation*}
    cuyas soluciones son $J_{\nu}(ik\rho)$. Se definen las funciones de Bessel y de Neumann modificadas como
    \begin{equation*}
        I_{\nu}(k\rho) = i^{-\nu}J_{\nu}(ik\rho)
    \end{equation*}
    y
    \begin{equation*}
        K_{\nu}(k\rho) = \frac{\pi}{2}i^{\nu+1}
        \left[
            J_{\nu}(ik\rho) +iN_{\nu}(ik\rho)
        \right]
    \end{equation*}
    Se puede ver que $I_{\nu}(k\rho)$ no tiene raíces reales en $k\rho > 0$, con lo cual no pueden formar base de soluciones. \\
    \textbf{Obs:} Las funciones $I_{\nu}$ divergen en el infinito y son finitas en el origen: $I_{\nu}\to\infty$ si $x \to \infty$\\
    \textbf{Obs:} Las funciones de $K_{\nu}$ divergen en el origen y son finitas en el infinito: $N_{\nu}\to \infty$ si $x \to 0$\\
\end{enumerate}
después viene un ejemplo de la resolución de la mitad de un cilindro, donde para las caras exteriores el potencial es $0$ y en las tapas superiores hay un potencial $V$. Entonces se tienen condiciones de contorno triviales (problema de Sturm Liouville) en la dirección $\rho$ y $\phi$, con lo cual hay base de soluciones (senos y cosenos para $\phi$ y las funciones de Bessell para $\rho$) y exponenciales reales para $z$, donde dada la simetría del problema termina resultando que se simplifica en un coseno hiperbólico\footnote{Ver apéndice 16.1 (aunque no me dio J3J3)}.


%%%%%%%%%%%%%%%%%%%%%%%%%%%%%%%%%%%%%%%%%%%%%%%%
%%%%%%%%%%%%%%%%%%%%%%%%%%%%%%%%%%%%%%%%%%%%%%%%


\subsection{Función de Green}
El problema que se quiere resolver es el de hallar el potencial de una distribución de carga (o corriente) arbitraria dentro de un dado recinto, usando la técnica de separación de variables, y con condiciones de contorno conocidas. Hay que resolver la ecuación de Poisson para una distribución que puede tener cualquier forma y no es posible separarla de forma fácil en regiones donde valga la ecuación de Laplace, más una distribución de carga superficial en contornos.\\
\indent Entonces lo que se propone es resolver dos problemas separados y luego sumarlos usando superposición. Uno de los problemas es hallar el potencial $\varphi_{1}$ del recinto, con paredes del contorno a tierra, pero con la densidad de carga (o corriente) arbitraria en su interior. El otro problema es resolver el potencial $\varphi_{2}$ para el mismo recinto, pero sin la densidad de carga (o corriente) y con un potencial $V$ fijo en el contorno.\\
\indent Para el caso en el que se tienen condiciones de contorno de Dirichlet, es decir, el potencial bien definido en los contornos, habría que resolver el siguiente problema
\begin{equation*}
    \left\{
        \begin{matrix}
            \varphi_{1}\ \mbox{tal que}\ 
            \nabla^{2}\varphi_{1} = -4\pi\rho\ \mbox{y}\ \varphi_{1}\big)_{s} = 0\\
            \\
            \varphi_{2}\ \mbox{tal que}\ 
            \nabla^{2}\varphi_{1} = 0\ \mbox{y}\ \varphi_{2}\big)_{s} = \varphi\big)_{2} = 0\\
        \end{matrix}
    \right.
\end{equation*}
De esta forma se tiene $\varphi = \varphi_{1} + \varphi_{2}$.\\
\indent Para resolver $\varphi_{1}$ no es necesario resolver la distribución más general posible, si no resolver el problema para el caso en el que se tiene una carga puntual $q$ de valor unidad, en cualquier posición $\textbf{r}'$ del recinto (que esto sí se puede separar en dos problemas de Laplace y un contorno con distribución superficial). Es decir que el resultado de este problema será una expresión que nos diga cuánto vale el potencial en todo punto $\textbf{r}$ del recinto, si la carga $q = 1$ se encuentra en cualquier posición $\textbf{r}'$.\\
\indent Para el caso en que se tienen condiciones de contorno de Dirichlet (es decir, se conoce el valor del potencial en la superficie del recinto), la solución al problema es la \textbf{Función de Green de Dirichlet} $G_{D}$. El problema a resolver es
\begin{equation*}
    \nabla^{2}G_{D}(\textbf{r},\textbf{r}') = -4\pi\delta (\textbf{r} - \textbf{r}')
\end{equation*}
con condiciones de contorno $G_{D}\big)_{s} = 0$, donde $s$ es la superficie del recinto de interés. Por ejemplo, si $s\to \infty$, la función de Green es
\begin{equation*}
    G_{D}(\textbf{r},\textbf{r}') = \frac{1}{|\textbf{r}-\textbf{r}'|}
\end{equation*}
dada la geometría del recinto, se obtiene de forma unívoca la función de Green y el potencial puede calcularse como
\begin{equation*}
    \varphi_{1}(\textbf{r}) = \int \rho(\textbf{r}')G_{D}(\textbf{r},\textbf{r}')dV'
\end{equation*}
La función de Green es simétrica en $\textbf{r}$, $\textbf{r}'$, es decir $G_{D}(\textbf{r},\textbf{r}') = G_{D}(\textbf{r}',\textbf{r})$. Para probar esta propiedad se usa la identidad de Green \eqref{ec:TeoremaGreen}, tomando $\phi = G_{D}(\textbf{x},\textbf{r})$ y $\psi = G_{D}(\textbf{x},\textbf{r}')$\footnote{Ver apéndice 17.4}.\\
\indent Todavía falta hallar $\varphi_{2}$ para resolver enteramente el problema. Usando nuevamente la función de Green y el Teorema de Green \eqref{ec:TeoremaGreen}, considerando $\phi = \varphi_{2}(\textbf{r})$ y $\psi = G_{D}(\textbf{r},\textbf{r}')$ se llega a que
\begin{equation*}
    \varphi_{2}(\textbf{r}) =
    -\frac{1}{4\pi}\int_{s(v)} \varphi_{2}(\textbf{r}')
    \frac{\partial}{\partial \textbf{n}}G_{D}(\textbf{r},\textbf{r}')\,dS'
\end{equation*}
ahora es cuestión de sumar ambas contribuciones
\begin{equation*}
    \varphi(\textbf{r})= 
    \int\limits_{V} \rho(\textbf{r}')G_{D}(\textbf{r},\textbf{r}')dV'
    -\frac{1}{4\pi}\int_{s(v)} \varphi(\textbf{r}')
    \frac{\partial}{\partial \textbf{n}}G_{D}(\textbf{r},\textbf{r}')\,dS'
\end{equation*}
la interpretación física de esta expresión es que, el primer término es el potencial electrostático dentro del recinto, cuando los contornos están conectados a tierra. El segundo término es la corrección a las condiciones de contorno para adecuar el potencial en los bordes.\\
\indent Ahora voy a mirar el caso en el que tengo condiciones de contorno de Neumann, es decir, conozco cuánto vale el campo eléctrico en la superficie del recinto, entonces el problema a resolver es de la pinta
\begin{equation*}
    \left\{
        \begin{matrix}
            \varphi_{1}\ \mbox{tal que}\ 
            \nabla^{2}\varphi_{1} = -4\pi\rho\ \mbox{y}\ 
            \frac{\partial\varphi_{1}}{\partial \textbf{n}}\big)_{s} = 0\\
            \\
            \varphi_{2}\ \mbox{tal que}\ 
            \nabla^{2}\varphi_{1} = 0\ \mbox{y}\ \frac{\partial\varphi_{2}}{\partial \textbf{n}}\big)_{s} = \frac{\partial\varphi}{\partial \textbf{n}}\big)_{2} = 0\\
        \end{matrix}
    \right.
\end{equation*}
y en este caso el resultado al que se llega \footnote{ver apéndice 18.7} es
\begin{equation*}
    \varphi(\textbf{r})
    = 
    \int\limits_{V}G_{N}(\textbf{r},\textbf{r}')
    \rho(\textbf{r}')dV'
    +
    \frac{1}{4\pi}
    \int\limits_{S(V)}G_{N}(\textbf{r},\textbf{r}')
    \frac{\partial \varphi}{\partial \textbf{n}}\,dS'
\end{equation*}
\textcolor{red}{Acá mininni dio un ejemplo que lo voy a hacer a mano a ver si vale la pena escribir algo importante acá}

%%%%%%% FIN CLASE 8 %%%%%%%%%%%%%

%%%%%%%%%%%%%%%%%%%%%%%%%%%%%%%%%%%%%%%%%%%%%%%%
%%%%%%%%%%%%%%%%%%%%%%%%%%%%%%%%%%%%%%%%%%%%%%%%


\subsection{Método de Imágenes}
El método de imágenes es un método alternativo para resolver el potencial eléctrico (o potencial vector) en un dado recinto con condiciones de contorno. Por ejemplo, para el caso electrostático, se tiene una distribución de carga arbitraria $\rho(\textbf{r})$ dentro de un recinto arbitrario de volumen $V$ y se conoce el valor del potencial $\varphi$ en la superficie del recinto.\\
\indent El procedimiento consisten en proponer una solución para el potencial en la región de interés de la siguiente forma
\begin{equation*}
    \varphi(\textbf{r}) = 
    \int\limits_{}\frac{\rho(\textbf{r}')}{|\textbf{r}-\textbf{r}'|}\,dV' + \varphi'
\end{equation*}
donde el primer término es la expresión general del potencial de una distribución de carga arbitraria en todo el espacio del recinto, y $\varphi'$ es una corrección del potencial dadas las condiciones de contorno. Entonces, tiene que pasar que
\begin{equation*}
    \left\{
        \begin{matrix}
            &\nabla^{2}\varphi' = 0\ \forall\ \textbf{r}\ \in\ V\\
            &\\
            &\varphi\Big)_{s} = V(\textbf{r}) 
            = \int\frac{\rho(\textbf{r}')}{|\textbf{r}-\textbf{r}'|}\,dV'\Big)_{s} + \varphi'\Big)_{s}
        \end{matrix}
    \right.
\end{equation*}
o sea, no puede haber distribución de carga debido al potencial $\varphi'$ que se agrega para corregir las condiciones de contorno, dentro del recinto y, además, el potencial total en la superficie es conocido. Con estas dos ecuaciones, se determina unívocamente el potencial $\varphi'$ en la superficie, simplemente despejándolo de la última expresión.\\
\indent Un posible candidato para $\varphi'$ podría ser simplemente
\begin{equation*}
    \varphi'(\textbf{r}) = \int\frac{\rho'(\textbf{r}')}{|\textbf{r} -\textbf{r}'|}\,dV'
\end{equation*}
donde $\rho'(\textbf{r}')$ sería la distribución de carga imagen, vale cero dentro del recinto y es distinta de cero fuera de él, y es tal que satisface las condiciones de contorno. Notar que la distribución de carga imagen no es unívoca, pero aún así la solución del potencial en el recinto de interés sí lo es. 
\textcolor{red}{Acá da un ejemplo de una carga puntual. Hacer en el apéndice}\\ 
\indent Acá (clase 10) mininni siguió hablando un poco más del método de imágenes y cómo usarlo para obtener la función de Green en un \textit{semiespacio}.\\
\indent Hasta ahora se tienen 3 formas para resolver la ecuación de Poisson (y Laplace) en un dado recinto cerrado: Separación de variables, el método de la función de Green (que en parte usa separación de variables) y el método de imágenes. Estos tres métodos son herramientas que deben usarse inteligentemente dado el problema en cuestión, de forma de arribar a una solución de la forma más sencilla posible. \footnote{Las cuentas que pasan acá, las haré después }


%%%%%%%%%%%%%%%%%%%%%%%%%%%%%%%%%%%%%%%%%%%%%%%%
%%%%%%%%%%%%%%%%%%%%%%%%%%%%%%%%%%%%%%%%%%%%%%%%


\subsection{Desarrollo Multipolar en esféricas}
Se supone el escenario (para electrostática) que se tiene una distribución de carga arbitraria, centrada en el origen, con tamaño característico $d$. Se quiere estudiar el potencial electrostático en $\textbf{r}$, para $r >> d$. Parto de la expresión \eqref{ec:DesarrolloEsfericas} para $q = 1$, que no es más que la función de Green para todo el espacio en coordenadas esféricas. El potencial viene dado por su forma más general,
\begin{equation*}
    \varphi(\textbf{r}) = 
    \int \frac{\rho(\textbf{r})}{|\textbf{r}-\textbf{r}'|}dV'
\end{equation*}
y quiero desarrollarlo para $r>>d$. Recordar que el desarrollo multipolar en cartesianas del potencial $\varphi$ podía escribirse como
\begin{equation*}
    \varphi(\textbf{r}) =
    \frac{Q}{r} 
    + \frac{\textbf{r}\cdot\textbf{r}}{r^{3}}
    + \frac{1}{2}Q_{ij}\frac{r_{i}r_{j}}{r^{5}}
    + \cdots
\end{equation*}
Uso entonces \eqref{ec:DesarrolloEsfericas} para reescribir $\varphi$, recordando la notación de $r_{>}$ y $r_{<}$:
\begin{equation*}
    \left\{
        \begin{matrix}
            r_{>} = max\{\textbf{r},\textbf{r}' \}\\
            \\
            r_{<} = min\{\textbf{r},\textbf{r}' \}
        \end{matrix}
    \right.
\end{equation*}
entonces, para $r>>d$ estoy en el caso en el que $r_{>} = r$ y $r_{<} = r'$. Reemplazando se tiene
\begin{equation*}
    \varphi(\textbf{r}) = 
    \sum\limits_{l,m} \frac{4\pi}{2l + 1}
    \frac{Y_{l,m}(\theta,\phi)}{r^{(l+1)}}
    \int Y_{l,m}^{*}(\theta',\phi')r'^{l}\rho(\textbf{r}')\,dV'
\end{equation*}
y defino también 
\begin{equation*}
    q_{l,m} = \int Y_{l,m}^{*}(\theta',\phi')r'^{l}\rho(\textbf{r}')\,dV'
\end{equation*}
son los momentos multipolares. Algunos ejemplos
\begin{equation*}
    q_{00} = \frac{1}{\sqrt{4\pi}}\int\rho(\textbf{r}')dV'
    = \frac{Q}{\sqrt{4\pi}}
    \Longrightarrow \varphi^{(0,0)} = \frac{Q}{r}
\end{equation*}
\begin{equation*}
    q_{10} = \sqrt{\frac{3}{4\pi}}\int\rho(\textbf{r}')r'\cos{(\theta')}dV'
    = \sqrt{\frac{3}{4\pi}}p_{z}
\end{equation*}
\begin{equation*}
    q_{1,\pm1} =\mp \sqrt{\frac{3}{8\pi}} \int\rho(\textbf{r}')r'\sin{(\theta')}e^{\pm i \phi}dV'
    = \mp \sqrt{\frac{3}{8\pi}}(p_{x}\pm ip_{y})
\end{equation*}
qué es lo importante del desarrollo multipolar en esféricas? que es la forma natural de expandir dadas las simentrías del problema, además de que los armónicos esféricos son la representación irreducible del grupo de rotaciones en tres dimensiones (SO3), es por eso que son la forma natural de expandirlos.


%%%%%%%%%%%%%%%%%%%%%%%%%%%%%%%%%%%%%%%%%%%%%%%%
%%%%%%%%%%%%%%%%%%%%%%%%%%%%%%%%%%%%%%%%%%%%%%%%
%%%%%%%%%%%%%%%%%%%%%%%%%%%%%%%%%%%%%%%%%%%%%%%%

\newpage
\section{Energía electrostática}
Parto de que la fuerza electrostática que sufre una carga $q$ en presencia de un campo eléctrico está definida como el producto del valor de la carga, por el campo eléctrico, $\textbf{F} = q \textbf{E}$. Además, como el campo eléctrico es un campo conservativo, es decir, que proviene de un potencial $-\nabla \varphi = \textbf{E}$. Entonces se puede calcular el trabajo para traer una carga $q$ desde el infinito hasta una posición $\textbf{r}$ en presencia de una distribución de carga como
\begin{equation*}
    W = \int\limits_{\infty}^{\textbf{r}}q\textbf{E}\cdot d\textbf{l} = q\varphi(\textbf{r})
\end{equation*}
es decir, el trabajo termina siendo independiente del camino, y sólo depende de los extremos.


%%%%%%%%%%%%%%%%%%%%%%%%%%%%%%%%%%%%%%%%%%%%%%%%
%%%%%%%%%%%%%%%%%%%%%%%%%%%%%%%%%%%%%%%%%%%%%%%%


\subsection{Energía de interacción}
Se define una energía potencial de interacción entre cargas: Sea una carga $q_{1}$ y una carga $q_{2}$, la energía potencial de interacción será
\begin{equation*}
    U = \frac{q_{1}q_{2}}{|\textbf{r}_{1}-\textbf{r}_{2}|}
\end{equation*}
donde lo que hay que observar de esta última es que si saco cualquiera de las dos cargas de esta expresión, lo que me termina quedando es el potencial debido a la carga restante, es decir
\begin{equation*}
    U = \frac{q_{1}q_{2}}{|\textbf{r}_{1}-\textbf{r}_{2}|}
    = \frac{q_{1}}{|\textbf{r}_{1}-\textbf{r}_{2}|}q_{2}
    = \varphi_{1}q_{2}
    = q_{1}\varphi_{2}
\end{equation*}
Esto permite poder escribir a la energía potencial de interacción simetrizada, es decir, considerando la mitad de las contribuciones de cada potencial
\begin{equation*}
    U = \frac{1}{2}
    \left(
        \varphi_{1}q_{2} + q_{1}\varphi_{2}
    \right)
\end{equation*}
Para el caso en que se tienen $N$ cargas interactuando, la energía potencial de interacción de las $N$ cargas será
\begin{equation*}
    U = \sum\limits_{i=1}^{N}
    \sum\limits_{j > 1}^{N}
    \frac{q_{i}q_{j}}{|\textbf{r}_{i}-\textbf{r}_{j}|}
\end{equation*}
que de la misma forma se puede escribir de forma simetrizada
\begin{equation*}
    U = \frac{1}{2}
    \sum\limits_{i=1}^{N}
    \sum\limits_{j>i}^{N}
    q_{i}\varphi_{j}
\end{equation*}
el próximo paso que se puede hacer una vez que se tiene la energía potencial de interacción de una distribución de $N$ cargas puntuales, es considerar el caso de una distribución de carga continua, donde la suma pasa a una integral en el volumen
\begin{equation*}
    U = \int \rho(\textbf{r})\varphi(\textbf{r})\,dV
\end{equation*}
\textbf{Obs:} A esta expresión se llegó usando puramente la expresión del potencial de Coulomb, que es una ley experimental. Además Mininni aclara que estrictamente hablando, la integral debería no considerar un diferencial del volumen de la distribución de carga porque no autointeractúa con ella misma, pero como esta porción se la puede tomar tan pequeña como se quiera, la contribución a la integral es 0.\\
\indent Considerando que $\nabla^{2}\varphi = 4\pi \rho$, entonces se puede escribir en la expresión anterior $\rho(\textbf{r}) = \frac{1}{4\pi}\nabla^{2}\varphi$. Reescribiendo la expresión que queda\footnote{Ver apéndice 23.9} y la energía puede escribirse finalmente como
\begin{equation*}
    U = \frac{1}{8\pi}\int |\textbf{E}|^{2}\,dV
\end{equation*}
de esta se define una densidad de energía
\begin{equation*}
    u = \frac{1}{8\pi}|\textbf{E}|^{2} 
\end{equation*}
que vale para todo el espacio pero para recintos finitos con condiciones de contorno dadas hay que tener cuidado.\\
\indent Lo más interesante de estos resultados es que la expresión sugiere que se puede almacenar energía electrostática en cada punto del espacio. Si antes el campo eléctrico estaba definido de forma tal de tener una carga de prueba en presencia del mismo, ahora se prescinde de esta carga de prueba porque esta energía almacenada puede utilizarse para realizar trabajo. Ya no es una mera definición matemática si no que cobra sentido físico tangible. 
\indent \textbf{Ejemplos que voy a hacer a mano}



%%%%%%%%%%%%%%%%%%%%%%%%%%%%%%%%%%%%%%%%%%%%%%%%
%%%%%%%%%%%%%%%%%%%%%%%%%%%%%%%%%%%%%%%%%%%%%%%%


\subsection{Energía de interacción con campo eléctrico externo}
Considero una distribución de carga uniforme, de tamaño finito, centrada en la posición $\textbf{r}_{0}$ y en presencia de un campo eléctrico externo $\textbf{E}_{ext}$. Si el tamaño característico de la distribución de carga $\rho$ es $d$, se supone que la longitud característica de la variación del potencial eléctrico asociado al campo eléctrico externo es mucho mayor que $d$. De esta forma, el potencial externo es prácticamente constante para la distribución de carga.\\
\indent La energía de interacción entre la distribución de carga y el campo eléctrico externo viene dada por
\begin{equation*}
    U_{int} = \int \rho(\textbf{r})\varphi_{ext}(\textbf{r})\,dV
\end{equation*}
la idea es realizar una expansión de esta expresión alrededor de $\textbf{r}_{0}$ e interpretar físicamente cada término resultante, a donde se llega\footnote{Ver apéndice 25.2}
\begin{equation*}
    U_{int} = 
    Q\varphi(\textbf{r}_{0}) - \textbf{P}\cdot \textbf{E}(\textbf{r}_{0}) - 
    \frac{1}{6}
    \left.
        \partial_{i} E_{j}
    \right)_{\textbf{r}_{0}}Q_{ij} + \cdots
\end{equation*}
donde la expresión quedó en función de los momentos multipolares del potencial eléctrico: $Q$ es la carga total, $\textbf{P}$ es el momento dipolar de la distribución de carga, $Q_{ij}$ es el tensor cuadrupolar.\\
\indent De acá se deduce como la carga eléctrica total de la distribución solo interactúa con el potencial eléctrico, el momento dipolar $\textbf{P}$ sólo interactúa con el campo eléctrico y por último el tensor cuadrupolar solo interactua con el gradiente del campo eléctrico.



%%%%%%%%%%%%%%%%%%%%%%%%%%%%%%%%%%%%%%%%%%%%%%%%
%%%%%%%%%%%%%%%%%%%%%%%%%%%%%%%%%%%%%%%%%%%%%%%%


\subsection{Energía electrostática en medios materiales}
Hasta ahora se había definido al potencial de una dada distribución de carga como menos el trabajo realizado para traer las cargas desde el infinito para armar la configuración. Para el caso en el que se tienen medios materiales (ej, dieléctricos) hay que considerar también el trabajo realizado por las cargas en polarizar el medio material.\\
\indent Como la energía es una función de estado termodinámica en el sentido más general posible, para evitar intentar calcularla a partir del mecanismo de traer las cargas desde el infinito, se va a intentar construir una energía que cumpla que sea función de estado para este problema.\\
\indent Considero un medio y una distribución de cargas libres fijada de antemano en sus posiciones, a la cual se le aumenta gradualmente desde $0$ hasta su valor total de carga, algo así como pasar de $\rho_{l} \longrightarrow \rho_{l}+\delta\rho_{l}$. La ventaja de este método es que si bien la polarización del medio cambia a medida que la intensidad de la carga cambia, la geometría de la polarización (si el medio es lineal) no cambia. En cambio al mover cargas esto deja de ser cierto.\\
\indent Así, la variación de la energía se puede escribir (versión no simetrizada)
\begin{equation*}
    \delta U = \int \delta \rho_{l} \varphi\,dV
\end{equation*}
si tomo que $\nabla\cdot (\delta \textbf{D}) = 4\pi \delta \rho_{l}$, despejo $\rho_{l}$ y reemplazo en la expresión de la energía se llega a \footnote{Hacer apéndice 25.4}
\begin{equation*}
    \delta U = \frac{1}{4\pi} 
    \int \textbf{E}\cdot \delta \textbf{D}\,dV
\end{equation*}
una vez que se tiene así escrita la expresión, se puede aumentar el valor del vector desplazamiento eléctrico $\textbf{D}$ de 0 a su valor final, entonces la integral se reescribe como
\begin{equation*}
    U = \frac{1}{4\pi}\int dV
    \int\limits_{0}^{\textbf{D}} 
    \textbf{E}\cdot \delta\textbf{D}
\end{equation*}
la cual no es una expresión fácil de resolver dado que el campo eléctrico $\textbf{E}$ es función del desplazamiento eléctrico $\textbf{D}$. Sin embargo, para medios lineales, el vector desplazamiento se puede escribir como $\textbf{D} = \bar{\bar{\varepsilon}}\textbf{E}$, finalmente\footnote{Ver apéndice 25.4}
\begin{equation*}
    U = \frac{1}{8\pi}\int \textbf{E}\cdot \textbf{D}\,dV
\end{equation*}


%%%%%%%%%%%%%%%%%%%%%%%%%%%%%%%%%%%%%%%%%%%%%%%%
%%%%%%%%%%%%%%%%%%%%%%%%%%%%%%%%%%%%%%%%%%%%%%%%
%clase 12


\subsubsection{Ejemplo}
Carga puntual colocada frente a un dieléctrico, sobre el eje $z$ a una distancia $d$. Se divide todo el espacio en dos regiones, una región $I$ donde está el dieléctrico y una región $II$ donde no hay dieléctrico y se encuentra solamente la carga puntual $q$. Se quiere ver el potencial electrostático generado por la presencia de la carga y la polarización del medio dieléctrico en en todo el espacio, o sea $\varphi^{(1)}$ y $\varphi^{(2)}$. Se va a formar una distribución superficial de cargas de polarización en el dieléctrico, dado que en el interior del dieléctrico la carga neta es nula porque se anulan entre sí los dipolos generados.\\
\indent Se quiere usar el método de imágenes para buscar la solución a este problema. Entonces de nuevo se propone una solución para la región $II$ de la pinta
\begin{equation*}
    \varphi^{(2)} = \frac{q}{|\textbf{r}-d\hat{z}|} + \varphi'
\end{equation*}
donde $\varphi'$ es un potencial auxiliar que viene a corregir el potencial de la configuración por la presencia del dieléctrico en las condiciones de contorno.\\
\indent Tiene que pasar que el Laplaciano del potencial $\varphi'$ sea nulo, es decir $\nabla^{2}\varphi' = 0$ en la región $II$ ($z > 0$). Además el potencial total debe anularse en el infinito, es decir $\varphi^{(2)} \to 0$ si $z \to \infty$. Además tiene que pasar que el potencial en la región $I$ (lado dieléctrico) $\varphi{(1)}$ tiene que ser igual al potencial fuera del dieléctrico (región $II$) $\varphi^{(2)}$ cuando $z = 0$, es decir, en la interfaz: $\varphi^{(1)}(z = 0) = \varphi^{(2)}(z = 0)$. Y por último, no hay cargas libres en la superficie, solo de polarización: 
\begin{equation*}
    \left(
        \textbf{D}^{(2)} - \textbf{D}^{(1)} 
    \right)\cdot \hat{n}
    = 4\pi \sigma_{l} = 0
\end{equation*}
lo cual implica que la componente normal de $\textbf{D}$ debe ser continua. Asumimos que le medio es lineal, isótropo y homogéneo.\\
\indent ¿Qué función hace el dieléctrico? Lo que hace el dieléctrico es básicamente disminuir/apantallar la intensidad del campo eléctrico, pero solo en la región del dieléctrico. Es decir: Supongo que tengo una carga puntual en el vacío y miro el potencial en todo el espacio, el campo eléctrico será de la pinta
\begin{equation*}
    \textbf{E}(\textbf{r}) = q\frac{\textbf{r}-\textbf{r}'}{|\textbf{r}-\textbf{r}'|^{3}}
\end{equation*}
ahora, si en vez de tener la carga puntual en el vacío, lleno todo el espacio con dieléctrico, el campo eléctrico resultante será de la pinta
\begin{equation*}
    \textbf{E}(\textbf{r}) = q'
    \frac{\textbf{r}-\textbf{r}'}{|\textbf{r}-\textbf{r}'|^{3}}
\end{equation*}
donde $q' < q$. Es decir, la forma del campo no se modifica en nada (suponiendo siempre un dieléctrico isótropo y homogéneo), solo disminuye su intensidad en el espacio ocupado por el dieléctrico.\\
\indent Entonces qué pasa cuando cuando tengo una región ocupada por dieléctrico y otra región donde hay vacío? Bueno, en la región donde hay dieléctrico se puede pensar que se tiene el campo eléctrico que se tendría si no hubiera medio material, pero con intensidad inferior o apantallado. Sin embargo, en la región donde hay vacío, el campo eléctrico se verá modificado en intensidad y forma debido a las cargas de polarización generadas en el medio material.\\
\indent Con esto en mente lo que se hace es, para la región $I$ que es el siemespacio ocupado por el dieléctrico, proponer una solución como si hubiera una carga puntual $q_{im}'$ en $z = d$ y que solo vale en $z < 0$ (que esto es considerar que dentro del dieléctrico, el campo eléctrico sólo se ve modificado en intensidad y no en forma, par a un medio lineal, isótropo y homogéneo). Para la región $II$, donde es vacío, lo que se hace es sumar una contribución al potencial de la carga puntual en el vacío, para corregir los efectos de las cargas de polarización. Para esto se coloca una carga imagen $q_{im}$, en un $z_{im}$. En definitiva lo que se propone es
\begin{equation*}
    \left\{
        \begin{matrix}
            \mbox{Región I:}\ 
            \varphi^{(1)} = \frac{q_{im}'}{|\textbf{r}-d\hat{z}|}\ \mbox{:}\ z < 0\\
            \\
            \mbox{Región II:}\ 
            \varphi^{(2)} = 
            \frac{q}{|\textbf{r}-d\hat{z}|} +
            \frac{q_{im}}{|\textbf{r}+z_{im}\hat{z}|}\ \mbox{:}\ z > 0
        \end{matrix}
    \right.
\end{equation*}
Con esto resta determinar 3 incógnitas: $q_{im}$, $q_{im}'$ y $z_{im}$. Parecería que tengo dos ecuaciones y 3 incógnitas, pero todavía tengo la ecuación de la condición de contorno, que me dice que $\varphi^{(1)}(z = 0) = \varphi^{(z = 0)}$. Entonces hay 3 ecuaciones y 3 incógnitas, se puede resolver el sistema y la solución\footnote{Ver apéndice HACER APÉNDICE} es
\textcolor{red}{O quizás no tengo 3 incógnitas, si no que proponga que $z_{im} = -d$ y de ahí impuse a mano una de las incógnitas y las demás se despejan de ahí}
\begin{equation*}
    \varphi(\textbf{r}) = 
    \left\{
        \begin{matrix}
            \frac{2}{1+\varepsilon}
            \frac{q}{|\textbf{r} - d\hat{z}|}
            & & z < 0\\
            & & \\
            \frac{q}{|\textbf{r}-d\hat{z}|} + \frac{1}{1+\varepsilon}\frac{q(1-\varepsilon)}{|\textbf{r}+d\hat{z}|} 
            & & z > 0
        \end{matrix}
    \right.
\end{equation*}
En el sistema CGS Gaussiano, el vacío tiene $\varepsilon = 1$ y cualquier medio material dieléctrico tiene $\varepsilon > 1$, entonces para $z<0$ efectivamente se tiene un campo eléctrico apantallado como se esperaba. En cambio en donde no hay dieléctrico, el campo eléctrico es modifica su forma debido a la contribución de las cargas de polarización.\\
\indent Ahora quisiera calcula la energía potencial de la configuración. Ahora que se conoce el potencial en todo el espacio puede calcularse.\\
\indent Quiero calcular la energía de interacción entre la carga puntual $q$ y el dieléctrico. Para eso considero $q \to \alpha q$ y tomo $\alpha$ de $0$ a $1$ (aumento el valor de la carga poco a poco). La energía de interacción venía dada por
\begin{equation*}
    \delta U = \int \delta \rho_{l} \varphi\,dV
\end{equation*}
con $\delta \rho_{l} = q \delta(\textbf{r}-d\hat{z})\delta \alpha$. Como interesa solo la interacción entre la carga y el dieléctrico, si al integrar yo pongo todo el potencial $\varphi^{(2)}$, terminaría considerando también la \textit{autointeracción de la carga consigo misma}, la cual diverge, pero igual puede considerase como constante y sacarse del valor de la energía, de esta forma
\begin{equation*}
    U = \int\limits_{0}^{1} d\alpha\,q
    \frac{\alpha q(1-\varepsilon)}{2d(1+\varepsilon)} 
    = \frac{q^{2}}{4d}
    \left(
        \frac{1-\varepsilon}{1+\varepsilon}
    \right)< 0
\end{equation*}
con lo cual, la energía de interacción es negativa, lo que significa que el dieléctrico atrae a la carga. ¿Por qué? ¿Qué pasa si muevo la carga de $d$ a $c$, con $c < d$? tengo en el denominador un número más chico, con lo cual la fracción se hace todavía más grande. Es decir, si acerco la carga al dieléctrico, estoy teniendo un valor de energía, en módulo, mayor. O sea que tengo una energía potencial de interacción cada vez más negativa. O sea que cada paso que la carga se acerca al dieléctrico, la energía se minimiza más y más. En la naturaleza los sistemas evolucionan a los estados de menor energía, por eso el resultado natural es que si dejo que la carga se mueva libremente, esta será atraída por el dieléctrico de forma de minimizar la energía de interacción. Para formalizar esto último se introduce el principio de los trabajos virtuales.

%%%%%%%%%%%%%%%%%%%%%%%%%%%%%%%%%%%%%%%%%%%%%%%%
%%%%%%%%%%%%%%%%%%%%%%%%%%%%%%%%%%%%%%%%%%%%%%%%

\subsection{Principio de trabajos virtuales}
Se considera una distribución de carga, con cargas puntuales, y se desplaza cada carga una distancia $\Delta \textbf{r}_{i}$. La variación de energía de la configuración debida al desplazamiento de estas cargas resulta ser
\begin{equation*}
    \Delta U = 
    U(q_{i},\textbf{r}+\Delta \textbf{r}_{i})
    - U(q_{i},\textbf{r}_{i})
\end{equation*}
y cuya diferencia se tiene que deber a la suma de los trabajos virtuales realizados al \textit{mover} las cargas.
\begin{equation*}
    \Delta U = - F_{i}^{(\alpha)}\Delta r_{i}^{(\alpha)}
\end{equation*}
donde el índice $i$ corre sobre todas las cargas y el índice $\alpha$ corre sobre las 3 direcciones del espacio. $F_{i}^{(\alpha)}$ es la fuerza sobre la carga $i$-ésima en la coordenada $\alpha$. Entonces puedo mirar la variación diferencial de energía y despejar la fuerza $F_{i}^{(\alpha)}$ de la siguiente forma
\begin{equation*}
    F_{i}^{(\alpha)} 
    = 
    \left.
        - \frac{\partial U}{\partial r_{i}^{(\alpha)}}
    \right)_{q}
    \Longrightarrow
    \textbf{F}_{i} = - \nabla_{i}U\big)_{q}
\end{equation*}
donde tengo variación de energía respecto a la posición a carga $q$ constante.\\
\indent Esta fue una descripción del principio de los trabajos virtuales cuando el valor de la carga $q$ es constante, y a partir de la cual se puede calcular la fuerza sobre cada carga. De acá se deduce que si hay una variación negativa de la energía, entonces hay una fuerza (atractiva(?????)).\\
\indent Ahora se quiere analizar lo mismo pero en vez de tener fijas las cargas, tener fijos los potenciales, lo cual se logra \textit{conectando baterías}. Termodinámicamente esto es como pasar de un sistema cerrado (que por ejemplo no intercambia calor) a un sistema abierto (en contacto con una fuente térmica que en este caso es una fuente de voltaje).\\
\indent Hay dos formas para calcular las fuerzas en este caso. Dado que la energía es una función de estado (es decir, no depende del camino) lo que se puede hacer es, cada vez que se va a mover un conductor con un dado potencial, se lo desconecta de la fuente y se lo mueve. Una vez en su nueva posición, se lo vuelve a conectar. Entonces el procedimiento es el siguiente: En el primer paso, desconecto los conductores de las baterías y los desplazo. Como fueron desconectados, la carga que había en los conductores no varió, con lo cual la variación de la energía de interacción por la variación de energía de cargas es nula. Sin embargo, al desplazar el conductor desconectado con cargas, el potencial en el mismo necesariamente cambió, así que la única contribución a la variación de la energía de interacción en el primer paso es debido a la variación en el potencial del conductor. El segundo paso es volver a conectar los conductores a las baterías, las cuales se encaran de restablecer el potencial a su valor original. De esta forma, la variación en la energía de interacción en el segundo paso es la variación en el potencial al conectar las baterías. De esta forma, la variación total de la energía de interacción termina siendo igual a menos la variación de la energía de interacción por desplazar los conductores a carga constante, que es el resultado que se obtuvo antes, de forma que la fuerza virtual termina siendo\footnote{Ver apéndice 24.3}
\begin{equation*}
    \textbf{F}_{i} = \nabla_{i}U\big)_{v}
\end{equation*}
la otra forma, también usando que $U$ es función de estado, es usando termodinámica. Como el caso de $q$ constante se puede considerar como un sistema cerrado, la energía entonces es función de las cargas $q$ y de las posiciones $\textbf{r}_{i}$, es decir $U = U(q_{i},\textbf{r}_{i})$. Lo que quiero ahora es escribir $U$ en función de un nuevo parámetro de control termodinámico, que va a ser el potencial $V_{i}$ de cada conductor. Para eso lo que se hace es una transformada de Legendre para la energía
\begin{equation*}
    \Tilde{U} = U -q_{i}V_{i}
\end{equation*}
a partir de la cual se vuelve a obtener el mismo resultado que antes\footnote{Ver apéndice 24.3}
\begin{equation*}
    \textbf{F}_{i} = \nabla_{i}U\big)_{v}
\end{equation*}


%%%%%%%%%%%%%%%%%%%%%%%%%%%%%%%%%%%%%%%%%%%%%%%%
%%%%%%%%%%%%%%%%%%%%%%%%%%%%%%%%%%%%%%%%%%%%%%%%


\subsection{Termodinámica de dieléctricos}
No se si vale la pena esto, asi que lo voy a saltear. Después viene dos ejemplos en la clase 13, que también los voy a saltear. Veré más adelante si vale la pena.\\
\indent En esta sección Mininni da dos ejemplos de dieléctricos en campos eléctricos, con la particularidad de que los dieléctricos son un líquido y una esfera deformable (gota).\\
\indent En el primer ejemplo se tiene el dieléctrico líquido de una dada densidad $\rho$ confinado entre 2 placas de un capacitor (solo importan la dirección $x$ (distancia entre las placas) y $z$ (altura del líquido) y se asume simetría de traslación en $y$). Se quiere estudiar qué pasa con el líquido dado un campo eléctrico $E$ fijo y gravedad. Para resolver el problema se va a escribir la energía libre de Helmholtz, que viene a ser la contribución dada por la energía potencial gravitatoria del líquido y la energía electrostática de la configuración. Como sabemos que un sistema en el equilibrio minimiza la energía libre de Helmholtz, una vez escrita esta hay que variarla para hallar el mínimo y ese mínimo dará la información necesaria del sistema.\\
\indent Resulta que el resultado que se obtiene es una expresión para la altura de la columna de dieléctrico líquido que depende tanto campo eléctrico aplicado al cuadrado, $E^{2}$ y de la densidad $\rho$ del dieléctrico: A medida que aumenta el campo eléctrico, la altura del líquido aumenta (se expande), a medida que la densidad de líquido aumenta, la altura del líquido disminuye.
\begin{equation*}
    h = \frac{E^{2}}{8\pi \rho g}(\varepsilon-1)
\end{equation*}
En el segundo ejemplo se quiere estudiar en qué forma de deforma la gota de dieléctrico inmersa en un campo eléctrico constante $\textbf{E} = E_{0}\hat{x}$. Nuevamente se sigue el mismo procedimiento: Hay que escribir la energía libre de Helmholtz y minimizarla para que corresponda a un estado de equilibrio del sistema. En este caso, para escribir la energía libre hay que tener en cuenta la energía de interacción electrostática y la energía proveniente de la tensión superficial de la gota, que intentará minimizar la superficie. Después de muchas cuentas de por medio se llega a
\begin{equation*}
    e^{2} = \frac{9}{12\pi}
    \left(
        \frac{\varepsilon-1}{\varepsilon+2}
    \right)^{2}
    \frac{RE_{0}^{2}}{\gamma}
\end{equation*}
donde $e$ es la excentricidad de la elipse\footnote{excentricidad: $c/a$, donde $c$ es la distancia del centro de la elipse al punto focal y $a$ es la distancia desde el centro de la elipse al borde más lejano} donde $\gamma$ es la tensión superficial de la gota. Entonces se ve que la deformación disminuye si la tensión superficial es grande y aumenta si el campo eléctrico externo es es mayor.


%%%%%%%%%%%%%%%%%%%%%%%%%%%%%%%%%%%%%%%%%%%%%%%%
%%%%%%%%%%%%%%%%%%%%%%%%%%%%%%%%%%%%%%%%%%%%%%%%
%%%%%%%%%%%%%%%%%%%%%%%%%%%%%%%%%%%%%%%%%%%%%%%%

\newpage
\section{Fenómenos dependientes del tiempo}
Acá arranca la segunda mitad de la materia, que trata de los campos electromagnéticos cuando estos varían en el tiempo y se desplazan en el espacio. La pregunta que se trata de responder en parte es si se puede seguir utilizando toda la maquinaria (o parte de ella) desarrollada en la parte anterior para resolver estos problemas. Para los casos electrostáticos y magnetostático se vio que se podía resolver una gran variedad de problemas utilizando las formulaciones de los potenciales eléctrico y potencial vector, que cumplen las ecuaciones de Poisson y Laplace dada un cierta configuración. Entonces, para campos dependientes del tiempo, sirve utilizar los potenciales? La respuesta es que sí, solo que de manera diferente. El problema surge de que ahora las ecuaciones de Maxwell tienen términos extra dependientes del tiempo.
%%%%%%%%%%%%%%%%%%%%%%%%%%%%%%%%%%%%%%%%%%%%%%%%
%%%%%%%%%%%%%%%%%%%%%%%%%%%%%%%%%%%%%%%%%%%%%%%%

\subsection{Ley de inducción de Faraday}
La ley de Faraday surge de muchos experimentos que realizó estudiando espiras en las que circulaba corriente. La ley de Faraday dice que la \textit{f.e.m}, o fuerza electromotriz inducida en una espira conductora es directamente proporcional a la variación temporal del flujo de campo magnético que la atraviesa. Más tarde, Lenz corrigió la ley, formulando la ley de Lenz, que dice que la fuerza electromotriz inducida en una espira conductora cerrada es tal que el campo magnético producida por ella se opone al campo magnético que la generó.
\begin{equation*}
    \varepsilon = -\frac{1}{c}\frac{d \Phi}{dt}
\end{equation*}
donde $\varepsilon$ es la \textit{f.e.m}. La ley de Faraday se escribía
\begin{equation*}
    \nabla \times \textbf{E} = -\frac{1}{c}\frac{\partial \textbf{B}}{\partial t}
\end{equation*}
mientras que la ley de Ampere, que relacionaba el rotor del campo magnético con las corrientes, por sí sola tenía el problema que de violar la conservación de carga, de modo que Maxwell le agrega el término de las corrientes de desplazamiento y la ley queda
\begin{equation*}
    \nabla \times \textbf{H} = 
    \frac{4\pi}{c}\textbf{j}_{l} 
    + \frac{1}{c}\frac{\partial \textbf{D}}{\partial t}
\end{equation*}
entonces, todo el electromagnetismo se resume en las ecuaciones de Maxwell, tanto para el vacío como para medios materiales y donde 2 de ellas son ecuaciones con fuentes y 2 son ecuaciones sin fuentes.

\subsection{Ecuaciones de Maxwell}
Estas se resumen en
\begin{center}
\textbf{En el vacío}
\end{center}
\begin{equation*}
    \left\{
        \begin{matrix}
            \nabla \cdot \textbf{E} = 4\pi\rho,
            &
            \nabla \times \textbf{B} = 
            \frac{4\pi}{c}\textbf{j}
            + \frac{1}{c}\frac{\partial \textbf{E}}{\partial t}
            &
            \longleftarrow
            \mbox{Con fuentes}\\
            & & \\
            \nabla \cdot \textbf{B} = 0,
            &
            \nabla \times \textbf{E} =
            -\frac{1}{c}\frac{\partial \textbf{B}}{\partial t}
            &
            \longleftarrow \mbox{Sin fuentes}
        \end{matrix}
    \right.
\end{equation*}

\begin{center}
\textbf{En medios materiales}
\end{center}
\begin{equation*}
    \left\{
        \begin{matrix}
            \nabla \cdot \textbf{D} = 4\pi\rho_{l},
            &
            \nabla \times \textbf{H} = 
            \frac{4\pi}{c}\textbf{j}_{l}
            + \frac{1}{c}\frac{\partial \textbf{D}}{\partial t}
            &
            \longleftarrow
            \mbox{Con fuentes}\\
            & & \\
            \nabla \cdot \textbf{B} = 0,
            &
            \nabla \times \textbf{E} =
            -\frac{1}{c}\frac{\partial \textbf{B}}{\partial t}
            &
            \longleftarrow \mbox{Sin fuentes}
        \end{matrix}
    \right.
\end{equation*}
donde las fuentes son las densidades de carga y corriente. Usando que tanto el campo magnético $\textbf{B}$ como el campo eléctrico provienen de potenciales, se pueden reescribir las ecuaciones de Maxwell en función de estos, en particular $\textbf{B} = \nabla \times \textbf{A}$, de forma de buscar las ecuaciones equivalentes a la ecuación de Poisson en estática. Haciendo esto y un par de operaciones\footnote{Ver apéndice 32.5 } se llega a
\begin{equation*}
    \left\{
        \begin{matrix}
            -\nabla^{2}\varphi - \frac{1}{c}\frac{\partial}{\partial t}
            \nabla \cdot \textbf{A}
            = 4\pi\rho\\
            \\
            \nabla (\nabla \cdot \textbf{A})
            - \nabla^{2}\textbf{A} = \frac{4\pi}{c}\textbf{j}
            - \nabla 
            \left(
                \frac{1}{c}\frac{\partial \varphi}{\partial t}
            \right)
            -\frac{1}{c^{2}}\frac{\partial^{2}\textbf{A}}{\partial t^{2}}
        \end{matrix}
    \right.
        \label{ec:MaxwellyPotencialesenTiempo}
\end{equation*}
y hay libertad de gauge, es decir, elegir el valor de $\nabla \cdot \textbf{A}$. Hay dos gauge más comunes, que se elige entre uno y otro dependiendo del problema a tratarse. El gauge de Coulomb, que es útil cuando se tiene un sistema de referencia fijo al medio material o el gauge de Lorenz, que es invariante de Lorentz, con lo cual es útil para sistemas de referencia relativistas o cuando o hay medios materiales y se trabaja en vacío, haciendo que no haya una posición privilegiada del espacio donde uno se quiera posicionar.
\subsubsection{Gauge de Lorenz}
Consiste en tomar la divergencia del potencial vector como
\begin{equation*}
    \nabla \cdot \textbf{A} = -\frac{1}{c}\frac{\partial \varphi}{\partial t}
\end{equation*}
de modo que las ecuaciones \eqref{ec:MaxwellyPotencialesenTiempo} quedan\footnote{Ver apéndice 32.6}
\begin{equation*}
    \left\{
        \begin{matrix}
            \frac{1}{c^{2}}\frac{\partial^{2}\varphi}{\partial t^{2}} - \nabla^{2}\varphi = 4\pi\rho\\
            \\
            \frac{1}{c^{2}}\frac{\partial^{2}\textbf{A}}{\partial t^{2}} - \nabla^{2}\textbf{A} = 4\pi\textbf{j}
        \end{matrix}
    \right.
\end{equation*}
que son dos ecuaciones desacopladas para $\varphi$ y $\textbf{A}$. Son las ecuaciones de Maxwell escritas en el gauge de Lorenz. Son ecuaciones de onda de onda para el potencial escalar y para el potencial vector. Más adelante se va a ver que las ecuaciones de los campos también son ecuaciones de onda. Por otro lado se ve que la velocidad de propagación de las ondas es la velocidad de la luz $c$.\\
\indent También está bueno notar que hay infinitos gauge de Lorenz, dado que se le puede sumar al potencial vector un gradiente de una dada magnitud $\chi$ y al potencial escalar una derivada temporal y eso no modifica las ecuaciones, es decir, puedo definir $\textbf{A}' = \textbf{A} + \nabla \chi$ y $\varphi' = \varphi - \frac{1}{c}\frac{\partial \chi}{\partial t}$ y con esto $\nabla \times \textbf{A}' = \textbf{B}$ como antes y $-\nabla \varphi' - \frac{1}{c}\frac{\partial \textbf{A}'}{\partial t} = \textbf{E}$ como antes.


%%%%%%%%%%%%%%%%%%%%%%%%%%%%%%%%%%%%%%%%%%%
%%%%%%%%%%%%%%%%%%%%%%%%%%%%%%%%%%%%%%%%%%%

\subsubsection{Gauge de Coulomb}
Consiste en tomar que la divergencia del potencial vector sea 0, es decir
\begin{equation*}
    \nabla\cdot \textbf{A} = 0
\end{equation*}
y las ecuaciones ahora quedan\footnote{Ver apéndice 32.8}

\begin{equation*}
    \left\{
        \begin{matrix}
            \nabla^{2}\varphi = -4\pi\rho \\
            \\
            -\nabla^{2}\textbf{A} 
            + \frac{1}{c^{2}}\frac{\partial^{2}\textbf{A}}{\partial t^{2}} = \frac{4\pi}{c}\textbf{j} 
            -\nabla 
            \left(
                \frac{1}{c}\frac{\partial \varphi}{\partial t}
            \right)
        \end{matrix}
    \right.
\end{equation*}
En este gauge las ecuaciones quedan acopladas. Notar que a diferencia del gauge de Lorenz, en el gauge de Coulomb se obtiene una ecuación de Poisson y una ecuación de ondas con fuentes, con lo cual, la solución para $\varphi$, por ejemplo, en todo el espacio es el potencial electrostático conocido, solo que $\rho$ depende del tiempo ahora.\\
\indent Ahora tomo la segunda ecuación (la acoplada) y a la corriente la descompongo según su componente longitudinal y transversal, es decir, $\textbf{j} = \textbf{j}_{L} + \textbf{j}_{T}$. De este modo, una componente tiene rotor nulo mientras que la otra tiene divergencia nula, es decir, una es irrotacional y la otra es solenoidal
\begin{equation*}
    \left\{
        \begin{matrix}
            \nabla \times \textbf{j}_{L} = 0\quad \mbox{(Irrotacional)}\\
            \\
            \nabla \cdot \textbf{j}_{T} = 0\quad
            \mbox{(Solenoidal)}
        \end{matrix}
    \right.
\end{equation*}
esto se puede usar gracias al teorema de Helmholtz. Reescribiendo estas corrientes en función de la corriente total y reemplazando en la otra expresión, además notando que por la conservación de la carga se puede hacer aparecer a la densidad de carga $\rho$ se llega\footnote{Ver apéndice 32.8}
\begin{equation*}
    \frac{1}{c^{2}}\frac{\partial^{2}\textbf{A}}{\partial t^{2}} - \nabla^{2}\textbf{A} = 
    \frac{4\pi}{c}\textbf{j}_{T}
\end{equation*}
de forma que se pudo recuperar una ecuación de ondas con fuente, desacoplada de $\varphi$ para el potencial vector $\textbf{A}$, donde la fuente es sólo la parte transversal de la densidad de corriente que no produce acumulación de cargas.\\
\indent Ahora se tienen ecuaciones de la pinta 
\begin{equation}
    \nabla^{2}\psi- \frac{1}{c^{2}}\frac{\partial^{2}\psi}{\partial t^{2}} = -4\pi f(\textbf{r},t)
        \label{ec:MaxwellEcsGaugeLorenz}
\end{equation}
Lo que hace falta para resolverlas ahora son condiciones iniciales y de contorno y además hay que tener en cuenta que los volúmenes ahora evolucionan en el tiempo.

%Fin clase 13 (ya la vi)
%%%%%%%%%%%%%%%%%%%%%%%%%%%%%%%%%%%%%%%%%%%
%%%%%%%%%%%%%%%%%%%%%%%%%%%%%%%%%%%%%%%%%%%
%Inicio clase 14 (no la vi)

\subsection{Balance de energía para un sistema de cargas}
Lo primero que se hace para intentar resolver problemas que analíticamente son muy difíciles de encarar, es buscar las leyes de conservación (integrales primeras de las ecuaciones de movimiento). Para el caso continuo, donde por ejemplo un campo puede llenar todo el espacio, se usa que el balance de energía se mantiene constante y no que, por ejemplo, se mantiene constante la energía mecánica del sistema. Esto es porque la energía en sus diferentes formas puede \textit{fluir} por el espacio, de forma que la energía mecánica podría variar en el tiempo para un dado punto del espacio, sin embargo, el balance total de energía en todo el recinto se mantiene constante.\\
\indent Vamos a considerar el balance de energía para un sistema de cargas. Para una carga puntual inmersa en un campo electromagnético en una dada región del espacio, la fuerza que experimenta la carga viene dada por la fuerza de Lorentz
\begin{equation*}
    \textbf{F} = q
    \left(
        \textbf{E} + \frac{\textbf{v}}{c}\times \textbf{B}
    \right)
\end{equation*}
donde $\textbf{v}$ es la velocidad de la carga $q$. La generalización para distribuciones de carga da una densidad de fuerza $f$ como sigue
\begin{equation*}
    \textbf{f} = 
    \rho\textbf{E} + \frac{\textbf{j}}{c}\times \textbf{B}
\end{equation*}
Se quiere calcular el trabajo que hacen los campos electromagnéticos sobre un sistema de cargas por unidad de tiempo, es decir, la potencia entregada al sistema. El diferencial de trabajo $dW$ es el producto de la fuerza de Lorentz por el diferencial de longitud $\textbf{F}\cdot d\textbf{r}$, y para tener la potencia hay que dividir por un diferencial de tiempo $dt$, es decir
\begin{equation*}
    \frac{dW}{dt} 
    = \textbf{F}\cdot \frac{d\textbf{r}}{dt}
    = 
    q\left(
        \textbf{E} + \frac{\textbf{v}}{c}\times \textbf{B}
    \right)
    \cdot \frac{d\textbf{r}}{dt}
    = q\,\textbf{E}\cdot \textbf{v}
\end{equation*}
donde $d\textbf{r}/dt = \textbf{v}$ y la velocidad es perpendicular a $\textbf{v}\times \textbf{B}$, con lo cual ese producto interno es nulo. La potencia termina siendo el producto de la carga, por el campo por la velocidad de la carga, o en general para distribuciones de carga en el caso continuo será
\begin{equation}
    \frac{dW}{dt} = \int \textbf{j}\cdot \textbf{E}\,dV
        \label{ec:dW/dt-1}
\end{equation}
Lo que se busca es escribir una ecuación de balance de energía para este sistema. Es decir, una ecuación que diga que no se puede ni crear ni destruir una dada magnitud o que su variación en el recinto es igual al flujo de ella a través de su superficie. Un ejemplo de esto era la ecuación de conservación de carga o de continuidad. Se quiere obtener algo semejante a eso, pero para la energía. \\
\indent Despejando de la ley de Ampere (para medios materiales), se puede reemplazar la densidad de corriente. Operando sobre los productos vectoriales se busca llevar las expresiones a la forma de divergencias. Además, asumiendo que los medios son lineales, finalmente se puede llegar a\footnote{Ver apéndice 30.2}
\begin{equation}
    \frac{dW}{dt} 
    = -\int\limits_{S(V)}
    \frac{c}{4\pi}
    \left(
        \textbf{E}\times\textbf{H}
    \right)\cdot d\textbf{A}
    -
    \frac{d}{dt}
    \int\limits_{V}
    \frac{1}{8\pi}
    \left(
        \textbf{B}\cdot\textbf{H} + \textbf{E}\cdot \textbf{D}
    \right)\,dV
        \label{ec:dW/dt-2}
\end{equation}
como estaba buscando una expresión de balance de energía, algo de la pinta de \textit{variación temporal de una magnitud} más \textit{flujo de otra magnitud} igual a cero, necesariamente necesitaba que aparezca un flujo, que viene a ser la primera integral en superficie. Es el flujo de la magnitud $\textbf{E}\times \textbf{H}$ en la superficie $S(V)$. Por otro lado, el segundo término es el correspondiente a la variación temporal que estaba buscando. Pero qué es esa magnitud entre paréntesis? Claramente es parte de la energía. En el caso estático, $\textbf{E}\cdot \textbf{D}$ es la energía potencial electrostática mientras que análogamente $\textbf{B}\cdot\textbf{H}$ es la energía potencial magnetostática. Es decir, en el caso dinámico, la suma de esas dos contribuciones es la densidad de energía del campo electromagnético, con lo cual se define
\begin{equation}
    \textbf{u} = \frac{1}{8\pi}
    \left(
        \textbf{B}\cdot\textbf{H} + \textbf{E}\cdot \textbf{D}
    \right)
        \label{ec:DensidadEnergiaEM}
\end{equation}
como la densidad de energía electromagnética. Al integrarla en todo el espacio será la energía electromagnética total. El trabajo $W$ puede convertirse en energía mecánica o calor. Si no hay disipación, la variación de energía mecánica es igual al trabajo, con lo cual $W_{ext}$ se convertirá en energía mecánica $U_{mec}$ 
\begin{equation*}
    \frac{dW}{dt} = \frac{dU_{mec}}{dt} = - \int\limits_{S(V)}
    \textbf{S}\cdot d\textbf{A} - \frac{dU_{EM}}{dt}
\end{equation*}
con $U_{EM} = \int \textbf{u}\,dV$ y con $\textbf{S}$ el vector de Poynting (densidad de flujo), definido como
\begin{equation}
    \textbf{S} = \frac{c}{4\pi}\textbf{E}\times \textbf{H}
        \label{ec:VectorPoynting}
\end{equation}
Se puede escribir la variación de energía total de un sistema cerrado como
\begin{equation*}
    \frac{dU_{tot}}{dt} = 
    - \int\limits_{S(V)}\textbf{S}\cdot d\textbf{A}
\end{equation*}
Entonces, si se tiene un dado recinto cerrado, la energía total en su interior solo puede variar si hay flujo de energía electromagnética a través de las paredes del recinto. Ese flujo de energía es determinado por $\textbf{S}$, con lo cual el vector de Poynting es el flujo de energía electromagnética por unidad de área y unidad de tiempo.\\
\indent También se puede escribir la forma diferencial, combinando las expresiones \eqref{ec:dW/dt-1}, \eqref{ec:dW/dt-2}, \eqref{ec:DensidadEnergiaEM} y \eqref{ec:VectorPoynting}, que equivale a la ecuación de conservación de la energía como
\begin{equation*}
    \frac{\partial \textbf{u}}{\partial t} + \nabla \cdot \textbf{S} = -\textbf{j}\cdot \textbf{E}
\end{equation*}
\textbf{Obs:} Esta expresión implica que si no hay disipación $\textbf{j}\cdot \textbf{E}$, la energía total (mecánica más electromagnética) se conserva. Además, si en una dada región cerrada varía la cantidad de energía total, es porque el campo electromagnético llevó energía hacia el exterior del recinto.\\
\textbf{Obs:} Esta expresión tiene una implicación importante que es que dado el vector de Poynting, proporcional a $\textbf{E}\times \textbf{H}$, como ambos campos decaen como $\sim 1/r$, el producto de ambos decae como $\sim 1/r^{2}$. Ahora, como en la expresión tengo la divergencia del vector de Poynting, es decir $\nabla \cdot \textbf{S} \sim \nabla \cdot (\textbf{E}\times \textbf{H})$, si aplico el teorema de Gauss e integro en una esfera cuyo radio tiende a infinito, como en esféricas me aparece del jacobiano un $r^{2}$, el integrando de esa expresión no necesariamente se hace $0$ a medida que el radio de la esfera de integración tiende a infinito. Esto dice que el vector de Poynting puede transportar energía al infinito.

%%%%%%%%%%%%%%%%%%%%%%%%%%%%%%%%%%%%%%%%%%%
%%%%%%%%%%%%%%%%%%%%%%%%%%%%%%%%%%%%%%%%%%%

\subsection{Balance de impulso}
Ahora se quiere hacer lo propio con el impulso lineal. Voy a considerar para este caso que no hay medios materiales, se trabaja en el vacío (porque mininni dice que si no las cuentas son mucho más difíciles). Se quiere hallar una ecuación de balance del impulso lineal. Se puede escribir la fuerza de Lorentz como la variación temporal del impulso mecánico
\begin{equation*}
    \frac{d\textbf{p}_{mec}}{dt} = \textbf{F}
    = \int 
    \left(
        \rho \textbf{E} + \frac{\textbf{j}}{c}\times \textbf{B}
    \right)\,dV
\end{equation*}
Nuevamente se quiere buscar una expresión que relacione una variación temporal de una mangnitud respecto de del flujo de otra magnitud a través de las paredes del recinto. Se puede escribir todo en términos de los campos, cambiando las densidades de carga y de corriente convenientemente, agrupando términos y haciendo cuentas, se llega a\footnote{Ver apéndice 31.4} 
\begin{equation*}
    \frac{d\textbf{p}_{mec}}{dt} + \frac{d\textbf{p}_{EM}}{dt} =
    \frac{1}{4\pi}
    \int\limits_{V}
    \left[
        \textbf{E}
        \left(
            \nabla \cdot \textbf{E}
        \right)
        +
        \left(
            \nabla \times \textbf{E}
        \right) \times \textbf{E}
        +
        \left(
            \nabla \times \textbf{B}
        \right) \times \textbf{B}
    \right]\,dV
\end{equation*}
y para expresar una conservación quiero escribir el término de la derecha como una divergencia, operando con índices\footnote{Ver apéndice 31.4} y se define el tensor de Maxwell
\begin{equation}
    T_{ij} = \frac{1}{4\pi}
    \left[
        E_{i}E_{j}
        +
        B_{i}B_{j}
        -
        \frac{\delta_{ij}}{2}
        \left(
            E^{2} + B^{2}
        \right)
    \right]
        \label{ec:TensorMaxwell}
\end{equation}
En definitiva, el tensor de Maxwell $T_{ij}$ representa el flujo de impulso con componente $i$-ésima que atraviesa una superficie con normal $j$-ésima, por unidad de tiempo y de área.\\
\indent Es importante notar que este es un resultado \textbf{clásico}. En ningún momento se cuantizó el campo electromagnético, de forma que no se está pensando en fotones, si bien también es correcto. Pero se deduce clásicamente que el campo electromagnético puede transmitir tanto energía como impulso, un resultado muy sorprendente.\\
\indent Para el caso tanto electrostático como magnetostático, la integral de $T_{ij}$ representa los esfuerzos en un diferencial de volumen (compresión y cizalla), dado que la fuerza es la variación temporal del impulso mecánico y el impulso electromagnético es cero. Este tensor es útil para calcular las fuerzas sobre una dada configuración de cargas o de corrientes. Los términos diagonales serán los términos de presión y tensión, mientras que los términos fuera de la diagonal son los esfuerzos de cizalla.\\
\indent Finalmente, se tiene
\begin{equation}
    \frac{d\textbf{p}_{mec}}{dt} + \frac{d\textbf{p}_{EM}}{dt} =
    \int\limits_{V} \nabla\cdot \bar{\bar{T}}\,dV = 
    \int\limits_{S(V)} \bar{\bar{T}}\cdot d\textbf{S}
        \label{ec:BalanceImpulsoEM}
\end{equation}
de esta se sigue que la variación total de impulso (mecánico más electromagnético) dentro de un recinto cerrado es igual al flujo de impulso ($T_{ij}dS_{i}$) que atraviesa las paredes.


%%%%%%%%%%%%%%%%%%%%%%%%%%%%%%%%%%%%%%%%%%%
%%%%%%%%%%%%%%%%%%%%%%%%%%%%%%%%%%%%%%%%%%%


\subsection{Balance de impulso angular}
\textbf{Sin hacer las cuentas, ver los apéndice}.\\
\indent Se define el tensor $M_{kj}$
\begin{equation*}
    M_{kj} = \varepsilon_{kli}r_{l}T_{ij}
    \equiv \textbf{r}\times \bar{\bar{T}}
\end{equation*}
y la conservación del impulso angular queda
\begin{equation*}
    \bar{\tau} + \frac{d\textbf{L}_{EM}}{dt}
    = 
    \int\limits_{V}
    \left(
        \partial_{j}M_{kj}\hat{k}\,dV
    \right)
    =
    \int\limits_{S(V)} \bar{\bar{M}}\cdot d\textbf{S}
\end{equation*}
y en ausencia de fuerzas externas, $\bar{\tau} = \frac{d\textbf{L}_{mec}}{dt}$, con lo cual
\begin{equation*}
    \frac{d\textbf{L}_{mec}}{dt}
    + \frac{d\textbf{L}_{EM}}{dt}
    = \int\limits_{S(V)}\bar{\bar{M}}\cdot d\textbf{S}
\end{equation*}
\textcolor{red}{Acá viene un ejemplo de cálculo de na fuerza sobre una dada distribución de cargas.}\\
\indent En resumen, de las integrales primeras de las ecuaciones de Maxwell se encuentra que si la energía, el impulso lineal y el impulso angular se conservan, entonces hay que aceptar que el campo electromagnético almacena impulso y energía. O sea que no se tiene solo energía mecánica, impulso lineal y angular mecánico, si no que hay una contribución electromagnética a estas cantidades.\\
\indent Además, esas cantidades se conservan, pero la expresión de la conservación es una expresión de la forma de ecuación de balance: la variación en el tiempo de una magnitud es igual a cuánto entra o cuánto sale de las paredes del recinto de dicha magnitud. Por otro lado el campo de radiación puede llevar estas cantidades hasta el infinito.


%%%%%%%%%%%%%%%%%%%%%%%%%%%%%%%%%%%%%%%%%%%
%%%%%%%%%%%%%%%%%%%%%%%%%%%%%%%%%%%%%%%%%%%
%inicio clase 15

\subsection{Funciones de Green avanzadas y retardadas}
Vuelvo a las ecuaciones de Maxwell para los potenciales que eran de la forma
\begin{equation*}
    \nabla^{2}\psi- \frac{1}{c^{2}}\frac{\partial^{2}\psi}{\partial t^{2}} = -4\pi f(\textbf{r},t)
\end{equation*}
tanto para el gauge de Coulomb como para el gauge de Lorentz. Hay que lograr resolver este tipo de ecuaciones que son ecuaciones de onda dependientes del tiempo. Para intentar resolverlas lo que se hace es usar transformadas de Fourier, de forma de bajarle complejidad a la ecuación diferencial. Si tomo las transformadas y las antitransformadas como
\begin{multicols}{2}
\begin{equation*}
    \hat{\psi}(\textbf{r},\omega) = \int\psi(\textbf{r},t)e^{i\omega t}\,dt
\end{equation*}
\begin{equation*}
    \hat{f}(\textbf{r},\omega) = \int f(\textbf{r},t)e^{i\omega t}\,dt
\end{equation*}
\begin{equation*}
    \psi(\textbf{r},t) = \frac{1}{2\pi} \int\psi(\textbf{r},\omega)e^{-i\omega t}\,d\omega
\end{equation*}
\begin{equation*}
    f(\textbf{r},t) = 
    \frac{1}{2\pi}
    \int f(\textbf{r},\omega)e^{-i\omega t}\,d\omega
\end{equation*}
\end{multicols}
entonces, la expresión general que hay que resolver queda
\begin{equation*}
    \nabla^{2}\hat{\psi}(\textbf{r},\omega) +             \left(    
        \frac{\omega}{c}
    \right)^{2}\hat{\psi}(\textbf{r},\omega)
    = -4\pi\hat{f}(\textbf{r},\omega)
\end{equation*}
donde las únicas derivadas que hay son espaciales y las frecuencias quedan desacopladas.\\
\indent Ahora se introducen las funciones de Green. En el caso estático lo que se quería resolver en general era una ecuación de Poisson de la forma de $\nabla^{2}\psi = -4\pi f$, donde $f$ era la fuente, en este caso se tiene una ecuación de la forma $\nabla^{2}\psi - \mbox{algo}\psi = - 4\pi f$. Es muy similar a diferencia de un nuevo término. Entonces la idea es proceder de forma similar a Poisson en el caso estático: Cambio la fuente (en general extensa) por una fuente puntual (en este caso va a ser una delta). Tomando $k^{2} = \omega^{2}/c^{2}$ y escribiendo en términos de la función de Green que se quiere hallar se tiene
\begin{equation}
    \left\{ 
        \begin{matrix}
            \left(
                \nabla^{2} + k^{2}
            \right)
            G_{k}(\textbf{r},\textbf{r}')
            = -4\pi\delta(\textbf{r}-\textbf{r}')\\
            \\
            G_{k}\to 0\ \mbox{cuando}\ |\textbf{r}|\to \infty
        \end{matrix}
    \right.
        \label{ec:GreenEspacioFourier}
\end{equation}
O sea, estoy pidiendo que el potencial tienda a cero cuando $|\textbf{r}|$ se va a infinito. Notar que si bien se redujo la fuente a una delta de Dirac, al haberlo hecho en el espacio de Fourier, esto no implica que en el espacio real se tiene un problema donde la fuente es constante para todo tiempo. Al contrario, como la fuente es constante en el espacio de Fourier, en el espacio real la fuente está modulada por $e^{i\omega t}$, de forma que esta se \textit{prende y apaga} con frecuencia $\omega$.\\
\indent Si buscamos una solución particular, tomando $G_{k}(\textbf{r},\textbf{r}') = G_{k}(\textbf{R})$, con $\textbf{R} = \textbf{r}-\textbf{r}'$, dado que el problema es esféricamente simétrico (estoy considerando el problema en todo el espacio y no en un recinto cerrado de una dada geometría), entonces
\begin{equation*}
    \nabla^{2}G_{k}(\textbf{R}) + k^{2}G_{k}(\textbf{R}) = 
    -4\pi\delta(\textbf{R})
\end{equation*}
y tiene soluciones\footnote{Esto creo que en F2 se da así que ni en pedo hago las cuentas}
\begin{equation*}
    G_{k}(\textbf{R}) = 
    A\frac{e^{ikR}}{R} + B\frac{e^{-ikR}}{R}
\end{equation*}
representan ondas esféricas entrantes y salientes, donde se definen
\begin{equation*}
    G_{k}^{+}(R) = \frac{e^{ikR}}{R}
\end{equation*}
\begin{equation*}
    G_{k}^{-}(R) = \frac{e^{-ikR}}{R}
\end{equation*}
Ahora vuelvo al espacio real, para eso tomo la expresión \eqref{ec:GreenEspacioFourier} que está en el espacio de Fourier, pero la escribo en el espacio real, esto sería simplemente sacar $k^{2}$ y poner las derivadas segundas respecto al tiempo y el factor $c^{2}$ dividiendo. Como \eqref{ec:GreenEspacioFourier} en el espacio real equivalía a tener una fuente que se prendía y apagaba con frecuencia $\omega$, ahora lo que voy a hacer es agregarle una delta de Dirac temporal de forma de tener un pulso (en el espacio real) y no una fuente oscilante, es decir
\begin{equation*}
    \left(
        \nabla^{2} - \frac{1}{c^{2}}\frac{\partial ^{2}}{\partial t^{2}}
    \right)
    G(\textbf{r} , t; \textbf{r}', t') = 
    -4\pi\delta(\textbf{r}-\textbf{r}')\delta(t - t')
\end{equation*}
Ahora, transformando Fourier se obtiene de nuevo \eqref{ec:GreenEspacioFourier} pero con un factor $e^{i\omega t'}$ multiplicando.
\begin{equation*}
    \left(
        \nabla^{2} + k^{2}
    \right)
    G_{k}(\textbf{r}, \textbf{r}')
    = -4\pi\delta(\textbf{R})e^{i\omega t'}
\end{equation*}
entonces, las soluciones son las que ya había encontrado antes, pero con el agregado de la exponencial de $t'$ multiplicando.
\begin{equation*}
    G_{k}^{\pm}(\textbf{r}-\textbf{r}') = 
    \frac{e^{\pm ikR}}{R}\,e^{i\omega t'}
\end{equation*}
Resta volver al espacio real. Antitransformando, lo que depende de $R$ sale fuera de la integral y lo que queda termina siendo una delta de Dirac: $\delta\left(\pm \frac{R}{C}  +t - t'\right)$. Reemplazando $R$ por $|\textbf{r}-\textbf{r}'|$ en todas partes se termina llegando a la expresión final para las funciones de Green
\begin{equation}
   G^{\pm}(\textbf{r},t;\textbf{r}',t') =
   \frac%
    {%
        \delta
        \left[
            t' - 
            \left(
                t \mp \frac{|\textbf{r}-\textbf{r}'|}{c}
            \right)
        \right]
    }%
    {
        |\textbf{r}-\textbf{r}'|
    }
        \label{ec:FuncionGreenRetAva}
\end{equation}
¿Cuál es la interpretación física de que una función de Green sea una delta? La respuesta es que para el caso de un pulso, es decir, se prende y se apaga la fuente de forma instantánea, se obtiene un frente de onda esférico que se propaga con velocidad de la luz $c$ y que es cero en todas partes y en todo tiempo, menos $t'$. O sea que si se tienen, por ejemplo, dos observadores, uno en $r$, $t$(campo) y otro en $r'$ y $t'$ (fuente), el observador en $r$ y $t$ no verá nada hasta el momento en que $t = t' + |\textbf{r}-\textbf{r}'|/c$, o dicho en cristiano: No verá nada hasta el momento en el que el \textit{cascarón} esférico del frente onda lo alcance. Una vez que el frente de onda pase, tampoco verá nada.\\
\indent Estas son las funciones de Green avanzadas $G^{(-)}$ y retardadas $G^{(+)}$. Para la interpretación de las funciones de Green avanzadas se puede pensar que refieren a una onda electromagnética esférica que se contrae.\\
\indent Dada una fuente $f(\textbf{r},t)$, se tiene solución particular de la ecuación
\begin{equation*}
    \psi_{part}(\textbf{r},t) = 
    \int d^{3}r'\,dt'\,G^{\pm}(\textbf{r},t; \textbf{r}',t')f(\textbf{r}',t')
\end{equation*}
la solución general es 
\begin{equation*}
    \psi(\textbf{r},t) = \psi_{part}(\textbf{r},t) + \psi_{h}(\textbf{r},t)
\end{equation*}
con solución homogénea tal que satisfaga las condiciones de contorno. Hay que notar que la elección conveniente de $G^{\pm}$ facilita satisfacer las condiciones de contorno. \\
\indent Y acá viene un ejemplo de un dipolo puntual que se enciende en el orden para un dado tiempo $t$ y queda encendido. Para resolver el problema hay que escribir las fuentes del dipolo, que van a ser la densidad de carga $\rho$ y la densidad de corriente $\textbf{j}$, que como es un dipolo puntual apuntando en $z$, resulta ser $\textbf{j} = j_{z} \hat{z}$. La densidad de corriente del dipolo es la primera que se puede escribir y luego, usando la ecuación de continuidad se puede escribir la densidad de carga del dipolo. La densidad de corriente viene dada por el módulo de la intensidad del dipolo $p$, apuntando en dirección $\hat{z}$ y localizada en el origen de coordenadas, es decir, con deltas de Dirac. A su vez, como el dipolo se enciende una única vez, la densidad de corriente tiene que ser 0 hasta que a tiempo $t'$ este se enciende y luego la corriente vuelve a ser 0 (porque las cargas quedan estáticas. Prender el dipolo sería como traer las cargas desde el infinito hasta el origen de forma instantánea y ese desplazamiento sería la corriente) . Eso se escribe con otra delta de Dirac. En definitiva $\textbf{j} = p \delta(x')\delta(y')\delta(z')\delta(t')\hat{z}$. Usando la ecuación de continuidad $\partial_{t}\rho = - \partial_{z}\textbf{j}$ se puede despejar la expresión para $\rho$, que resulta $\rho = -p \delta(x')\delta(y')\delta'(z')\Theta(t')$, donde aparece una función de Heaviside porque al prender el dipolo, este queda encendido a partir de $t'$ y de las cuentas sale de integrar la delta de Dirac en $t'$.\\
\indent Una vez que se tienen las fuentes, pueden calcularse los potenciales usando la función de Green \eqref{ec:FuncionGreenRetAva}. El potencial electrostático y el potencial vector resultan entonces
\begin{equation*}
    \varphi(r,t) = \frac{pz}{cr^{2}}\delta
    \left(
        t - \frac{r}{c}
    \right)
    +
    \frac{pz}{r^{2}}\Theta
    \left(
        t - \frac{r}{c}
    \right),
    \quad\quad\quad
    \textbf{A}(r,t) = \frac{p}{cr}\delta
    \left(
        t - \frac{r}{c}
    \right),
    \hat{z}
\end{equation*}
Y una vez que se tienen los potenciales, pueden calcularse los campos haciendo, para el campo eléctrico $\textbf{E} = -\nabla \varphi - \frac{1}{c}\frac{\partial \textbf{A}}{\partial t}$ y para el campo magnético haciendo $\textbf{B} = \nabla \times \textbf{A}$, entonces se llega 
\begin{align*}
    \textbf{E}(r,t) = &
    \left[
        \frac{3 (\textbf{p}\cdot \textbf{r})\textbf{r} - r^{3}\textbf{p}}{r^{5}}
    \right]
    \Theta 
    \left(
        t - \frac{r}{c}
    \right) & \longleftarrow \mbox{(i)}
    \\
    + &
    \frac{pz}{cr^{3}}\delta
    \left(
        t - \frac{r}{c}
    \right)\hat{r}
    +
    \frac{1}{c}
    \left[
        \frac{2(\textbf{p}\cdot\textbf{r})\textbf{r} - r^{2}\textbf{p}}{r^4}
    \right]
    \delta
    \left(
        t - \frac{r}{c}
    \right) & \longleftarrow \mbox{(ii)}
    \\
    + &
    \frac{p}{rc^{2}}\delta'
    \left(
        t - \frac{r}{c}
    \right)
    \sin{(\theta)}\hat{\theta} & \longleftarrow \mbox{(iii)}
\end{align*}
donde (i) representa el retardo en el llegado de la información y es muy parecido a la solución estacionaria del dipolo, (ii) son términos que caen como $1/r^{2}$, o sea que no llegan al infinito y dependen de la derivada primera respecto de $t$ de la evolución de las cargas (algo que va como \textit{la velocidad}) y por último (iii) que depende de la derivada segunda respecto a $t$ de la evolución de las cargas (algo que va como \textit{la aceleración}) y cae como $1/r$, con lo cual transmite información al infinito, es el campo de \textbf{radiación}.\\
\indent El campo magnético queda $\textbf{B}$
\begin{equation*}
    \textbf{B}(r,t) = \frac{p}{cr^{2}}
    \delta
    \left(
        t - \frac{r}{c}
    \right)\hat{\phi} 
    +
    \frac{p}{c^{2}r}\delta'
    \left(
        t - \frac{r}{c}
    \right)\hat{\phi}
\end{equation*}
Donde el primer término es el campo de velocidad y el segundo es el campo de aceleración\footnote{Cuando se habla de velocidad y aceleración no se refiere a la velocidad o aceleración de los campos. Los campos se desplazas a velocidad de la luz $c$ constante. Cuando se habla de aceleración o velocidad se refiere a las cargas, las velocidad y aceleración de las cargas} o radiación (cae como $1/r$ y transmite información hasta el infinito)\footnote{Ver apéndice 35 para las cuentas}.\\
%
%
%fin clase 15
%%%%%%%%%%%%%%%%%%%%%%%%%%%%%%%%%%%%%%%%%%%
%%%%%%%%%%%%%%%%%%%%%%%%%%%%%%%%%%%%%%%%%%%
%inicio clase 16
%
%
\indent Resolver las ecuaciones de Maxwell dependientes del tiempo, en general, no es trivial. Vamos a generar mecanismos para poder resolver las ecuaciones de Maxwell dependientes del tiempo de forma aproximada. Dado que en el ejemplo anterior, la solución consistió en $3$ términos, donde el primero se parece a la solución estacionaria, el segundo era un término dependiente de las velocidades de las cargas y el tercero es el término de radiación, lo que se va a intentar hacer es usar aproximación para llegar al término deseado y no a todos, por ejemplo, si estamos muy cerca del dipolo, se va a hacer una aproximación \textit{cuasiestacionaria} para obtener el primer término. Cuando estamos muy lejos del dipolo se va a hacer una aproximación de \textit{campo lejano(?)} para obtener el campo de radiación.\\


%%%%%%%%%%%%%%%%%%%%%%%%%%%%%%%%%%%%%%%%%%%
%%%%%%%%%%%%%%%%%%%%%%%%%%%%%%%%%%%%%%%%%%%


\subsection{Aproximación cuasiestacionaria}
Estamos en un régimen donde las distribuciones de corriente o de carga varían lentamente en función del tiempo o estamos en una región muy cercana a donde estas se encuentran. En cualquiera de los dos casos, la frecuencia $\omega$ de variación de las cargas es tal que la longitud de onda $\lambda$ de los campos es mucho más grande que el tamaño $L$ característico del sistema en cuestión.\\
\indent Tomando las ecuaciones de Maxwell (recordar: para conductores vale la relación $\textbf{j} = \sigma \textbf{E}$, con $\sigma$ la conductividad eléctrica) y planteando soluciones de la forma
\begin{equation*}
    \textbf{E}(\textbf{r},t) = \textbf{E}(\textbf{r})e^{i\omega t}
\end{equation*}
\begin{equation*}
    \textbf{B}(\textbf{r},t) = \textbf{B}(\textbf{r})e^{i\omega t}
\end{equation*}
las ecuaciones de Maxwell se reescriben como
\begin{equation*}
    \left\{
        \begin{matrix}
            \nabla \cdot \textbf{E} = 4\pi\rho,
            &
            \nabla \times \textbf{B} = 
            \frac{4\pi}{c}\textbf{j}
            + \frac{i\omega}{c}\textbf{E}
            &
            \longleftarrow
            \mbox{Con fuentes}\\
            & & \\
            \nabla \cdot \textbf{B} = 0,
            &
            \nabla \times \textbf{E} =
            -\frac{i\omega}{c}\textbf{B}
            &
            \longleftarrow \mbox{Sin fuentes}
        \end{matrix}
    \right.
\end{equation*}
donde la parte espacial se resuelve de forma perturbativa en $k = \frac{\omega}{c}$, es decir, una suma de términos con potencias crecientes del número de onda $k$
\begin{equation*}
    \left\{
        \begin{matrix}
            \textbf{E}(\textbf{r}) = \sum\limits_{n}k^{n}\textbf{E}^{(n)}(\textbf{r})\\
            \\
            \textbf{B}(\textbf{r}) = \sum\limits_{n}k^{n}\textbf{B}^{(n)}(\textbf{r})\\
        \end{matrix}
    \right.
\end{equation*}
entonces, teniendo estas expansiones espaciales para los campos eléctrico y magnético, se lo puede meter en las ecuaciones de Maxwell y evaluar los operadores rotor y divergencia como sigue
\begin{equation*}
    \nabla \cdot \textbf{E}(\textbf{r}) = 
    \nabla \cdot \textbf{E}^{(0)}(\textbf{r}) + 
    \nabla \cdot k\textbf{E}^{(1)}(\textbf{r}) + 
    \nabla \cdot k^{2}\textbf{E}^{(2)}(\textbf{r}) + \cdots
    = 4\pi\rho(\textbf{r},t)
\end{equation*}
\begin{equation*}
    \nabla \times \textbf{E}(\textbf{r}) = 
    \nabla \times \textbf{E}^{(0)}(\textbf{r}) + 
    \nabla \times k\textbf{E}^{(1)}(\textbf{r}) + 
    \nabla \times k^{2}\textbf{E}^{(2)}(\textbf{r}) + \cdots
    = -\frac{i\omega}{c}\textbf{B}(\textbf{r},t)
\end{equation*}
y para el campo magnético
\begin{equation*}
    \nabla \cdot \textbf{B}(\textbf{r}) = 
    \nabla \cdot \textbf{B}^{(0)}(\textbf{r}) + 
    \nabla \cdot k\textbf{B}^{(1)}(\textbf{r}) + 
    \nabla \cdot k^{2}\textbf{B}^{(2)}(\textbf{r}) + \cdots
    = 0
\end{equation*}
\begin{equation*}
    \nabla \times \textbf{B}(\textbf{r}) = 
    \nabla \times \textbf{B}^{(0)}(\textbf{r}) + 
    \nabla \times k\textbf{B}^{(1)}(\textbf{r}) + 
    \nabla \times k^{2}\textbf{B}^{(2)}(\textbf{r}) + \cdots
    = \frac{4\pi}{c}\textbf{j} + \frac{i\omega}{c}\textbf{E}
\end{equation*}
En particular, si las corrientes también las escribo moduladas en función del tiempo, es decir $\textbf{j}(\textbf{r},t) = \textbf{j}(\textbf{r})e^{i\omega t}$, y la densidad de carga $\rho = 0$, a orden cero obtengo los campos
\begin{equation*}
    \left\{
        \begin{matrix}
            \nabla \cdot \textbf{E}^{(0)} = 0,
            &
            \nabla \times \textbf{B}^{(0)} = \frac{4\pi}{c}\textbf{j}\\
            & \\
            \nabla \cdot \textbf{B}^{(0)} = 0,
            &
            \nabla \times \textbf{E}^{(0)} = 0
        \end{matrix}
    \right.
\end{equation*}
donde $\nabla \cdot \textbf{E}^{(0)} = 0$ y $\nabla \cdot \textbf{B}^{(0)} = 0 $ trivialmente porque vale para los campos completos, pero nótese que $\nabla \times \textbf{E}^{(0)} = 0$ a pesar de que el campo completo es proporcional a $\textbf{B}$. Esto es porque $i\omega/c = ik$ y necesito igualar términos de un lado y del otro de la ecuación con iguales potencias del número de onda $k$. Como $\nabla \times \textbf{E}^{(0)}$ no baja ningún $k$, no puedo igualarlo a $ik \textbf{B}$, entonces es 0.\\
\indent O sea que efectivamente, a orden 0, los campos son estacionarios y, en particular, el campo magnético es el campo magnetostático. Entonces, a orden $0$, la solución completa son los campos que cumplen estas expresiones, modulados por $e^{i\omega t}$.\\
\indent A orden $1$ en en la expansión, se tiene $\nabla \cdot k\textbf{E}^{(1)} = 0$ y $\nabla \cdot k\textbf{B}^{(1)} = 0$ trivialmente, ya que los campos completos valen $0$. Los rotores en cambio, para el campo eléctrico y magnético no son tan triviales. Para el campo eléctrico, ahora se iguala a la aproximación de orden $0$ del campo magnético ya que $\nabla \times k\textbf{E}^{(1)}$ tiene la misma amplitud que $ik\textbf{B}^{(0)}$. Por otro lado, para el campo magnético $\nabla \times k\textbf{B}^{(1)}$ ya no se puede igualar a la corriente, porque ya está tomada en cuenta y además tienen diferente amplitud, con lo cual solo queda la contribución del campo eléctrico, pero el campo eléctrico a orden $0$ es cero, de modo que $\nabla \times k \textbf{B}^{(1)} = 0$. Así, a orden $1$ los campos quedan
\begin{equation*}
    \left\{
        \begin{matrix}
            \nabla \cdot \textbf{E}^{(1)} = 0,
            &
            \nabla \times \textbf{B}^{(1)} = 0\\
            & \\
            \nabla \cdot \textbf{B}^{(1)} = 0,
            &
            \nabla \times \textbf{E}^{(1)} = -i \textbf{B}^{(0)}
        \end{matrix}
    \right.
\end{equation*}
Entonces, a orden $1$, efectivamente recupero las expresiones de la estática en esta aproximación. Con lo cual, nuevamente, la solución completa son los campos tales que cumplen estas ecuaciones moduladas por la exponencial $e^{i\omega t}$.\\
\indent Considero un conductor y quiero ver qué tanto penetra una dada onda electromagnética en el interior del conductor antes de que su amplitud decaiga. Como es un conductor vale $\textbf{j} = \sigma \textbf{E}$, entonces, para la aproximación a orden cero tengo
\begin{equation*}
    \nabla \times \textbf{B}
    = \frac{4\pi}{c}\textbf{j} 
    = \frac{4\pi \sigma}{c}\textbf{E}
\end{equation*}
si quiero prescindir del campo $\textbf{E}$, tomo rotor de la expresión y uso que a orden $1$ el roto de $\textbf{E}^{(1)}$ es $-i\textbf{B}^{(0)}$
\begin{equation*}
    \nabla\times\nabla\times \textbf{B} = \frac{-4i\pi\sigma}{c}\textbf{B} 
\end{equation*}
y $\nabla \times \nabla \times \textbf{B}$ en aproximación cuasiestacionaria y usando la identidad $\nabla \times \nabla \times \textbf{A} = -\nabla^{2} \textbf{A} + \nabla (\nabla \cdot \textbf{A})$ llego a
\begin{equation*}
    \nabla^{2}\textbf{B} = \frac{4 i \pi \sigma \omega}{c^{2}} \textbf{B}
\end{equation*}
que es una ecuación de difusión, lo cual era lo que quería porque quería saber como difunde una onda electromagnética de frecuencia $\omega$ al penetrar un conductor. Para determinar el $\delta$ de penetración, uso análisis dimensional. Como el Laplaciano tiene unidades de $1/L^{2}$, reemplazo por eso del lado izquierdo, simplifico los campos $B$ y queda que la longitud $L$ es proporcional a la raíz cuadrada de la velocidad de la luz al cuadrado sobre la frecuencia y la conductividad del material, es decir
\begin{equation*}
    \delta \sim \sqrt{\frac{c^{2}}{\omega \sigma}}
\end{equation*}
con lo cual, la penetración en el material depende de la frecuencia y de las características del material que determinan su conductividad. Para frecuencias más altas, la penetración será menor y lo mismo sucederá para conductividades más altas. Tiene sentido porque un material con alta conductividad reordena sus cargas libres de más rápidamente anulando el campo en su interior de forma más veloz.



%%%%%%%%%%%%%%%%%%%%%%%%%%%%%%%%%%%%%%%%%%%
%%%%%%%%%%%%%%%%%%%%%%%%%%%%%%%%%%%%%%%%%%%


\subsection{Movimiento de un conductor en un campo magnético}
Para el caso de un conductor en reposo, la ley de Ohm es $\textbf{j} = \sigma \textbf{E}$, pero para el caso de un conductor que se mueve con velocidad constante $\textbf{v}$, el campo eléctrico no puede ser le mismo, con lo cual $\textbf{j} = \sigma \textbf{E}'$, donde la prima indica que es el campo eléctrico medido desde el referencial del conductor. Considero que la velocidad de desplazamiento es mucho menor que la velocidad de la luz $v << c$. Para un conductor que es un circuito cerrado, la ley de Faraday era
\begin{equation*}
    \oint\limits_{C}\textbf{E}'\cdot d\textbf{l} 
    = -\frac{1}{c}\frac{d}{dt}\int\limits_{S(C)}\textbf{B}\cdot d\textbf{S}
\end{equation*}
y tengo una derivada total respecto del tiempo. Hay que reescribirla para que contemple que el conductor se se desplaza con velocidad $\textbf{v}$ y que el campo magnético medido desde el sistema de laboratorio depende de $t$. Para eso tomo la derivada material (o convectiva de $\textbf{B}$)
\begin{equation*}
    \lim_{\delta t\to 0} \frac{1}{\delta t}
    \bigg[
        \textbf{B}(\textbf{x} + \textbf{v}\delta t, t + \delta t ) - \textbf{B}(\textbf{x}, t)
    \bigg]
    \approx
    \left(
        \frac{\partial}{\partial t} + \textbf{v}\cdot \nabla 
    \right)\textbf{B}(\textbf{x},t)
\end{equation*}
que es la derivada del campo $\textbf{B}$ dado que depende del tiempo y dado que se desplaza el conductor. Tomo la siguiente identidad\footnote{Ver apéndice 37.5}
\begin{equation*}
    \nabla \times (\textbf{B} \times \textbf{v})
    = 
    \left(
        \textbf{v} \cdot \nabla
    \right)\textbf{B}
\end{equation*}
y gracias a esta identidad recuperé uno de los términos de la derivada convectiva. Puedo reemplazar todo en la expresión para la ley de Faraday y queda
\begin{equation*}
    \oint\limits_{C} \textbf{E}'\cdot d\textbf{l} = 
    -\frac{1}{c}
    \left[
        \int\limits_{S(C)}
        \left(
            \frac{\partial}{\partial t} + \textbf{v}\cdot \nabla
        \right)\textbf{B}(\textbf{x},t)\,d\textbf{S}
    \right]
    =
    -\frac{1}{c}
    \int\limits_{S(C)}
    \frac{\partial \textbf{B}}{\partial t}\cdot d\textbf{S} 
    -
    \frac{1}{c}
    \int\limits_{S(C)}
    \nabla \times (\textbf{B}\times \textbf{v})\,d\textbf{S}
\end{equation*}
Notar que si $\textbf{v} = 0$, se recupera la ley de Faraday usual. Se puede hacer Stokes en la última integral y despejarla para el lado izquierdo donde tengo otra integral curvilínea y, cambiándole el orden al producto vectorial $\textbf{B} \times \textbf{v} = - \textbf{v} \times \textbf{B}$ tengo
\begin{equation*}
    \oint\limits_{C}
    \left(
        \textbf{E}' - \frac{1}{c}\textbf{v} \times \textbf{B}
    \right)
    \cdot d\textbf{l}
    = 
    -\frac{1}{c}\int\limits_{S(C)}\frac{\partial \textbf{B}}{\partial t}\cdot d\textbf{S}
\end{equation*}
se recupera una Ley de Faraday modificada, donde del lado derecho se tiene la misma expresión que en el caso con $\textbf{v} = 0$, pero del lado derecho se tiene un nuevo campo. Necesariamente ese nuevo campo tiene que ser el campo eléctrico total debido al desplazamiento del conductor en el campo magnético. Con lo cual, el campo eléctrico en el referencial del conductor $\textbf{E}'$, para velocidades de movimiento (en el referencial de laboratorio) mucho menores que la velocidad de la luz, es el campo eléctrico medido en el referencial del laboratorio más un término que viene del campo magnético medido en el referencial del laboratorio también, es decir
\begin{equation*}
    \textbf{E}' = \textbf{E} + \frac{1}{c}
    \textbf{v}\times \textbf{B}
\end{equation*}
Con esto, la ley de Ohm en el referencial del conductor se reescribe como
\begin{equation*}
    \textbf{j}' = \sigma
    \left(
        \textbf{E} + \frac{1}{c}
        \textbf{v}\times \textbf{B}
    \right)
\end{equation*}
Entonces, de las ecuaciones de Maxwell se puede obtener \footnote{Ver apéndice 37.5} una ecuación de inducción de la forma
\begin{equation*}
    \frac{\partial \textbf{B}}{\partial t} 
    = \nabla \times 
    \left(
        \textbf{v}\times \textbf{B}
    \right)
    +
    \eta \nabla^{2}\textbf{B}
\end{equation*}
donde el término $\eta \nabla^{2}\textbf{B}$ es un término de disipación Óhmica. El término $\nabla \times (\textbf{v} \times \textbf{B})$ es un término lineal en $\textbf{B}$, con lo cual va a tener soluciones de la forma $\sim \textbf{B}(\textbf{r})e^{\gamma t}$ con $\gamma$ que podrá ser real o complejo y físicamente significa que el movimiento de un conductor en un campo magnético puede generar un nuevo campo magnético por inducción.


%%%%%%%%%%%%%%%%%%%%%%%%%%%%%%%%%%%%%%%%%%%
%%%%%%%%%%%%%%%%%%%%%%%%%%%%%%%%%%%%%%%%%%%


\subsection{Teorema de Alfvén}
En un conductor, las lineas de campo magnético están \textit{congeladas} al medio material y el flujo magnético se conserva. De modo que si el conductor pudiera deformarse, las lineas del campo se deformarían junto con él.\\
\indent Para demostrarlo considero que la difusividad magnética es $0$, es decir, el $\eta = 0$. Entonces la variación de flujo magnético en el tiempo viene dado por
\begin{equation*}
    \frac{d \Phi}{dt} =
    \frac{d}{dt}\int\limits_{S(C)} \textbf{B}\cdot d\textbf{S}
    = - c \int\limits_{C}\textbf{E}'\cdot d\textbf{l}
    = -\frac{c}{\sigma} \int\limits_{C}\textbf{j}\cdot d\textbf{l} 
    = -\eta \frac{4\pi}{c}
    \int\limits_{C}\textbf{j}\cdot d\textbf{l}
\end{equation*}
entonces 
\begin{equation*}
    \frac{d\Phi}{dt} = 0
    = \frac{d}{dt}\int\limits_{S(C)}\textbf{B}\cdot d\textbf{S}
\end{equation*}
con lo cual el flujo magnético a través de cualquier curva material que se desplaza con el conductor se conserva. Si consideramos un tubo material de lineas de campo \textbf{B}, delimitadas por una curva $C$. Como el flujo de campo magnético que pasa por la sección $s$ del tubo inicial se conserva, si se estira el tubo, las lineas de campo se vuelven más densas de forma que el flujo de campo a través de la nueva sección $s'$ sea igual al flujo a través de $s$. Para que esto suceda necesariamente la intensidad del campo magnético debe haber aumentado (a menor superficie, igual flujo, entonces mayor campo magnético). Con lo cual, se transformó energía mecánica al estirar el tubo en intensidad de campo magnético! Con la intensidad de campo magnético, también aumenta la energía magnética!\\
\indent Este mecanismo es necesario para explicar el campo magnético de la tierra. Si dominara la difusión Óhmica, es decir $\eta \neq 0$ y este teorema no vale, el campo terrestre de la tierra decaería en diez mil años.


%fin clase 16
%%%%%%%%%%%%%%%%%%%%%%%%%%%%%%%%%%%%%%%%%%%
%%%%%%%%%%%%%%%%%%%%%%%%%%%%%%%%%%%%%%%%%%%
%inicio clase 17


\subsubsection{Magnetohidrodinámica}
En esta parte lo que se va a deducir es una especie de ecuaciones de Navier-Stokes pero para campos magnéticos dentro de fluidos. El problema en cuestión es un fluido que se mueve según un campo de velocidades $\textbf{u}$, con densidad $\rho$ con presión $P$. La deducción de las ecuaciones se lleva a cabo usando la conservación del impulso del campo electromagnético \eqref{ec:BalanceImpulsoEM} en la aproximación de $u << c$, de forma que el campo electromagnético varia muy poco con el desplazamiento de los elementos de fluido y la ecuación de conservación de masa de fluidos, idéntica a la conservación de la carga, es decir:
\begin{equation*}
    \frac{d\textbf{P}_{mec}}{dt} 
    + \cancel{\frac{d\textbf{P}_{EM}}{dt}} 
    = \int\limits_{V}\partial_{j}(T_{ij} + \sigma_{ij})\,dV,
    \quad
    \quad
    \quad
    \frac{\partial \rho}{\partial t} + \nabla \cdot
    \left(
        \rho\, \textbf{u}
    \right)
    = 0
\end{equation*}
donde se agregó a la ecuación de balance de impulso un tensor $\sigma_{ij}$ que es el encargado de añadir la contribución de los esfuerzos mecánicos sobre el fluido (mientras que $T_{ij}$ es el tensor de esfuerzos electromagnéticos).\\
\indent Considero al fluido tal que $\rho = cte$ y eléctricamente neutro (hay tantas cargas positivas como negativas en cada elemento de volumen). El tensor de esfuerzos electromagnéticos es el tensor de Maxwell para el caso de $E = 0$, es decir
\begin{equation*}
    T_{ij} = \frac{1}{4\pi}
    \left(
        B_{i}B_{j} - \frac{B^{2}}{2}\delta_{ij}
    \right)
\end{equation*}
El tensor de esfuerzos mecánicos es el que sale de analizar los esfuerzos en un elemento de volumen. Se pueden tener esfuerzos de presión normales a cada cara del elemento de volumen y esfuerzos de corte, tangenciales a cada cara del elemento de fluido. Estos últimos solo puede deberse a la velocidad relativa de desplazamiento entre elementos de fluidos adyacentes, de modo que los esfuerzos tangenciales deben ser proporcionales a la diferencia de velocidad entre ambos elementos. Con esto en mente, el tensor de esfuerzos mecánicos queda expresado como
\begin{equation*}
    \sigma_{ij} = 
    - P\delta_{ij}
    + \rho \nu
    \left(
        \partial_{j}u_{i} + \partial_{i}u_{j}
    \right)
\end{equation*}
donde el primer término son los esfuerzos de presión y el segundo término son los esfuerzos viscosos.\\
\indent Como el impulso mecánico $\textbf{P}_{mec}$ es proporcional al desplazamiento de masa de fluido, se escribe como $\textbf{P}_{mec} = \rho \textbf{u}$. Reemplazando en la expresión total del balance de impulso se llega\footnote{Ver apéndice 38.4}
\begin{equation*}
    \rho 
    \left(
        \frac{\partial \textbf{u}}{\partial t} 
        +
        \textbf{u}\cdot \nabla \textbf{u}
    \right)
    =
    -\nabla 
    \left(
        p + \frac{B^{2}}{8\pi}
    \right)
    +
    \frac{1}{4\pi} \textbf{B}\cdot \nabla \textbf{B} 
    +
    \rho\, \nu\nabla^{2}\textbf{u}
\end{equation*}
donde los dos primeros términos de la derecha son términos provenientes del campo magnético en el fluido por la fuerza de Lorentz, el primero considera la presión mecánica en el fluido y la presión magnética ($B^{2}/8\pi$), mientras que el segundo tiene en cuenta la tensión magnética. El último término es el debido a la viscosidad del fluido.\\
\indent Por último, tomando $\textbf{b} = \frac{\textbf{B}}{\sqrt{4\pi\rho}}$ (velocidad de Alfvén) se llega a
\begin{equation*}
    \left\{
        \begin{matrix}
            \frac{\partial \textbf{u}}{\partial t}
            = -\textbf{u}\cdot \nabla \textbf{u}
            + \textbf{b}\cdot \nabla \textbf{b}
            -\nabla
            \left(
                \frac{p}{\rho} + \frac{b^{2}}{2}
            \right)
            +
            \nu \nabla^{2}\textbf{u}\\
            \\
            \frac{\partial \textbf{b}}{\partial t} 
            = \nabla \times (\textbf{u}\times \textbf{b}) 
            + \eta\,\nabla^{2}\textbf{b}
        \end{matrix}
    \right.
\end{equation*}


%%%%%%%%%%%%%%%%%%%%%%%%%%%%%%%%%%%%%%%%%%%
%%%%%%%%%%%%%%%%%%%%%%%%%%%%%%%%%%%%%%%%%%%


\subsection{Ondas electromagnéticas en medios no dispersivos}
Hasta ahora se lograron mecanismos para hallar soluciones exactas a las ecuaciones de Maxwell tanto en el caso estacionario como en el caso dependiente del tiempo con funciones de Green (retardadas y avanzadas), pero con el ejemplo del dipolo que se enciende queda expuesta la dificultad de analizar de forma exacta sistemas simples. La dificultad para encontrar soluciones exactas a las ecuaciones de Maxwell dependientes del tiempo escala muy rápidamente.\\
\indent Las ecuaciones para los potenciales $\varphi$ y $\textbf{A}$ son ecuaciones inhomogéneas, es decir, tienen fuentes. Entonces ahora vamos a considerar el caso de las ecuaciones de Maxwell homogéneas (sin fuentes) dependientes del tiempo y ver qué tipo de soluciones aparecen. Para atacar este problema se usará como aproximación a las ondas electromagnéticas como ondas planas. Esto quiero decir que considero que la fuente que genera las ondas electromagnéticas es tan lejana que el efecto que se observo es que llegan ondas planas. Usando esta aproximación se quiere ver a qué tipo de soluciones exactas se puede arribar.\\
\indent Las ecuaciones de Maxwell homogéneas (sin fuentes) serán
\begin{equation*}
    \left\{
        \begin{matrix}
            \nabla \cdot \textbf{E} = 0 \\
            \\
            \nabla \cdot \textbf{B} = 0 \\
            \\
            \nabla \times \textbf{E} 
            + 
            \frac{1}{c}\frac{\partial \textbf{B}}{\partial t} = 0\\
            \\
            \nabla \times \textbf{B} 
            = \frac{\mu \epsilon}{c} \frac{\partial \textbf{E}}{\partial t}
        \end{matrix}
    \right.
\end{equation*}
a partir de las cuales, aplicándoles el rotor a las dos ultimas se puede llegar a las siguientes expresiones\footnote{Ver apéndice 39.2}
\begin{equation*}
    \nabla^{2}\textbf{E} 
    = \frac{\mu\epsilon}{c}\frac{\partial^{2}\textbf{E}}{\partial t^{2}}
\end{equation*}
\begin{equation*}
    \nabla^{2}\textbf{B} 
    = \frac{\mu\epsilon}{c}\frac{\partial^{2}\textbf{B}}{\partial t^{2}}
\end{equation*}
a estas ecuaciones transformándolas Fourier (que es básicamente derivar respecto de $t$ y cambiar su dependencia en $t$ por $\omega$) se llega\footnote{Ver apéndice 39.2}
\begin{equation}
    \left(
        \nabla^{2} + \frac{\mu \epsilon}{c^{2}}\omega^{2}
    \right)
    \textbf{E}(\textbf{r}, \omega) 
    = 0,
    \quad
    \quad
    \quad
    \left(
        \nabla^{2} + \frac{\mu \epsilon}{c^{2}}\omega^{2}
    \right)
    \textbf{B}(\textbf{r}, \omega) 
    = 0
        \label{ec:OndasPlanasEM1}
\end{equation}
y ahora impongo como soluciones para estos campos las ondas planas (en coordenadas cartesianas), para ver qué resultados físicos arribo. Entonces, si propongo
\begin{equation*}
    \textbf{E}(\textbf{r},\omega) = \textbf{E}_{0}\,e^{i \textbf{k}\cdot \textbf{r}},
    \quad
    \quad
    \quad
    \textbf{B}(\textbf{r},\omega) = \textbf{B}_{0}\,e^{i \textbf{k}\cdot \textbf{r}}
\end{equation*}
donde las amplitudes $\textbf{E}_{0}$ y $\textbf{B}_{0}$ son complejas, $\textbf{k} = (k_{x}, k_{y}, k_{z})$. Si reemplazo estas soluciones en las expresiones \eqref{ec:OndasPlanasEM1} obtengo que\footnote{Ver apéndice 39.2}
\begin{equation*}
    \nabla^{2} \textbf{E} + \frac{\mu\epsilon}{c^{2}}\omega^{2}\textbf{E} = 
    -k^{2} \textbf{E} + \frac{\mu\epsilon}{c^{2}}\omega^{2}\textbf{E} 
    \Longrightarrow k^{2} = \frac{\mu \epsilon}{c^{2}}\omega^{2}
\end{equation*}
de donde se puede hallar la relación de dispersión $k = \sqrt{\frac{\mu \epsilon}{c^{2}}}\omega = \frac{\omega}{c}n$, con $n$ el índice de refracción. En el espacio real, la solución sería
\begin{equation*}
    \textbf{E}(\textbf{r},t) = \textbf{E}_{0}\,e^{i(\textbf{k}\cdot \textbf{r} - \omega t)}
\end{equation*}
Se define la velocidad de fase $v_{f} = \omega/k = c/\sqrt{\mu \epsilon} = c/n$, con lo cual es independiente de la frecuencia $\omega$. Además, esto significa que si el medio es el vacío, de forma que $\mu = \epsilon = 1$, entonces el índice de refracción $n = 1$ y la velocidad de fase es la velocidad de la luz $c$. Con lo cual, para medios distintos del vacío, la velocidad de fase es siempre menor a la velocidad de la luz en el vacío, lo que implica que la luz viaja más lento en el interior de los medios (lineales, isótropos y homogéneos).\\
\indent Observación que hace Mininni: ¿Por qué decimos que son soluciones de ondas planas? ¿Cómo sabemos que son planas? Eso se ve si tomamos una fase de la solución y la igualamos a una constante, es decir $\textbf{k}\cdot \textbf{r} - \omega t = k_{x}x + k_{y}y + k_{z}z - \omega t = cte$, se tiene que para un tiempo fijo dado $t$, $k_{x}x + k_{y}y + k_{z}z = \omega \tilde{t}$ es la ecuación del plano. Entonces todos los puntos que tienen la misma amplitud y dirección del campo resultan ser planos.\\
\indent Ahora, puede verse en qué dirección se propaga el campo. Si calculo la divergencia de la solución, es decir, tomo $\nabla \cdot \textbf{E}(\textbf{r},t) = 0$, tengo que
\begin{equation*}
    \nabla \cdot \textbf{E}(\textbf{r},t)
    = \textbf{E}_{0}\nabla \cdot e^{i(\textbf{k}\cdot \textbf{r}-\omega t)}
    = \textbf{E}_{0}e^{i(\textbf{k}\cdot \textbf{r}-\omega t)} 
    \nabla \cdot \textbf{k}\cdot\textbf{r}
    = \textbf{k} \cdot \textbf{E}_{0}e^{i(\textbf{k}\cdot \textbf{r}-\omega t)} 
    = \textbf{k} \cdot \textbf{E}(\textbf{r},t) = 0
\end{equation*}
lo cual significa que el vector de onda $\textbf{k}$ es perpendicular al campo eléctrico. Lo mismo sucede para el campo magnético. De forma que los campos electromagnéticos son transversales a su vector de desplazamiento.\\
\indent Ahora puede verse también que el campo magnético es perpendicular al campo eléctrico tomando la ley de Faraday: $\nabla \times \textbf{E} = - \frac{i\omega}{c}\textbf{B}$. Entonces, si calculo el rotor del campo eléctrico\footnote{Ver apéndice 39.2} se llega a 
\begin{equation*}
    \nabla \times \textbf{E}(\textbf{r},t) =
    \textbf{k} \times \textbf{E}(\textbf{r},t) =
    \frac{\omega}{c}\textbf{B}
\end{equation*}
con lo cual, por las propiedades del producto vectorial, si tengo dos vectores $\textbf{v}$ y $\textbf{w}$ el producto vectorial entre ambos resulta perpendicular a cualquiera de los dos, es decir $\textbf{v}\times \textbf{w} \perp \textbf{w}$ y $\textbf{v}\times \textbf{w} \perp \textbf{v}$, del mismo modo resulta $\textbf{k}\times \textbf{E} \perp \textbf{E}$, entonces $\textbf{B} \perp \textbf{E}$. Finalmente se llega a 
\begin{equation*}
    \textbf{B}(\textbf{r},t) 
    = \sqrt{\mu \epsilon}\, \hat{k}\times \textbf{E}(\textbf{r},t)
    = n\, \hat{k} \times \textbf{E}(\textbf{r},t)
\end{equation*}
y resulta que $\textbf{E}$ y $\textbf{B}$ están en fase! \textbf{Obs:} la triada $\textbf{k}$, $\textbf{E}$ y $\textbf{B}$ definen una terna derecha. 

\subsubsection{Flujo de energía para una onda plana}
Se quiere calcular la energía que se lleva la onda electromagnética plana. Para eso lo que hay que ver es cómo queda el vector de Poyting para ondas planas.\\
\indent El vector de Poynting venía dado por
\begin{equation*}
    \textbf{S} = \frac{c}{4\pi}\textbf{E}\times \textbf{H}
\end{equation*}
pero no es trivial meter las soluciones halladas de los campos en el vector de Poynting porque este es cuadrático en los campos. Para eso hay que usar la siguiente propiedad: Sea dos magnitudes dependientes del tiempo $A(t)$ y $B(t)$ tales que
\begin{equation*}
    \left\{
        \begin{matrix}
            A(t) = \Re{A\,e^{-i\omega t}}\\
            \\
            B(t) = \Re{B\,e^{-i\omega t}}\\
        \end{matrix}
    \right.
\end{equation*}
defino la magnitud $S(t) = A(t)B(t)$ y vale\footnote{Ver apéndice 39.5}
\begin{equation*}
    S(t) = \frac{1}{2}\Re{ABe^{-2i\omega t}} + \frac{1}{2}\Re{AB^{*}}
\end{equation*}
o sea que la nueva magnitud se escribe como un término que oscila con el doble de la frecuencia original, más otro término que es constante en el tiempo. Por ejemplo, para la luz visible, $2\omega$ es muy rápido y solo se observa el valor medio temporal de la magnitud $S$, es decir $\mean{S(t)} = \frac{1}{2}\Re{AB^{*}}$.\\
\indent Usando esto para el vector de Poynting, se tiene que el valor medio del vector de Poynting es
\begin{equation*}
    \mean{\textbf{S}}
    = \frac{c}{8\pi}
    \sqrt{\frac{\epsilon}{\mu}}|E_{o}|^{2}\hat{k}
\end{equation*}
también puede escribirse la densidad de energía de cada uno de los campos como usando \eqref{ec:DensidadEnergiaEM}
\begin{equation*}
    u_{E} 
    = \frac{1}{8\pi} \textbf{E}\cdot \textbf{D}
    \Longrightarrow \mean{u_{E}} = \frac{\epsilon}{16\pi}|\textbf{E}_{0}^{2}|
\end{equation*}
\begin{equation*}
    u_{B} 
    = \frac{1}{8\pi} \textbf{B}\cdot \textbf{H}
    \Longrightarrow \mean{u_{B}} = \frac{\epsilon}{16\pi}|\textbf{B}_{0}^{2}|
\end{equation*}
y como $\textbf{B} = n\,\hat{k}\times \textbf{E}$, entonces vale que $|\textbf{B}| = n |\textbf{E}|$, de esta forma, el valor medio de la energía de los campos es la misma: $\mean{u_{E}} = \mean{u_{B}}$.
\begin{equation*}
    \mean{u} = \frac{\epsilon}{8\pi}|E_{0}|^{2}
\end{equation*}


%%%%%%%%%%%%%%%%%%%%%%%%%%%%%%%%%%%%%%%%%%%
%%%%%%%%%%%%%%%%%%%%%%%%%%%%%%%%%%%%%%%%%%%


\subsubsection{Polarización}
Dada las soluciones de ondas planas de los campos, de la forma $\textbf{E} = \textbf{E}_{0}\,e^{i(\textbf{k}\cdot \textbf{r} - \omega t)}$, como $\textbf{k}\cdot \textbf{E} = 0$, o sea que el campo eléctrico (y magnético) es perpendicular al vector de onda $\textbf{k}$, el campo eléctrico (y magnético) solo puede vivir en el plano perpendicular al desplazamiento de la onda electromagnética. Con lo cual, el campo eléctrico (y magnético) solo tiene dos grados de libertad en las direcciones del plano $\hat{e}_{1}$ y $\hat{e}_{2}$ perpendiculares entre sí. El campo magnético deberá cumplir lo mismo y además ser perpendicular al campo eléctrico.\\
\indent La polarización está dada por la figura imaginaria que puede dibujarse desde el vector resultante de la suma de las amplitudes del campo eléctrico y el campo magnético, en función del tiempo.\\
\indent Con esto en mente, la amplitud del campo eléctrico puede escribirse sobre su plano como
\begin{equation*}
    \textbf{E}_{0} = E_{0}^{(1)}\hat{e}_{1} + E_{0}^{(2)}\hat{e}_{2}
\end{equation*}
donde $\hat{e}_{1}$ y $\hat{e}_{2}$ son dos direcciones que forman base del plano.\\
\indent Se define la polarización de la onda plana como la trayectoria del campo eléctrico $\textbf{E}$ en el plano perpendicular al vector de onda $\textbf{k}$. Si escribo al campo eléctrico como
\begin{equation*}
    \textbf{E} = 
    \left(
        A\hat{e}_{1} + Be^{i(\phi_{2}-\phi_{1})}\hat{e}_{2}
    \right)\,e^{i(\textbf{k}\cdot\textbf{r} - \omega t + \phi_{1})}
\end{equation*}
puedo escribir la trayectoria como\footnote{Ver apéndice 40.2}
\begin{equation*}
    \left(x
        \frac{E_{2}}{B}
    \right)^{2}
    +
    \left(
        \frac{E_{1}}{A}
    \right)^{2}
    -\frac{2E_{1}E_{2}}{AB}\cos{(\phi)}
    = \sin^{2}{(\phi)}
\end{equation*}
%fin clase 17
%inicio clase 18
Para ver las trayectorias se puede escribir la expresión anterior en forma matricial (que resulta ser simétrica) y, además, como es simétrica se la puede escribir diagonalizada en la base de sus autovectores.  Para saber cómo es el \textit{dibujo} que se genera (la polarización) necesito saber el signo de los autovalores de la matriz, $\lambda_{1}$, $\lambda_{2}$. Miro el determinante y lo igualo a $\sin^{2}{(\phi)}$ y llego a\footnote{Ver apéndice 40.2}
\begin{equation*}
    \lambda_{1}{E'}_{1}^{2} +\lambda_{2}{E'}_{2}^{2} = \sin^{2}{(\phi)}
\end{equation*}
entonces, si $\lambda_{1}$ y $\lambda_{2}$ son positivos, puedo tener elipses (en el caso general) o círculos si $\lambda_{1} = \lambda_{2}$.\\
\indent Casos particulares:
\begin{itemize}
    \item $\sin{(\phi)} = 0$: Si el seno de $\phi$ es cero, entonces el coseno de $\phi$ debe ser $\pm 1$, de modo que se tiene
    \begin{equation*}
        \left(
            \frac{E_{1}}{A} \pm \frac{E_{2}}{B}
        \right)^{2}
        = 0
    \end{equation*}
    Es el caso de la polarización lineal,
    \item $\cos{(\phi)} = 0$ se tiene
    \begin{equation*}
        \left(
        \frac{E_{2}}{B}
    \right)^{2}
    +
    \left(
        \frac{E_{1}}{A}
    \right)^{2}
    = \sin{(\phi)}
    \end{equation*}
    y si $A = B$, se tiene polarización circular.
\end{itemize}
Ya se tienen las formas posibles de polarización: Elíptica, circular y lineal. Falta ver la evolución temporal, en qué sentido \textit{gira} la polarización. Sin entrar en detalles (ver si vale la pena hacer las cuentas), resulta que si $\sin{(\phi)} > 0$ es antihorario y si $\sin{(\phi)} < 0$ es horario.


%%%%%%%%%%%%%%%%%%%%%%%%%%%%%%%%%%%%%%%%%%%
%%%%%%%%%%%%%%%%%%%%%%%%%%%%%%%%%%%%%%%%%%%


\subsection{Reflexión y refracción de ondas planas en una interfaz}
Supongo el caso en el que tengo dos medios diferentes, con una interfaz en el plano $x$-$y$, sobre el cual incide una onda plana $\textbf{k}$ desde el medio (1) que choca con la interfaz formando un ángulo $i$ con el eje $z$. Entonces habrá una onda reflejada $\textbf{k}'$ (en el medio (1)) que se reflejará de la interfaz formando un ángulo $r$ con el eje $z$ y una onda transmitida $\textbf{k}''$ hacia el medio (2), formando un ángulo $t$ con el eje $z$. Los medios (1) y (2) tiene constantes dieléctricas y permitividades magnéticas $\epsilon$, $\mu$ y $\epsilon'$, $\mu'$ respectivamente. Defino una normal $\hat{n}$ que va en la misma dirección que el eje $z$.\\
\indent La onda incidente se escribe con los campos eléctrico y magnético como
\begin{equation*}
    \left\{
        \begin{matrix}
            \textbf{E} = \textbf{E}_{0}\,e^{i(\textbf{k}\cdot \textbf{r} - \omega t)}\\
            \\
            \textbf{B} = \sqrt{\mu\epsilon}\,\hat{k}\times \textbf{E}
        \end{matrix}
    \right.
\end{equation*}
puedo de la misma manera escribir a las ondas reflejadas y transmitidas en función de los parámetros conocidos y de los desconocidos. En este caso, los parámetros desconocidos son los ángulos $r$ y $t$, las amplitudes $\textbf{E}_{0}'$, $\textbf{E}_{0}''$, $\textbf{B}_{0}'$ y $\textbf{E}_{0}''$, los vectores de onda $\textbf{k}'$ y $\textbf{k}''$ y por últimos las frecuencias $\omega'$ y $\omega''$. Las magnitudes conocidas son las permitividades $\mu$, $\mu'$, $\epsilon$, $\epsilon'$, la amplitud inicial de los campos y el vector de onda incidente.\\
\indent entonces escribo la onda reflejada como
\begin{equation*}
    \left\{
        \begin{matrix}
            \textbf{E}' = \textbf{E}_{0}'e^{i(\textbf{k}'\cdot \textbf{r} - \omega' t)}\\
            \\
            \textbf{B}' = \sqrt{\mu\epsilon}\,\hat{k}'\times \textbf{E}'
        \end{matrix}
    \right.
\end{equation*}
y la onda transmitida
\begin{equation*}
    \left\{
        \begin{matrix}
            \textbf{E}'' = \textbf{E}_{0}''e^{i(\textbf{k}''\cdot \textbf{r} - \omega'' t)}\\
            \\
            \textbf{B}'' = \sqrt{\mu'\epsilon'}\,\hat{k}''\times \textbf{E}''
        \end{matrix}
    \right.
\end{equation*}
para hallar las frecuencias, los ángulos y las demás magnitudes que son incógnitas hay que hacer uso de las condiciones de contorno del problema. Primeramente miro las condiciones de contorno naturales que se pueden imponer dada la naturaleza ondulatoria de la teoría (y no necesariamente del electromagnetismo), que serían: La suma de las ondas incidentes y reflejadas de un lado de la interfaz, tiene que ser igual a la onda transmitida del otro lado de la interfaz. Esto se puede escribir como
\begin{equation*}
    \alpha\,e^{i(\textbf{k}\cdot \textbf{r} - \omega t)} +
    \beta\,e^{i(\textbf{k}'\cdot \textbf{r} - \omega' t)} +
    \gamma\,e^{i(\textbf{k}''\cdot \textbf{r} - \omega'' t)} = 0
\end{equation*}
donde $\alpha$, $\beta$ y $\gamma$ son constantes arbitrarias que representan alguna componente de las amplitudes de los campos, pero como no hacen al resultado que voy a obtener, las escribo de forma genérica. Notar que para satisfacer esta condición para un $\textbf{r}$ fijo (por ejemplo $\textbf{r} = 0$ para facilitar la forma de pensarlo), necesariamente las frecuencias $\omega$, $\omega'$ y $\omega''$ deben ser iguales. Es decir, no existe combinación lineal tal que se satisfaga esa igualdad para todo tiempo $t$, si no se cumple que $\omega = \omega' = \omega''$. De acá se deduce que \textbf{la frecuencia no cambia ni al reflejarse la onda ni al transmitirse}. \textbf{Obs:} Sí cambia la longitud de onda! la relación de dispersión en cada medio es diferente, y si la frecuencia no cambia, necesariamente lo hará la longitud de onda.\\
\indent Dado que las frecuencias son iguales, en la expresión de arriba se simplifican los términos temporales y queda 
\begin{equation*}
    \alpha\,e^{i\textbf{k}\cdot \textbf{r}} +
    \beta\,e^{i\textbf{k}'\cdot \textbf{r}} +
    \gamma\,e^{i\textbf{k}''\cdot \textbf{r}} = 0
\end{equation*}
ahora esta condición tiene que valer para $\textbf{r}$ perteneciente al plano de la interfaz. Si convenientemente pongo a la interfaz en el plano dado por $z = 0$, tiene que cumplirse que 
\begin{equation*}
    \textbf{k}\cdot\textbf{r} = 
    \textbf{k}'\cdot\textbf{r} = 
    \textbf{k}''\cdot\textbf{r}
\end{equation*}
de donde se puede escribir
\begin{equation*}
    k_{x}\hat{x} + k_{y}\hat{y} =
    k_{x}'\hat{x} + k_{y}'\hat{y} =
    k_{x}''\hat{x} + k_{y}''\hat{y}
\end{equation*}
simplificando la notación puedo escribirlo como $\textbf{k}_{yx} = \textbf{k}'_{yx} = \textbf{k}''_{yx}$. ¿Qué significa esto? Que las componentes de los vectores de onda incidente, reflejado y transmitido, son iguales en el plano $x$-$y$ o $z = 0$. O sea que las 3 ondas viven en el mismo plano, siempre. Físicamente significa que si una onda incide sobre una interfaz, la onda reflejada y la transmitida no pueden dispersarse sobre algún otro plano arbitrario, necesariamente viven en el mismo plano.\\
\indent Con estos supuestos ya se pueden empezar a hallar demás constantes. Por ejemplo, si defino el plano el $x$-$z$ (o $y = 0$) como el plano de la onda incidente, puedo analizar todo el problema desde ese plano (acá haría mucha falta un dibujito...) y la interfaz \textit{sale de la hoja} (plano $z = 0$ o $x$-$y$). Con simple trigonometría\footnote{Ver apéndice 41.2} se obtiene que el angulo reflejado es igual al ángulo incidente
\begin{equation*}
    \sin{(r)} = \sin{(i)}
\end{equation*}
y además se obtiene la Ley de Snell
\begin{equation*}
    n\sin{(i)} = n'\sin{t}
\end{equation*}
donde $n = \sqrt{\mu\epsilon}$ y $n' = \sqrt{\mu'\epsilon'}$ son los índices de refracción del medio (1) y el medio (2) respectivamente.\\
\indent Notar que de la relación de dispersión ($k^{2} = \mu\epsilon \omega^{2}/c^{2}$) ya sé cuánto valen $k'$ y $k''$, dado que solo dependen de los índices de refracción de los medios, entonces $k' = k$ y $k'' = \sqrt{\mu'\epsilon'}\frac{\omega}{c}$. Lo que puedo calcular con eso es la relación entre las componentes $x$ y $z$ de los vectores de onda como
\begin{equation*}
    k_{x}^{2} + k_{z}^{2} = \mu\epsilon\frac{\omega^{2}}{c^{2}}
    \Longrightarrow
    k_{x}^{2} = \mu\epsilon\frac{\omega^{2}}{c^{2}} - k_{z}^{2}
\end{equation*}
que vale lo mismo para $k'$ y para $k''$ solo hay que ponerle los primados a las permitividades.\\
\indent Hasta acá tengo definidas las frecuencias, los vectores de onda (cómo se propagan las ondas) y los ángulos. Sólo me hace falta conocer como se ven afectadas las amplitudes de las ondas. Esta es la parte más trabajosa. Para eso hay que hacer uso de las condiciones de contorno derivadas del electromagnetismo, que son, para el caso en el que no hay cargas libres
\begin{equation*}
    \left\{
        \begin{matrix}
            \left(
                \textbf{D}^{(2)} - \textbf{D}^{(1)}
            \right)\cdot \hat{n} = 0 
            &
            \Longrightarrow 
            \left[
                \epsilon'\textbf{E}_{0}' 
                -\epsilon 
                \left(
                    \textbf{E}_{0} + \textbf{E}_{0}''
                \right) 
            \right]\cdot \hat{n} = 0
            \\
            &\\
            \hat{n}\times 
            \left(
                \textbf{E}^{(2)} -\textbf{E}^{(1)}
            \right) = 0
            &
            \Longrightarrow
            \hat{n}\times
            \left[
                \textbf{E}_{0}'
                -
                \left(
                    \textbf{E}_{0} + \textbf{E}_{0}''
                \right)
            \right] = 0
            \\
            &\\
            \left(
                \textbf{B}^{(2)} - \textbf{B}^{(1)}
            \right)\cdot \hat{n} = 0 
            &
            \Longrightarrow 
            \left[
                \sqrt{\mu'\epsilon'}
                \hat{k}'\times \textbf{E}_{0}'
                - 
                \sqrt{\mu\epsilon}
                \left(
                    \hat{k}\times \textbf{E}_{0}
                    +
                    \hat{k}''\times \textbf{E}_{0}''
                \right)
            \right]\cdot \hat{n} = 0
            \\
            &\\
            \hat{n}\times 
            \left(
                \textbf{H}^{(2)} -\textbf{H}_{(1)}
            \right) = 0
            &
            \Longrightarrow
            \hat{n}\times
            \left[
                \sqrt{\frac{\epsilon'}{\mu'}}\,
                \hat{k}'\times \textbf{E}_{0}'
                -
                \sqrt{\frac{\epsilon}{\mu}}
                \left(
                    \hat{k}\times \textbf{E}_{0} 
                    +\hat{k}''\times \textbf{E}_{0}''
                \right)
            \right] = 0
        \end{matrix}
    \right.
\end{equation*}
Para usar estas condiciones de contorno correctamente, hay que tener en cuenta la polarización de las ondas electromagnéticas. Para eso lo que se hace es escribir a las amplitudes en sus dos componentes perpendiculares, pero hay que separar en dos casos. Un caso en el cual el campo eléctrico varía de forma transversal a su plano de incidencia (en este caso, el plano $y = 0$ o $x$-$z$), que se llama \textbf{transverso eléctrico}; y el caso en el que el campo magnético varía de forma transversal al plano de incidencia, que se llama \textbf{transverso magnético}. Las cuentas son análogas para ambos casos y son un garrón. Están resueltas en detalle en otro pdf.\\
\indent Las relaciones entre las amplitudes terminan siendo
\subsubsection*{Caso transverso Eléctrico}
\begin{equation}
    R 
    =
    \frac{E_{0}'}{E_{0}} 
    %=
    %\frac{
    %    \frac{n}{\mu}
    %    \cos{(i)}
    %    -
    %    \frac{n'}{\mu'}
    %    \sqrt{1- \left( \frac{n}{n'}\right)^2 \sin^{2}{(i)}}
    %}
    %{
    %    \frac{n}{\mu}
    %    \cos{(i)}
    %    +
    %    \frac{n'}{\mu'} 
    %    \sqrt{1- \left( \frac{n}{n'}\right)^2 \sin^{2}{(i)}}
    %}
    =
    \frac%
    {%
        n\cos{(i)}
        -
        \mu /\mu'
        \sqrt{n'^2- n^2 \sin^{2}{(i)}}
    }
    {%
        n\cos{(i)}
        +
        \mu/\mu' 
        \sqrt{n'^2 -n^2 \sin^{2}{(i)}}
    }
        \label{ec:REr}
\end{equation}
y para la onda transmitida
\begin{equation}
    T
    =
    \frac{E_{0}''}{E_{0}} 
    %=
    %\frac{
    %    2\frac{n}{\mu}
    %    \cos{(i)}
    %}
    %{
    %    \frac{n}{\mu}
    %    \cos{(i)}
    %    +
    %    \frac{n'}{\mu'} 
    %    \sqrt{1- \left( \frac{n}{n'}\right)^2 \sin^{2}{(i)}}
    %}
    =
    \frac
    {%
        2n\cos{(i)}
    }
    {%
        n\cos{(i)}
        +
        \mu/\mu' 
        \sqrt{n'^2- n^2 \sin^{2}{(i)}}
    }
        \label{ec:TEt}
\end{equation}
\subsubsection*{Caso transverso Magnético}
\begin{equation}
    R 
    =
    \frac{E_{0}'}{E_{0}} 
    %=
    %\frac{
    %    \frac{\mu}{n}
    %    \cos{(i)}
    %    -
    %    \frac{\mu'}{n'}
    %    \sqrt{1- \left( \frac{n}{n'}\right)^2 \sin^{2}{(i)}}
    %}
    %{
    %    \frac{\mu}{n}
    %    \cos{(i)}
    %    +
    %    \frac{\mu'}{n'} 
    %    \sqrt{1- \left( \frac{n}{n'}\right)^2 \sin^{2}{(i)}}
    %}
    =
    \frac%
    {%
        \frac{\mu}{\mu'}n'^{2}\cos{(i)}
        -
        n\sqrt{n'^2- n^2 \sin^{2}{(i)}}
    }
    {%
        \frac{\mu}{\mu'}
        n'^{2}\cos{(i)}
        +
        n\sqrt{n'^{2} - n^{2} \sin^{2}{(i)}}
    }
        \label{ec:RTM}
\end{equation}
y para la onda transmitida\footnote{CHEQUEAR ESTAS CUENTAS PORQUE ENCONTRÉ ERRORES}
\begin{equation}
    T
    =
    \frac{E_{0}''}{E_{0}} 
    =
    %\frac{
    %    2\frac{n}{\mu}
    %    \cos{(i)}
    %}
    %{
    %    \frac{n}{\mu}
    %    \cos{(i)}
    %    +
    %    \frac{n'}{\mu'} 
    %    \sqrt{1- \left( \frac{n}{n'}\right)^2 \sin^{2}{(i)}}
    %}
    %=
    \frac%
    {%
        2nn'\cos{(i)}
    }
    {%
        \frac{\mu}{\mu'}
        n'^{2}\cos{(i)}
        +
        n\sqrt{n'^{2} - n^{2} \sin^{2}{(i)}}
    }
        \label{ec:TTM}
\end{equation}
A estos se los llama coeficientes de Fresnell.\\
\indent \textbf{Obs:} Todos los denominadores tienen una raíz, cuyo argumento es $n'^{2} - n^{2}\sin^{2}{(i)}$. Entonces, si $n'^{2} - n^{2}\sin^{2}{(i)} \geq 0$, los coeficientes de Fresnell son reales, y la ondas incidente y reflejada/incidente y transmitida solo puede estar en fase o contrafase (contrafase significa tener signo contrario). Si $n'^{2} - n^{2}\sin^{2}{(i)} < 0$ los coeficientes son imaginarios se abre la puerta a varios fenómenos.\\
%
%
%fin clase 18
%%%%%%%%%%%%%%%%%%%%%%%%%%%%%%%%%%%%%%%%%%%
%%%%%%%%%%%%%%%%%%%%%%%%%%%%%%%%%%%%%%%%%%%
%inicio clase 19
%
%
\indent \textbf{Reflexión total interna:} Si $n'^{2} - n^{2}\sin^{2}{(i)} < 0$ necesariamente tiene que pasar que $n > n'$, es decir, el índice de refracción del medio de la onda incidente es \textbf{mayor} que el de la onda transmitida. O sea que los coeficientes de Fresnell no pueden ser nunca complejos si no se da que $n > n'$.\\
\indent ¿Y cómo veo que para un dado ángulo incidente $i$, en el caso de $n > n'$, hay reflexión total interna? Sucede que si grafica el módulo del valor de las amplitudes reflejadas y transmitidas en función del ángulo incidente, se obtiene que para un dado valor de $i$, la amplitud de la onda reflejada se vuelve $1$\footnote{Ver clase 19, minuto 11 - \url{https://www.youtube.com/watch?v=RXygFZ-svT4\%ADsicaExactasUBA}}. Sin embargo, también sucede que la amplitud de la transmitida vale más que $1$ y a medida que el ángulo sigue creciendo, decae. O sea que para un rango de ángulos de incidencia se tiene que $T + R > 1$! ¿Esto no rompe nada? No, porque la potencia transmitida promedia 0 y la potencia reflejada es máxima. Se llaman ondas evanescentes.\\
\indent Recapitulando: Si $n > n'$, existe $i$ para el cual la amplitud de la onda reflejada es $1$. Esto significa que la onda no atraviesa la interfaz, por más que efectivamente la amplitud del campo no sea $0$. En las cuentas se puede ver que si el campo transmitido es
\begin{equation*}
    \textbf{E}'' = \textbf{E}_{0}''e^{i(k_{x}x + k_{z}''z - \omega t)}
\end{equation*}
donde, como las ondas son coplanares y están en el plano $x$-$y$, para la onda incidente, reflejada y transmitida, $k_{x}$ vale lo mismo, pero $k_{z} \neq k_{z}''$. Además $\textbf{E}_{0}''$ ya es conocido por los coeficientes de Fresnell. El chiste con esa expresión es reescribirla de forma que quede evidente lo que está sucediendo con el campo transmitido. Para eso voy a ver que de la relación de dispersión puedo obtener el valor de $k_{z}''$ como\footnote{Ver apéndice 43}
\begin{equation*}
    {k''_{z}}^{2} = n'\frac{\omega^{2}}{c^{2}}
    \left(
        1 - \frac{n^{2}}{n'^{2}}\sin^{2}{(i)} 
    \right)
    \leq 0
    \quad
    \quad
    \mbox{Para}\ i \geq 0
\end{equation*}
O sea que es un número imaginario puro, en este sentido escribo $k_{z}' \equiv ik_{z}^{*}$ y lo reemplazo en la expresión del campo transmitido\footnote{Ver apéndice 43}
\begin{equation*}
    \textbf{E}'' = \textbf{E}_{0}''\,e^{i(k_{x}x - \omega t)}e^{-\gamma z}
\end{equation*}
y esta es la expresión importante porque queda evidenciado que el campo transmitido solo se propaga en la dirección $\hat{x}$, mientras que en la dirección $\hat{z}$ este decae exponencialmente! En la dirección $\hat{z}$ hay ondas evanescentes.\\
\indent Finalmente, mirando solo el caso transverso eléctrico, cuando se dan estas condiciones, los coeficientes de Fresnell quedan expresados como, para la transmitida
\begin{equation*}
    T = \frac{E_{0}''}{E_{0}} = 
    \frac%
        {%
            2n\cos{(i)}
        }
        {%
            n\cos{(i)}
            +
            i(n^{2}\sin^{2}{(i)} - n'^{2})^{1/2}
        }
\end{equation*}
que pertenece a los complejos. Observar que el módulo de esta cantidad no se vuelve 0 nunca. Ahora, para la reflejada
\begin{equation*}
    R = \frac{E_{0}'}{E_{0}} = 
    \frac%
        {%
            n\cos{(i)}
            -
            i(n^{2}\sin^{2}{(i)} - n'^{2})^{1/2}
        }
        {%
            n\cos{(i)}
            +
            i(n^{2}\sin^{2}{(i)} - n'^{2})^{1/2}
        }
\end{equation*}
qué también pertenece a los complejos y, calcular el módulo de la división es calcular la división de los módulos, el módulo del denominador es idéntico al módulo del numerador y la expresión se simplifica a $1$, que es lo esperado.\\
\indent Recién analicé el caso en el que la raíz que aparece en todos los coeficientes de Fresnell no es real. Hay otro caso de interés que es cuando, tanto transverso eléctrico como magnético, el numerador de las ondas reflejadas se anula.\\
\indent \textbf{Ángulo de Brewster:}  Recordar que $\sqrt{n'^{2} - n^{2}\sin^{2}{(i)}}$ se puede escribir como el seno del ángulo transmitido $n'\cos{(t)} = \sqrt{n'^{2} - n^{2}\sin^{2}{(i)}}$ por la ley de Snell. Entonces quiero ver qué pasa en los casos en que
\begin{equation*}
    \mbox{\textbf{(TE)}:}\quad
    n\cos{(i)} - n'\cos{(t)} = 0
\end{equation*}
y
\begin{equation*}
    \mbox{\textbf{(TM)}:}\quad
    n'\cos{(i)} - n\cos{(t)} = 0
\end{equation*}
va a resultar que en ese caso la polarización no es reflejada. Ambos casos, transverso eléctrico y transverso magnético, son de la misma forma, un número por coseno del ángulo incidente menos otro número por el coseno del ángulo transmitido. Eso se puede resumir para generalizarlo como 
\begin{equation*}
    \alpha \cos{(i)} - \beta\cos{(t)} = 0
\end{equation*}
y busco qué tiene que cumplirse para que eso pase. Usando relaciones trigonométricas, elevando al cuadrado y buscando raíces se llega a\footnote{Ver apéndice 43}
\begin{equation*}
    \tan^{2}{(i)} =
    \frac%
    {%
        n'^{2}(\alpha^{2} - \beta^{2})
    }
    {%
        \beta^{2}(n'^{2}-n^{2})
    }
\end{equation*}
donde resulta que para el caso \textbf{TE}, tiene que suceder que $\tan^{2}{(i)} = -1$, lo cual no sucede nunca. Esto significa que la polarización transverso eléctrica no puede anularse nunca. Sin embargo, para el caso \textbf{TM}, tiene que pasar que $\tan^{2}(i) = \frac{n'^{2}}{n^{2}}$, entonces si existe ángulo de incidencia $i$ que cumpla esa condición, la polarización transverso magnética se anula totalmente y la onda reflejada solo tiene polarización transverso eléctrica.



%%%%%%%%%%%%%%%%%%%%%%%%%%%%%%%%%%%%%%%%%%%
%%%%%%%%%%%%%%%%%%%%%%%%%%%%%%%%%%%%%%%%%%%


\subsection{Ondas en medios conductores}
Se quiere estudiar como se comportan los campos electromagnéticos dependientes del tiempo cuando entran en contacto con un conductor. Cómo serán los campos en el interior del conductor, las corrientes y las cargas. Para simplificar voy a considerar que el conductor es medio lineal isótropo y homogéneo. De esta forma, la ley de Ohm se resumen en $\textbf{j}_{l} = \sigma \textbf{E}$, donde $\textbf{j}_{l}$ son las corrientes libres en el conductor y $\sigma$ es la conductividad eléctrica del mismo.\\
\indent En ausencia de fuentes, las ecuaciones de Maxwell se escriben como
\begin{equation*}
    \left\{
        \begin{matrix}
            \nabla \cdot \textbf{D} = \nabla \cdot \textbf{E} = 0,
            &
            \nabla \times \textbf{H} = 
            \frac{4\pi}{c}\sigma\textbf{E}
            + \frac{1}{c}\frac{\partial\textbf{D}}{\partial t}
            \\
            & \\
            \nabla \cdot \textbf{B} = 0,
            &
            \nabla \times \textbf{E} =
            -\frac{1}{c}
            \frac{\partial \textbf{B}}{ \partial t}
        \end{matrix}
    \right.
\end{equation*}
Haciendo $\nabla\times\nabla\times \textbf{E}$ y usando que la divergencia de $\textbf{E} = 0$, se tiene
\begin{equation*}
    \nabla \times \nabla \times \textbf{E} = 
    -\frac{\mu \sigma}{c^{2}}4\pi\frac{\partial \textbf{E}}{\partial t}
    -\frac{\mu \epsilon}{c^{2}}\frac{\partial^{2} \textbf{E}}{\partial t^{2}}
    = \nabla^{2}\textbf{E}
\end{equation*}
Donde el término con la derivada segunda respecto al tiempo es un término de inducción, o sea, si la derivada primera respecto a $t$ no estuviera, recuperamos la ecuación de ondas para los campos. Y el término con la derivada primera respecto al tiempo es un término disipativo, es decir, si no estuviera la derivada segunda respecto al tiempo, recuperamos una ecuación difusión para los campos.\\
\indent Pasando al espacio Fourier (que no es más que hacer la derivada temporal y cambiar la dependencia en $t$ de los campos por $\omega$), baja en la derivada primera respecto a $t$ un $i\omega$ y en la derivada segunda baja un $-\omega^{2}$. Agrupando términos
\begin{equation*}
    \nabla^{2}\textbf{E} 
    + \mu\epsilon
    \frac{\omega^{2}}{c^{2}}
    \left(
        1 + i\frac{4\pi\sigma}{\omega \epsilon}
    \right)\textbf{E} = 0
\end{equation*}
donde se define la constante dieléctrica compleja del medio como 
\begin{equation*}
    \hat{\epsilon} = 
    \epsilon
    \left(
        1 + i\frac{4\pi\sigma}{\omega \epsilon}
    \right)
\end{equation*}
de acá se halla una nueva relación de dispersión para las ondas electromagnéticas en los conductores que resulta
\begin{equation*}
    k^{2} =  
    \epsilon\mu\frac{\omega^{2}}{c^{2}}
    \left(
        1 + i\frac{4\pi\sigma}{\omega \epsilon}
    \right)
\end{equation*}
que pertenece a los complejos. Tiene una parte real y una imaginaria. Si reemplazo ese $k$ en la expresión para los campos, la parte imaginaria de la relación de dispersión se transforma en una exponencial real, lo que da lugar a decaimientos. Y resulta que ese decaimiento es función de la frecuencia $\omega$ de la onda electromagnética incidente. Esto quiere decir que cuando una onda electromagnética incide en un medio conductor, dependiendo de su frecuencia $\omega$\footnote{Si $\omega$ es real}, penetrará más o menos dentro del conductor. Esto se debe a que la velocidad con la que se reacomodan las cargas libres dentro del conductor para anular el campo eléctrico es finita, y depende del material. Con lo cual, frecuencias mucho más altas que la velocidad de repuesta del conductor podría generar que las cargas libres del material no lleguen nunca a reacomodarse para anular el campo y que, podría suceder, promedien un desplazamiento nulo transformando al conductor en un dieléctrico con sus \textit{cargas libres fijas}. Por el contrario, si la frecuencia de la onda electromagnética es mucho menor que la velocidad de repuesta del material, las cargas siempre pueden reacomodarse para anular el campo.\\
\indent Veamos los límites con las cuentas. Sea $\tau_{i} = \epsilon/2\sigma$ el tiempo de respuesta o inercia del material y quiero ver la relación $T/\tau$, donde $T$ es el periodo de la onda electromagnética, dado por $T = 2\pi/\omega$
\begin{itemize}
    \item Frecuencias altas (periodos cortos) respecto a la respuesta del material: $\frac{T}{\tau_{i}} = \frac{4\pi\sigma }{\omega \epsilon} << 1$. La parte imaginaria de la constante dieléctrica compleja del medio se vuelve despreciable y se convierte en casi todo constante dieléctrica del medio. La relación de dispersión se vuelve no dispersiva y corresponde a la de un medio material dieléctrico. El material no llega a acomodar sus cargas para anular el campo eléctrico y las corrientes generadas en él son pequeñas.
    \item Frecuencias bajas (periodos largos) respecto a la respuesta del material: $\frac{T}{\tau_{i}} = \frac{4\pi\sigma }{\omega \epsilon} >> 1$. La parte imaginaria de la constate dieléctrica compleja se vuelve dominante y la relación de dispersión termina siendo dispersiva. Es decir, las ondas \textit{se desarman} al ingresar en el medio. El material llega a reacomodar sus cargas para anular el campo eléctrico y las corrientes generadas por el desplazamientos de las cargas no son despreciables.
\end{itemize}
Si $\omega$, $\mu$ y $\epsilon$ son reales, entonces el vector $\textbf{k}$ es complejo y puedo escribir $\textbf{k} = \textbf{k}_{R} + i\textbf{k}_{C}$. Reemplazo en la expresión para el campo eléctrico y tendría (como dije antes)
\begin{equation*}
    \textbf{E}(\textbf{r},t) 
    = 
    \underbrace{%
        \textbf{E}_{0}\,e^{-\textbf{k}_{C}\cdot \textbf{r}}
    }_{\mbox{\tiny{Amplitud}}}
    \,e^{i(\textbf{k}_{R}\cdot\textbf{r} - \omega t})
\end{equation*}
donde se ve explícitamente la atenuación en la amplitud del campo eléctrico $\textbf{E}_{0}$ por la exponencial real. Además, la onda pierde energía dentro del conductor ya que vimos que la ecuación para los campos en su interior tiene un término disipativo. En efecto esta energía se pierde en calor por disipación óhmica. Además, está bueno notar que $\textbf{k}_{C}$ y $\textbf{k}_{R}$ no necesariamente son paralelos, con lo cual, hay una dirección en la que se da la atenuación y otra dirección distinta en la que se desplaza la onda.\\
\indent Miro el caso en el que el campo tiene una componente en $\textbf{k}_{R}$, supongo que $\hat{x}$. Entonces puedo escribir el campo eléctrico como
\begin{equation*}
    \textbf{E} = E_{x}(x,t)\hat{x} + \textbf{E}_{\perp}(x,t)
\end{equation*}
y
\begin{equation*}
    \textbf{H} = H_{x}(x,t)\hat{x} + \textbf{H}_{\perp}(x,t)
\end{equation*}
y $\textbf{k}_{C}\parallel \textbf{k}_{R}$. Por ejemplo, el caso de incidencia normal al conductor. Entonces la atenuación del campo se da en la misma dirección de propagación. En este caso estoy considerando (no sé si como excepción o qué), que la dirección del campo coincide con la dirección de propagación (o sea que acá no vale más que $\textbf{E} \perp \textbf{k}$?) o al menos es así para alguna componente del campo. Mirando solo la coordenada $x$ del campo y su dependencia temporal, integrando se puede llegar a \footnote{Ver apéndice 44.6}
\begin{equation*}
    E_{x}(t) = E_{0_{x}}\,e^{-\frac{4\pi\sigma}{\epsilon}t}
\end{equation*}
donde $\frac{4\pi\sigma}{\omega}$ es el tiempo característico para que el medio anule el campo longitudinal en el interior del conductor por disipación óhmica.\\
\indent Además, como $\nabla \times \textbf{E} = -\frac{1}{c}\frac{\partial \textbf{B}}{\partial t}$ y en la componente $x$ se tiene que $\frac{\partial B_{x}}{\partial t} = 0$ entonces $H_{x} = cte$. O sea que, si bien puedo tener campo eléctrico uniforme (espacialmente) dentro del conductor, este decae en el tiempo exponencialmente según su tiempo característico $\tau = \epsilon/4\pi\sigma$. Pero por otro lado, el campo magnético dentro del conductor podrá verse atenuado o no dentro del conductor, pero se mantiene contante a lo largo del tiempo.\\
\indent Se quiere ver ahora los campos transversales \footnote{Ver cuentas}, se llega a 
\begin{equation*}
    \textbf{E}_{\perp} 
    = \textbf{E}_{0}\,e^{-k_{0}\beta x}\,
    e^{i(k_{0}\alpha x - \omega t)}
\end{equation*}
con $k_{0} = \frac{\sqrt{\mu\epsilon}\omega}{c}$
\begin{equation*}
    \left\{
        \begin{matrix}
            \alpha = \frac{1}{\sqrt{2}}
            \left[
                \sqrt{1 + 
                \left(
                    \frac{4\pi\sigma}{\omega \epsilon}
                \right)^{2}} +1
            \right]^{1/2}\\
            \\
            \beta = \frac{1}{\sqrt{2}}
            \left[
                \sqrt{1 + 
                \left(
                    \frac{4\pi\sigma}{\omega \epsilon}
                \right)^{2}}  - 1
            \right]^{1/2}
        \end{matrix}
    \right.
\end{equation*}


%%%%%%%%%%%%%%%%%%%%%%%%%%%%%%%%%%%%%%%%%%%
%%%%%%%%%%%%%%%%%%%%%%%%%%%%%%%%%%%%%%%%%%%


\subsubsection{Casos límites}
\begin{itemize}
    \item \textbf{Mal conductor} $\frac{4\pi\sigma}{\epsilon\omega} << 1$. La relación de dispersión queda 
    \begin{equation*}
        k = \sqrt{\mu\epsilon}\frac{\omega}{c} 
        + i\frac{2\pi\sigma}{\epsilon}\sqrt{\mu\epsilon c}
    \end{equation*}
    y entonces
    \begin{equation*}
        \textbf{E}_{\perp} = \textbf{E}_{0}
        \,e^{\frac{2\pi\sigma}{\epsilon c}\sqrt{\mu \epsilon c}x}
        \,e^{i(k_{0}x - \omega t)}
    \end{equation*}
    se propaga como en el vacío pero decae con $x$. Notar que un paquete se propaga y atenúa sin deformarse (la relación de dispersión es lineal en $\omega$).
    \item \textbf{Buen conductor} $\frac{4\pi\sigma}{\epsilon\omega} >> 1$, la relación de dispersión queda
    \begin{equation*}
        k = \sqrt{\mu \epsilon}\frac{\omega}{c}
        \left(
            \frac{2\pi\sigma}{\epsilon\omega}
        \right)^{1/2}(1+i)
        = \frac{\sqrt{2\pi\sigma\mu\omega}}{c}(1+i)
    \end{equation*}
    donde se define $1/\delta = \frac{\sqrt{2\pi\sigma\mu\omega}}{c}$ longitud pelicular o de penetracion. El campo queda
    \begin{equation*}
        \textbf{E}_{\perp} = \textbf{E}_{0}\,
        e^{-\frac{x}{\delta}}\,
        e^{i\left(\frac{x}{\delta} - \omega t\right)}
    \end{equation*}
    Además, ahora $\omega = \omega(k)$ tal que $v_{f} = v_{f}(k)$ y el medio es dispersivo.
\end{itemize}


%fin clase 19
%%%%%%%%%%%%%%%%%%%%%%%%%%%%%%%%%%%%%%%%%%%%%%%%
%%%%%%%%%%%%%%%%%%%%%%%%%%%%%%%%%%%%%%%%%%%%%%%%
%%%%%%%%%%%%%%%%%%%%%%%%%%%%%%%%%%%%%%%%%%%%%%%%
%inicio clase 20

\newpage
\section{Teoría de la relatividad especial}


%%%%%%%%%%%%%%%%%%%%%%%%%%%%%%%%%%%%%%%%%%%
%%%%%%%%%%%%%%%%%%%%%%%%%%%%%%%%%%%%%%%%%%%


\subsection{Transformaciones de Galileo y covarianza de Newton}
Las ecuaciones de la mecánica son invariantes\footnote{i.e.: Covariante: que varía como un escalar} frente a transformaciones de Galileo, es decir, no cambian de forma. La ley de Newton que se escribe $\textbf{F} = m\textbf{a}$ en un sistema de referencia $S$, si me paro en un sistema de referencia rotado $S'$ la ley de Newton se escribirá como $\textbf{F}' = m\textbf{a}'$, donde $\textbf{F}'$ y $a'$ son la transformación adecuada de los vectores originales $\textbf{F}$ y $\textbf{a}$, pero que lo importante es que la forma fundamental de la ecuación no cambió.\\
\indent Las transformaciones de Galileo se escriben en general como
\begin{equation*}
    \left\{
        \begin{matrix}
            x' = x - vt\\
            y' = y\\
            z' = z\\
            t' = z
        \end{matrix}
    \right.
\end{equation*}
o en forma vectorial general
\begin{equation*}
    \left\{
        \begin{matrix}
            \textbf{r}' = \textbf{r} - \textbf{v}t\\
            t' = t
        \end{matrix}
    \right.
\end{equation*}
Se introduce el concepto de sistema de referencia inercial: Es el sistema en el que un cuerpo en ausencia de fuerzas externa se mueve con velocidad $\textbf{v} = cte$.\\
\indent Resulta que las ecuaciones de Maxwell no son invariantes frente a transformaciones de Galileo, y esto fue un problema porque entraba en conflicto con la mecánica Newtoniana que sí lo era y funcionaba bien. Pero a su vez, las ecuaciones de Maxwell también funcionaban bien. Entonces, ¿qué estaba pasando?\\
\indent Para ver que las ecuaciones de Maxwell no son invariantes de Galileo, por ejemplo, se puede ver que para el caso de los potenciales $\varphi$ y $\textbf{A}$, escritos en el gauge de Lorenz en el vacío y sin fuentes, satisfacen ecuaciones de la pinta
\begin{equation*}
    \nabla^{2}\psi - \frac{1}{c^{2}}\frac{\partial^{2}\psi}{\partial t^{2}} = 0
\end{equation*}
que son ecuaciones de onda homogéneas, que tienen solución unidimensional $\psi = f(x \pm ct)$ (ondas viajeras). Si le aplico una transformación de Galileo a la ecuación para los potenciales tengo\footnote{Ver apéndice 45.3}
\begin{equation*}
    \nabla^{2}\psi - \frac{1}{c^{2}}\frac{\partial^{2}\psi}{\partial t^{2}} = 0
    \Longrightarrow 
    \frac{\partial^{2} \psi}{\partial x'^{2}} 
    - \frac{1}{c^{2}}
    \left(
        \frac{\partial^{2}\psi}{\partial t'^{2}}
        +v^{2}\frac{\partial^{2}\psi}{\partial x'^{2}}
        -2v\frac{\partial^{2}\psi}{\partial x'\partial t'}
    \right)
    = 0
\end{equation*}
con lo cual la ecuación no queda invariante, cambió de forma al aplicarle la transformación. Pero, ¿Por qué es un problema que las ecuaciones de onda del electromagnetismo no sean invariantes de Galileo, pero no lo es para las ecuaciones de onda de la mecánica, por ejemplo, de la cuerda? La respuesta es que en la derivación de la ecuación de onda para una cuerda, se usan las ecuaciones de Newton y se calcula la perturbación en las amplitudes de oscilación, parándose en un sistema de referencia privilegiado que es el sistema de referencia fijo a cada punto de la cuerda. Dicho esto, es natural pensar que al aplicar una transformación de Galileo a otro sistema de referencia las ecuaciones de onda de la cuerda cambien. Sin embargo, para las ondas del electromagnetismo, que pueden desplazarse en el vacío, no siempre existe un medio privilegiado en donde pararse, de forma que no se podía explicar por qué no eran invariantes de Galileo.\\
\indent Para solventar esto, se introdujo el \textbf{éter}, y se propuso que las ecuaciones de Maxwell estaban escritas en el sistema de referencia fijo al éter, en reposo. Sin embargo, hubo tres experimentos que tenían conflictos con esta hipótesis:
\begin{itemize}
    \item Mediciones de la velocidad de la luz en fluidos en movimiento (Fizeau 1859)
    \item El experimento del interferómetro de Michelson-Morley (1887).
    \item La aberración estelar (corrimiento en la posición aparente de los astros)
\end{itemize}
El primer experimento se explicó asumiendo que el fluido arrastra parcialmente al éter, con una efectividad que depende del $n$ del medio.\\
\indent El segundo experimento falló intentando medir el movimiento de la tierra respecto al eter. Se solucionó con la contracción de Fitzgerald - Lorentz
\begin{equation*}
    L = L_{0}\sqrt{1 - \frac{v^{2}}{c^{2}}}
\end{equation*}
entonces, a principios del siglo XX las explicaciones posibles a estos problemas eran
\begin{itemize}
    \item Las ecuaciones de Maxwell son incorrectas ya que deberían ser invariantes de Galileo
    \item Las ecuaciones de Maxwell son válidas en el sistema de referencia fijo al éter, la descripción de la interacción de los campos y la materia con el éter seguía una gran cantidad de reglas complejas.
    \item Todas las ecuaciones de la física son covariantes, pero no ante transformaciones de Galileo. Hay alguna otra transformación que es la correcta.
\end{itemize}


%%%%%%%%%%%%%%%%%%%%%%%%%%%%%%%%%%%%%%%%%%%
%%%%%%%%%%%%%%%%%%%%%%%%%%%%%%%%%%%%%%%%%%%


\subsection{Postulados de la relatividad especial}
Einstein consideró la tercera opción y propuso:
\begin{itemize}
    \item Las ecuaciones de la física son las mismas (covariantes) en todos los sistemas en MRU respecto a otro (sistemas inerciales)
    \item La velocidad de la luz en el vacío es la misma en todos los sistemas de referencia y es independiente del movimiento de la fuente.
\end{itemize}
Estos postulados excluyen las transformaciones de Galileo. En estas transformaciones, la ley de transformación para la velocidad se transforma en
\begin{equation*}
    x' = x - vt
    \quad\Longrightarrow\quad
    \mathbf{V}' = \textbf{v} - v
\end{equation*}\footnote{No son vectores, son negritas nomás}
$x'$ es la posición en el sistema primado que se mueve con velocidad $v$, y $\mathbf{V}'$ es la velocidad en el sistema primado (que se mueve con velocidad $v$). Dado el segundo postulado de Einstein, que la velocidad de la luz es igual para todos los sistemas de referencia inerciales, si le aplicamos una transformación de Galileo a una linterna que se mueve con MRU, la velocidad de la luz emitida por la linterna, vista desde un sistema de referencia primado que se mueve con velocidad $v$ sería $\mathbf{V}' = \mathbf{v} - c$, donde $\mathbf{v}$ es la velocidad de movimiento de la linterna. Esto ya viola el segundo postulado de Einstein.\\
\indent \textbf{Definición de suceso:} Queda definido por $t$, $\textbf{r}$ en un sistema de referencia.\\
\indent Entonces, sea el suceso de emitir un pulso en $x_{1}$, $y_{1}$, $z_{1}$, $t_{1}$. Sea otro suceso recibir el pulso en un dado tiempo $t_{2}$, de forma tal que el pulso viajó una distancia $c(t_{2}-t_{1})$ y llegó a $x_{2}$, $y_{2}$, $z_{2}$. Entonces vale decir
\begin{equation*}
    c^{2}(t_{2}-t_{1}) - (x_{2}-x_{1})^{2} -(y_{2}-y_{1})^{2}-(z_{2}-z_{1})^{2} = 0
\end{equation*}
no es mas que escribir la distancia $\Delta \textbf{r}$ recorrida por el pulso usando las coordenadas espaciales iniciales y finales, pero también usando la velocidad de la luz (del pulso) y el tiempo transcurrido entre ambos sucesos.\\
\indent Supongo que tengo un observador en un sistema inercial $S'$ que observa ambos sucesos. Para este, el pulso partió de $x'_{1}$, $y'_{1}$, $z'_{1}$ y $t'_{1}$ y llegó en $x'_{2}$, $y'_{2}$, $z'_{2}$ y $t'_{2}$. Puedo escribir lo análogo a lo anterior pero en el sistema primado, de forma de tener
\begin{equation*}
    c^{2}(t'_{2}-t'_{1}) - (x'_{2}-x'_{1})^{2} -(y'_{2}-y'_{1})^{2}-(z'_{2}-z'_{1})^{2} = 0
\end{equation*}
Dado el segundo postulado de Einstein, como la velocidad de la luz es la misma para todo sistema inercial, en el sistema $S'$, la velocidad de la luz sigue siendo $c$ y es la constante que se repite en ambos sistemas de referencia. Notar que dados dos sucesos, vistos desde dos sistemas de referencia diferentes, la magnitud definida antes no vario. Es decir, encontré una magnitud que es constante para cualquier sistema de referencia inercial, dados dos sucesos. Entonces, dados dos sucesos cualesquiera se define el intervalo:
\begin{equation*}
    \Delta S_{12}^{2} = c^{2}\Delta t_{12}^{2} -\Delta x_{12}^{2} - \Delta y_{12}^{2} - \Delta z_{12}^{2}
\end{equation*}
y 
\begin{equation*}
    dS^{2} = c^{2}dt^{2} - dx^{2} - dy^{2} - dz^{2}
\end{equation*}
que es una noción de distancia en una geometría pseudoeuclidiana. Como ya vi que $dS = 0 \Longleftrightarrow dS' = 0$, en otro sistema de referencia tiene que ser $0$, con lo cual la transformación que lleva de un sistema de referencia a otro debe ser homogénea, es decir, no puede tener término independiente\footnote{O sea que si uno es cero, que el otro sea distinto de cero está prohibido} entonces la transformación debe ser de la pinta $dS^{2} = \alpha dS'^{2}$ y si $v \longrightarrow v'$ y $S \longrightarrow S'$ entonces $\alpha = 1$. Con lo cual  
\begin{equation*}
    \boxed{
        dS^{2} = dS'^{2}
    }
\end{equation*}
es un invariante relativista.\\
\indent Entonces, ¿el intervalo es siempre $0$? No, por ejemplo si el suceso inicial es que veo como una pelota sale disparada con velocidad $v$ ($< c$), y el segundo suceso es que la pelota impacta sobre una pared, el intervalo se escribe como $\Delta S^{2} = c^{2}\Delta t^{2} - v^{2}\Delta t^{2} > 0$ ya que la velocidad de la luz es mayor a $v$.\\
\indent Ahora lo que queremos es lograr encontrar cuáles son las transformaciones correctas para que todas las ecuaciones de la física sean covariantes (o invariantes frente a esta transformación). Consideremos ahora que el pulso se emite en $x_{1} = y_{1} = z_{1} = 0$, $t_{1} = 0$ y que en ese instante el origen de coordenadas de $S$ y $S'$ coinciden, con $t' = t$. Luego, como el intervalo $S$ es invariante (vale lo mismo en cualquier sistema de referencia inercial)
\begin{equation*}
    c^{2}t^{2} - x^{2} - y^{2} - z^{2} =
    c^{2}t'^{2} - x'^{2} - y'^{2} - z'^{2} 
\end{equation*}
elijo arbitrariamente que el desplazamiento relativo entre los sistemas de referencia sea en la dirección $\hat{x}$ (Boost en $\hat{x}$), de forma que puedo tomar $y' = y$ y $z' = z$ ya que no cambian. Entonces sólo hay que igualar los primeros dos términos, como sigue
\begin{equation*}
    \underbrace{
        c^{2}t^{2}
    }_{(x^{0})^{2}}
    -x^{2} = 
    \underbrace{
        c^{2}t'^{2}
    }_{(x'^{0})^{2}}
    -x'^{2}
\end{equation*}
y ahora, ¿Cuál es la transformación más general que puedo aplicarle a $c^{2}t^{2} - x^{2}$ para ir a $c^{2}t'^{2} - x'^{2}$ y que se preserven las distancias (dado que el intervalo es invariante)? resulta que, por alguna razón que desconozco, la transformación más general es una rotación hiperbólica(????????????
\begin{equation*}
    \left\{
        \begin{matrix}
            x'^{0} = x^{0}\cosh{(\xi)} - x\sinh{(\xi)}\\
            \\
            x' = -x^{0}\sinh{(\xi)} + x\cosh{(\xi)}
        \end{matrix}
    \right.
\end{equation*}
queremos saber cuánto vale en ángulo $\xi$ en función de la velocidad relativa entre los sistemas de referencia. Si me paro en el origen del sistema $S$, estoy en la posición $x=0$ de este sistema, o sea que en este caso en particular, en la transformación de arriba los términos con $x$ mueren. En el sistema $S'$, la coordenada $x'^{0} = ct' = x^{0}\cosh{\xi}$\footnote{Por definición, $x'^{0} = ct'$}, y ¿Cuánto vale la posición del origen del sistema $S$ visto desde el sistema $S'$? Será menos la velocidad de desplazamiento de $S'$ respecto de $S$, por el tiempo primado: $x' = -vt' = -x^{0}\sinh{(\xi)}$. Lo que se hizo fue siempre mirar el mismo punto: El origen de coordenadas del sistema $S$, desde ambos sistemas de referencia. Dividiendo una respecto a la otra se llega 
\begin{equation*}
    \tanh{(\xi)} = \frac{v}{c} \equiv \beta
\end{equation*}
y defino 
\begin{equation*}
    \gamma = \frac{1}{\sqrt{1 - \beta^{2}}}
\end{equation*}
finalmente la transformación se escribe como
\begin{equation*}
    \left\{
        \begin{array}{lll}
            x'&=& \gamma (x^{0} - \beta x^{1})\\
            x'^{1}&=& \gamma(x^{1} - \beta x^{0})\\
            x'^{2}&=& x^{2}\\
            x'^{3}&=& x^{3}
        \end{array}
    \right.
\end{equation*}
y es la transformación de Lorentz, donde $x^{0} = ct$, $x^{1} = x$, $x^{2} = y$, $x_{3} = z$. Notar que se puede llegar a las mismas transformaciones planteando una transformación lineal arbitraria, con coeficientes libres y resolviendo de forma de cumplir que la longitud del intervalo sea la misma. Usar las rotaciones hiperbólicas fue un truco que funciona para llegar al resultado correcto. Las transformaciones inversas salen de cambiar $\beta \longrightarrow -\beta$ y $x \longrightarrow x'$. Se introduce entonces el cuadrivector posición 
\begin{equation*}
    x^{\mu} = (x^{0}, x^{1}, x^{2}, x^{3}) = (x^{0}, \textbf{x})\quad\quad\quad
    \mu = 0,1,2,3
\end{equation*}


%%%%%%%%%%%%%%%%%%%%%%%%%%%%%%%%%%%%%%%%%%%
%%%%%%%%%%%%%%%%%%%%%%%%%%%%%%%%%%%%%%%%%%%


\subsection{Transformaciones de Lorentz}
Notar que $v < c$, para el límite de $c \to \infty$ las transformaciones de Lorentz se reducen a las transformaciones de Galileo.\\
\indent \textbf{Contracción espacial:} Consideremos una barra de longitud (propia) $L_{0}$ en reposo en el sistema de referencia $S$. En el sistema de referencia $S'$ también queremos medir la longitud de la barra. Para eso basta con medir $x'_{2}$ y $x'_{1}$ (los extremos de la barra), pero en el mismo instante de tiempo $t'$, medir $x'_{1}$ a $t'_{1}$ y $x'_{2}$ a $t'_{2}$ no tiene sentido. Usando la transformación inversa
\begin{equation*}
    x_{1} = \gamma(x'_{1} + \beta t')
\end{equation*}
\begin{equation*}
    x_{2} = \gamma(x'_{2} + \beta t')
\end{equation*}
entonces $x_{2} - x_{1} = L_{0} = \gamma(x'_{2} - x'_{1}) = \gamma L'$ y resulta que $L ' = L_{0}\sqrt{1 - \frac{v^{2}}{c^{2}}}$, que es la contracción de Fitzgerald-Lorentz. Con lo cual, la longitud de la barra en la dirección del desplazamiento del observador en el sistema de referencia $S'$, se contrae respecto a la longitud medida en el sistema de referencia fijo a la barra.\\
\indent \textbf{Dilatación temporal:} Ahora consideremos un reloj en reposo en el sistema $S'$, que se mueve con velocidad $v$ respecto de $S$. Sean dos sucesos que ocurren en $x'$, $y'$, $z'$ en $t'_{1}$ y $t'_{2}$, o sea que ambos sucesos ocurren en la misma posición del sistema de referencia $S'$. El tiempo transcurrido en $S'$ (tiempo propio) es $\Delta t' = t'_{2} - t'_{1}$. Quiero conocer el tiempo transcurrido entre los mismos sucesos para un reloj en el sistema de referencia $S$, $\Delta t$, que ve el reloj moverse. Para eso uso la inversa de las transformaciones de Lorentz (cambiando $\beta \longrightarrow - \beta$) escribo cada uno de los tiempos y luego los resto y llego
\begin{equation*}
    \begin{matrix}
    ct_{1} = \gamma(ct'_{1} + \beta x'_{1})\\
    \\
    ct_{2} = \gamma(ct'_{2} + \beta x'_{2})\\
    \\
    \Longrightarrow \Delta t = \frac{\Delta t'}{\sqrt{1 - \frac{v^{2}}{c^{2}}}}
    \end{matrix}
\end{equation*}
esto implica que $\Delta t < \Delta t'$, ya que $\gamma < 1$. ¿Qué significa esto? Que el observador en el sistema referencia $S$ que ve mover al reloj, percibe una cantidad de tiempo mayor a la que percibe el observador que se mueve junto con el reloj. Un ejemplo de esto es el tiempo de vida de los muones. El tiempo de vida (propio) de un muon es muy bajo, sin embargo, estos se la arreglan para venir del espacio, atravesar la atmósfera y que sean observables por detectores. Si bien el tiempo de vida propio de un muon no sería suficiente para que a la velocidad que se desplazan, llegan a la superficie de la tierra, esto efectivamente sucede porque desde un sistema de referencia ajeno al muón, el tiempo que tarda en decaer es mayor.\\
\indent Puedo definir entonces el tiempo propio $\Delta t'$ como
\begin{equation}
    \Delta \tau \equiv
    \Delta t' =
    \sqrt{1 - \frac{v^{2}}{c^{2}}}\,\Delta t
        \label{ec:TiempoPropio}
\end{equation}


%%%%%%%%%%%%%%%%%%%%%%%%%%%%%%%%%%%%%%%%%%%
%%%%%%%%%%%%%%%%%%%%%%%%%%%%%%%%%%%%%%%%%%%


\subsubsection{Transformación de la velocidad}
En un caso más general, en el que hay un observador moviéndose con velocidad $\textbf{v}$ en el sistema de referencia $S$. Quiero ver cómo se transforma esta velocidad para un observador en el sistema de referencia que se mueve con velocidad $v$. La velocidad por definición (en el sistema $S$) es $\textbf{v} = \Delta \textbf{x}/\Delta t$ y quiero transformarla $\textbf{v}' = \Delta \textbf{x}'/\Delta t'$. Entonces usando Lorentz tenemos
\begin{equation*}
    \begin{matrix}
        \Delta x' = \gamma(\Delta x - \beta c \Delta t)\\
        c\Delta t' = \gamma(c\Delta t - \beta \Delta x)\\
        \Delta y' = \Delta y\\
        \Delta z' = \Delta z
    \end{matrix}
\end{equation*}
entonces, dividiendo $\Delta x'$ por $\Delta t'$ y haciendo las cuentas se llega las transformaciones para las velocidades\footnote{Ver apéndice 47}
\begin{equation*}
    \left\{
        \begin{matrix}
            v'_{x} = \frac{v_{x} - v}{1 - \beta \frac{v_{x}}{c}}\\
            v'_{y} = 
            \frac%
            {%
                v_{y}
            }%
            {%
                \gamma
                \left(
                    1 - \beta \frac{v_{x}}{c}
                \right)
            }\\
            v'_{z} = 
            \frac%
            {%
                v_{z}
            }%
            {%
                \gamma
                \left(
                    1 - \beta \frac{v_{x}}{c}
                \right)
            }
        \end{matrix}
    \right.
\end{equation*}
Observar que todas las componentes de la velocidad se ven modificadas, no solo la componente paralela al movimiento del sistema inercial primado $S'$. Con lo cual, el observador en el sistema primado $S'$ no solo verá una velocidad diferente en la dirección paralela a su desplazamiento, si no que verá a todo el movimiento con un cierto ángulo relativo (que no se vería si no fuera porque $S'$ se está moviendo). Esto explica la aberración estelar.


%%%%%%%%%%%%%%%%%%%%%%%%%%%%%%%%%%%%%%%%%%%
%%%%%%%%%%%%%%%%%%%%%%%%%%%%%%%%%%%%%%%%%%%


\subsubsection{Diagramas de espacio tiempo}
Llamamos espacio de Minkowsky a un espacio con noción de distancia dada por el intervalo $dS$. En un diagrama $ct$-$x$, un dado suceso está representado por un punto. Se define la \textit{linea del universo} como la trayectoria de una partícula en el diagrama $ct$-$x$ (espacio-tiempo de Minkowsky) $(ct, \textbf{x})$. \textbf{Obs:} La pendiente de la linea del universo no puede ser menor a $1$, porque eso significaría que se está moviendo a una velocidad más rápida que la de la luz.\footnote{si $c\Delta t = m\Delta x$, despejo y tengo $c = m \frac{\Delta x}{\Delta t} = m v$ y si $m<1$, necesariamente $v > c$ para satisfacer la igualdad. Which is wrong!}\\
\indent Como la magnitud $\Delta S^{2}$ es un invariante relativista, para cualquier par de sucesos, el valor del intervalo vale lo mismo para todo sistema de referencia inercial. El valor del intervalo es \textit{absoluto}. Gracias a esto, los sucesos pueden clasificarse según el valor del invariante $\Delta S^{2}$, es como decir que hay 3 regiones en el espacio de Minkowsky
\begin{itemize}
    \item $\Delta S^{2} > 0$:
    \item $\Delta S^{2} < 0$:
    \item $\Delta S^{2} = 0$: rayo de luz: $c^{2}t^{2} = x^{2} + y^{2} + z^{2}$. Ambos sucesos se encuentran sobre una recta con pendiente $1$ en el diagrama $ct$-$x$
\end{itemize}
Puede suceder que los eventos que son simultáneos en $S$ (puntos sobre el eje $x$) no lo sean en $S'$ (en $S'$ los eventos simultáneos están sobre el eje $x'$). En particular, $A$ está en el futuro de $S$ y en el pasado de $S'$\footnote{Ver apunte Clase 20 página 8-9}.\\
\indent Los eventos con $\Delta S^{2} = c^{2}\Delta t^{2} - \Delta x^{2} - \Delta y^{2} - \Delta z^{2} > 0$ (que están dentro del cono de luz) se dice que están separados temporalmente, es decir, pueden estar causalmente conectados. Existe algún sistema de referencia que se mueve con la velocidad indicada para que para que la longitud del intervalo sea solo temporal. Consideremos que $\Delta y = \Delta z = 0$ en el sistema que se mueve con $\beta = \Delta x/c\Delta t$
\begin{equation*}
    \begin{matrix}
        \Delta x' = \gamma(\Delta x - \beta c \Delta t) = 0\\
        \Delta t' = \gamma(c\Delta t - \beta \Delta x) \neq 0
    \end{matrix}
\end{equation*}
entonces $\Delta S'^{2} = c^{2}\Delta t'^{2}$, el intervalo es solo temporal. \\
\indent Ahora miro el otro caso, en el que tengo dos eventos que cumplen que $\Delta S^{2} = c^{2}\Delta t^{2} - \Delta x^{2} - \Delta y^{2} - \Delta z^{2} < 0$, es decir, están en \textit{otra parte}, no tienen conexión causal y están solo separados espacialmente. Es decir, existe algún sistema de referencia $S'$ que se mueve según $\beta = c\Delta t/\Delta x$ de forma que $\Delta S'^{2} = -\Delta x^{2}-\Delta y^{2} - \Delta z^{2} < 0$\footnote{Ver apéndice 48}.


%fin clase 20
%%%%%%%%%%%%%%%%%%%%%%%%%%%%%%%%%%%%%%%%%%%
%%%%%%%%%%%%%%%%%%%%%%%%%%%%%%%%%%%%%%%%%%%
%inicio clase 21


\subsection{Formalismo de Cuadrivectores}
Se arma el formalismo de cuadrivectores para introducir una nueva notación conveniente, en el cual las ecuaciones de la física sean covariantes, o dicho de otra forma, que mantengan su forma.
Definimos el cuadrivector posición $x^{\mu}$ como 
\begin{equation*}
    x^{\mu} = (ct, x, y, z) = (x^{0}, x^{1}, x^{2}, x^{3}) = (x^{0},\textbf{x})
\end{equation*}
la longitud o módulo (o longitud del intervalo) del cuadrivector viene dado por
\begin{equation*}
    (x^{0})^{2} - (x^{1})^{2} - (x^{2})^{2} - (x^{3})^{2} = S^{2}
\end{equation*}
en el espacio de Minkowsky. Quiero ver qué transformaciones dejan invariante la longitud del cuadrivector $x^{\mu}$. Seguro que las transformaciones de Lorentz lo dejan invariante, por construcción. Las rotaciones espaciales también dejan invariante el módulo porque sólo actúan sobre las coordenadas del espacio, que pertenecen a un espacio euclídeo. En suma, las transformaciones de Lorentz y las rotaciones rígidas forman el \textbf{grupo de Poincaré} (más reflexiones).\\
\indent Del primer postulado de la relatividad especial, las ecuaciones de la física deben ser invariantes de forma bajo la acción de estas transformaciones, entonces las magnitudes involucradas tienen que poder escribirse como cuadrivectores (o tensores de rango más alto) que se transformen como el cuadrivector posición $x^{\mu}$.\\
\indent Consideremos cuatro cantidades $A^{\mu}$ y las transformo según las transformaciones de Lorentz
\begin{equation*}
    \left\{
        \begin{array}{ll}
             A'^{0} = & \gamma(A^{0} - \beta A^{1}) \\
             A'^{1} = & \gamma(A^{1} - \beta A^{0}) \\
             A'^{2} = & A^{2}\\
             A'^{3} = & A^{3}
        \end{array}
    \right.
\end{equation*}
donde $A^{\mu} = (A^{0}, \textbf{A})$ es un \textbf{cuadrivector contravariante}. Entonces, $4$ magnitudes que las puedo transformar con las transformaciones de Lorentz (se transforman como $x^{\mu}$) es un cuadrivector contravariante. Estas transformaciones se puede escribir en forma resumida usando la convención de la suma como sigue
\begin{equation}
    A'^{\mu} = \frac{\partial x'^{\mu}}{\partial x^{\nu}} A^{\nu}
        \label{ec:TrafoContravariante}
\end{equation}
donde $\frac{\partial x'^{\mu}}{\partial x^{\nu}}$ es el tensor de la transformación de Lorentz\footnote{Ver apéndice 48.1}. Notar que la distancia (la geometría del espacio tiempo) en el espacio de Minkowski está dada por el intervalo 
\begin{equation*}
    dS^{2} = 
    c^{2}dt^{2} - dx^{2} - dy^{2} - dz^{2} 
    = (dx^{0})^{2}-(dx^{1})^{2}-(dx^{2})^{2}-(dx^{3})^{2} 
    = g_{\mu\nu}\,dx^{\mu}dx^{\nu}    
\end{equation*}
donde $g_{\mu\nu}$ es el tensor métrico del espacio de Minkowsku y está dado por
\begin{equation*}
    g_{\mu\nu} = 
    \left(
        \begin{matrix}
            1   & 0   & 0     & 0\\
            0   & -1  & 0     & 0\\
            0   & 0   & -1    & 0\\
            0   & 0   & 0     & -1
        \end{matrix}
    \right)
\end{equation*}
O sea que el tensor métrico contiene la información de la geometría del espacio. Si, por ejemplo, estuviera en un espacio euclídeo $4$-dimensional, el tensor sería lo mismo pero con todos sus signos positivos.\\
\indent Esto define una especie de producto interno y con él, una noción de longitud en esta notación. Como el cuadrivector $A^{\mu}$ se transforma como $x^{\mu}$, si quiero calcular su longitud tengo que usar la nueva forma del producto interno, que sería $g_{\mu\nu}A^{\mu}A^{\nu}$. De esta forma, la longitud del cuadrivector $A^{\mu}$ en este espacio es $A^{2} = (A^{0})^{2}-(A^{1})^{2}-(A^{2})^{2}-(A^{3})^{2}$.\\
\indent Si solo tomo la contracción $g_{\mu\nu}A^{\nu}$ se introduce el \textbf{cuadrivector covariante} $A_{\mu} = (A_{0}, - \textbf{A})$ o $A_{\mu} = g_{\mu\nu}A^{\nu}$. El módulo del cuadrivector se calcula como
\begin{equation*}
    A^{\mu}A_{\mu} 
    = A^{0}A_{0} + A^{1}A_{1} + A^{2}A_{2} + A^{3}A_{3} 
    = (A^{0})^{2} - (A^{1})^{2} - (A^{2})^{2} - (A^{3})^{2}
\end{equation*}
que es un escalar y es invariante relativista de Lorentz.\\
\indent Un vector covariante se transforma como
\begin{equation}
    A'_{\mu} = \frac{\partial x^{\nu}}{\partial x'^{\mu}}A_{\nu}
        \label{ec:TrafoCovariante}
\end{equation}
donde $\frac{\partial x^{\nu}}{\partial x'^{\mu}}$ es la inversa del tensor de transformación de Lorentz.\\
\indent Notar que $A^{\mu} = g^{\mu\nu}A_{\nu} \Longrightarrow g^{\mu\nu} = g_{\mu\nu}$, o sea que el tensor métrico es igual a su inversa (para el espacio de Minkowski al menos), y eso se ve fácil multiplicando las dos matrices, se obtiene la identidad trivialmente. Entonces, si hago $g^{\mu\alpha}g_{\mu\beta} = \delta^{\alpha}_{\beta}$. Ahora que ya tengo definidos los cuadrivectores contravariantes y los covariantes, quiero saber qué es lo que los diferencia.\\
\indent Ahora, usando regla de la cadena quiero escribir el diferencial de una coordenada primada $dx'^{\mu}$, con $\mu = 0,1,2,3$ me queda
\begin{equation*}
    dx'^{\mu} = 
    \frac{\partial x'^{\mu}}{\partial x^{0}}dx^{0} + 
    \frac{\partial x'^{\mu}}{\partial x^{1}}dx^{1} + 
    \frac{\partial x'^{\mu}}{\partial x^{2}}dx^{2} + 
    \frac{\partial x'^{\mu}}{\partial x^{3}}dx^{3}
\end{equation*}
que se puede simplificar usando convención de la suma de Einstein y queda
\begin{equation*}
    dx'^{\mu} = \frac{\partial x'^{\mu}}{\partial x^{\nu}} dx^{\nu}
    \quad
    \quad
    \mbox{o su inversa}
    \quad
    \quad
    dx^{\mu} = \frac{\partial x^{\mu}}{\partial x'^{\nu}} dx'^{\nu}
\end{equation*}
donde resulta que el diferencial de la posición se transforma según $\frac{\partial x'^{\mu}}{\partial x^{\nu}}$, que no es más que la matriz de transformación de Lorentz! Si reemplazo $dx'^{\mu}$ por $A'^{\mu}$ tengo la transformación de un vector contravariante. \textbf{O sea que un cuadrivector contravariante se transforma como un diferencial de las coordenadas}.\\
\indent Por otro lado, el diferencial de una función escalar debe ser un invariante, es decir, cualquier número tiene que seguir valiendo lo mismo en todo sistema de referencia inercial. Nuevamente, si hago regla del la cadena al diferencial $d\varphi$ y lo escribo con la convención de la suma, llego a 
\begin{equation*}
    d\varphi = \frac{\partial \varphi}{\partial x^{\mu}}dx^{\mu}
\end{equation*}
donde como el $dx^{\mu}$ es un cuadrivector contravariante, para que $d\varphi$ sea invariante necesito que $\frac{\partial \varphi}{\partial x^{\mu}}$ sea un cuadrivector covariante. Frente a un cambio de coordenadas, sabemos que
\begin{equation*}
    \frac{\partial \varphi}{\partial x'^{\mu}} 
    = \frac{\partial \varphi}{\partial x^{\nu}}
    \frac{\partial x^{\nu}}{\partial x'^{\mu}}
    \quad
    \quad
    \mbox{o su inversa}
    \quad
    \quad
    \frac{\partial \varphi}{\partial x^{\mu}} 
    = \frac{\partial \varphi}{\partial x'^{\nu}}
    \frac{\partial x'^{\nu}}{\partial x^{\mu}}
\end{equation*}
y si reemplazo donde dice $\frac{\partial \varphi}{\partial x'^{\mu}} = A'_{\mu}$ y $\frac{\partial \varphi}{\partial x^{\nu}} = A_{\nu}$ recupero la expresión \eqref{ec:TrafoCovariante}, con lo cual deducimos que \textbf{un vector covariante se transforma como las derivadas de un escalar}. En geometrías no euclidianas, tenemos que distinguir entre los dos tipos de vectores.\\
\indent Se pueden generalizar estos conceptos a tensores, de la siguiente manera: dada una entidad de 16 cantidades ordenadas en una matriz de $4 \times 4$ que se transforman según
\begin{equation*}
    T'^{\mu\nu} = 
    \frac{\partial x'^{\mu}}{\partial x^{\alpha}}
    \frac{\partial x'^{\nu}}{\partial x^{\beta}}T^{\alpha\beta}
\end{equation*}
es un \textbf{tensor contravariante de rango dos}. Además vale que
\begin{itemize}
    \item $T^{\mu}_{~\nu} = g_{\alpha\nu}T^{\mu\alpha}$ - bajar un índice: Sin contraigo un sólo índice de un tensor de rango dos, el índice baja.
    \item $R^{\beta\gamma}_{~~\delta} = T^{\alpha\beta\gamma}W_{\alpha\delta}$ - contraer un índice: idem
\end{itemize}
Si se contraen todos los índices de un tensor, se obtiene un escalar, y los escalares son todos invariantes de Lorentz.


%%%%%%%%%%%%%%%%%%%%%%%%%%%%%%%%%%%%%%%%%%%
%%%%%%%%%%%%%%%%%%%%%%%%%%%%%%%%%%%%%%%%%%%


\subsubsection{Operador derivada Parcial}
Se define el cuadrivector covariante derivada parcial
\begin{equation*}
    \partial_{\mu} = 
    \left(
        \frac{\partial}{\partial x^{0}}, \nabla
    \right)
\end{equation*}
y no lleva el menos porque se transforma al revés que el cuadrivector posición, pero podemos construir el vector contravariante como
\begin{equation*}
    \partial^{\mu} = 
    \left(
        \frac{\partial}{\partial x^{0}}, -\nabla
    \right)
\end{equation*}
Finalmente, dado un vector contravariante $A^{\mu}$, resulta que $\partial_{\mu}A^{\mu}$ es invariante. Además, el operador $\partial_{\mu}\partial^{\mu}$ también debe ser un invariante y se llama D'Alambartiano
\begin{equation*}
    \square = \partial_{\mu}\partial^{\mu} =
    \frac{\partial^{2}}{\partial x^{02}} -
    \frac{\partial^{2}}{\partial x^{12}} -
    \frac{\partial^{2}}{\partial x^{22}} -
    \frac{\partial^{2}}{\partial x^{32}} =
    \frac{1}{c^{2}}\frac{\partial^{2}}{\partial t^{2}} - \nabla^{2}
\end{equation*}
y llego al esperado de la ecuación de ondas de las ecuaciones de Maxwell en el gauge de Lorenz (\eqref{ec:MaxwellEcsGaugeLorenz}), que resulta ser un invariante relativista dado que $\partial_{\mu}\partial^{\mu}$ lo es.

%%%%%%%%%%%%%%%%%%%%%%%%%%%%%%%%%%%%%%%%%%%
%%%%%%%%%%%%%%%%%%%%%%%%%%%%%%%%%%%%%%%%%%%


\subsection{Ecuaciones de Maxwell} 
Por el primer postulado de la relatividad especial, las leyes de la física (y en particular, las ecuaciones de Maxwell) deben poder expresarse de igual forma en todo sistema de referencia inercial, es decir, deben tener formulación covariante.\\
\indent Ahora quiero escribir las ecuaciones de Maxwell con esta nueva notación. Dada la definición del $\square = \partial_{\mu}\partial^{\mu}$, ya sabemos que la ecuación de ondas (en el gauge de Lorenz) es covariante.\\
\indent De la ecuación de continuidad $\frac{\partial \rho}{\partial t} + \nabla \cdot \textbf{j} = 0$ se sigue que es covariante si la densidad $\rho$ y la corriente $\textbf{j}$ forman un cuadrivector de la forma
\begin{equation*}
    J^{\mu} = (c\rho, \textbf{j})
\end{equation*}
que se llama cuadrivector corriente. La ecuación de continuidad escrita en este formalismo queda expresada como
\begin{equation*}
    \partial_{\mu}J^{\mu} = 0
\end{equation*}
notar que dada la definición de la derivada $\partial_{\mu} = (\partial_{t}/c, \nabla)$, para que la contracción $\partial_{\mu}J^{\mu}$ efectivamente sea un invariante y efectivamente me escriba la ecuación de conservación de la carga, $J^{\mu}$ debe ser contravariante (Como fue convenientemente escrito con el supraíndice $\mu$) con lo cual, las componentes de $J^{\mu}$ se transforman como $x^{\mu}$, es decir
\begin{equation*}
    c\rho' = \gamma(c\rho - \beta j_{x})
\end{equation*}
\begin{equation*}
    j'_{x} = \gamma(j_{x} - \beta c \rho)
\end{equation*}
donde elegí $j'_{y} = j_{y}$ y $j'_{z} = j_{z}$.\\
\indent Ahora que ya escribí una ecuación en esta notación para las fuentes, quiero buscar una análoga para los potenciales. Debe existir cuadrivector que condense la información tanto del potencial electrostático como del potencial vector. Recordemos que en el gauge de Lorenz, $\nabla \cdot \textbf{A} + \frac{1}{c}\frac{\partial \varphi}{\partial t} = 0$ y las ecuaciones de Maxwell con fuentes se escriben como 
\begin{equation*}
    \left\{
        \begin{matrix}
            \nabla^{2}\textbf{A} - \frac{1}{c^{2}}\frac{\partial^{2}\textbf{A}}{\partial t^{2}}
            = -\frac{4\pi}{c}\textbf{j}\\
            \\
            \nabla^{2}\varphi - \frac{1}{c^{2}}\frac{\partial^{2}\varphi}{\partial t^{2}}
            = -\frac{4\pi}{c}\rho
        \end{matrix}
    \right.
\end{equation*}
entonces voy a hacer que $\varphi$ y $\textbf{A}$ sean las componentes de un cuadrivector. Se define el \textbf{cuadrivector potencial del campo electromagnético} como
\begin{equation*}
    A^{\mu} = (\varphi, \textbf{A})
\end{equation*}
y con esta definición, el gauge de Lorenz en esta notación se escribe de la siguiente manera
\begin{equation*}
    \partial_{\mu}A^{\mu} = 0
\end{equation*}\footnote{
tiene forma de ley de conservación. Expresa que los estados del fotón no tienen proyección de espín $0$ (masa del fotón nula). De hecho, elimina el espín $0$ en la representación del gauge de Lorenz pues $k_{\mu}A^{\mu} = 0$}
y por último, para igualarlo a las ecuaciones de ondas de Maxwell con fuentes, uso la cuadricorriente y las ecuaciones de onda inhomogéneas quedan 
\begin{equation*}
    \square A^{\mu} 
    = \partial_{\nu}\partial^{\nu}A^{\mu}
    = \frac{4\pi}{c}J^{\mu}
\end{equation*}
son las ecuaciones de Maxwell con fuentes en el gauge de Lorenz en esta nueva notación.\\
\indent Ya tengo las fuentes y los potenciales escritos en esta nueva notación y ya sé como se transforman. Ahora veamos como se transforman \textbf{E} y \textbf{B}. Tenemos las ecuaciones para los campos (en el gauge de Lorenz) de la siguiente manera
\begin{equation*}
    \left\{
        \begin{matrix}
            \textbf{E} = \nabla \varphi - \frac{1}{c}\frac{\partial \textbf{A}}{\partial t}\\
            \\
            \textbf{B} = \nabla \times \textbf{A}
        \end{matrix}
    \right.
\end{equation*}
veamos si estos campos son las componentes de un tensor de rango dos. Definamos el tensor \textbf{intensidad de campo} como 
\begin{equation*}
    F^{\mu\nu} = \partial^{\mu}A^{\nu} - \partial^{\nu}A^{\mu}
\end{equation*}
que es antisimétrico, entonces los elementos diagonales son nulos y como es de $4 \times 4$, tiene $6$ componentes independientes.\footnote{Ver apéndice 50} El tensor termina siendo
\begin{equation*}
    F^{\mu\nu} 
    = 
    \left(
        \begin{matrix}
            0       & - E_{x} & - E_{y} & - E_{z}\\
            & & & \\
            E_{x}   & 0       & - B_{z} &  B_{y}\\
            & & & \\
            E_{y}   & B_{z}   & 0       & -B_{x}\\
            & & & \\
            E_{z}   & -B_{y}  & B_{x}   &   0
        \end{matrix}
    \right)
\end{equation*}
Observación: El cuadritensor covariante se obtiene haciendo $F_{\mu\nu} = g_{\mu\alpha}F^{\alpha\beta}g_{\beta\nu}$, que matricialmente es aplicar a cada lado de la representación matricial de $F^{\mu\nu}$, la representación matricial del tensor métrico $g_{\alpha\beta}$\footnote{Ver Jackson 2ed - página 550 del pdf}.\\
\indent Con lo cual se logró escribir a todas las componentes de los campos eléctrico y magnético dentro de un tensor que sabemos cómo se transforma. Se define el pseudotensor $\varepsilon^{\alpha\beta\gamma\delta}$ análogo al pseudotensor de Levi-civita: Vale $1$ para permutaciones pares, $-1$ para permutaciones impares y $0$ con índices repetidos.\\
\indent Dado un tensor antisimétrico $A^{\mu\nu}$, se define el dual (pseudotensor) como
\begin{equation*}
    \mathcal{A}^{\alpha\beta} = 
    \frac{1}{2}\varepsilon^{\alpha\beta\gamma\delta}A_{\mu\nu}
\end{equation*}
notar que $\mathcal{A}^{\mu\nu}A_{\mu\nu}$ es un pseudoescalar (recordar que el dual $\mathcal{V}$ de un espacio $V$ está formado por todas las funciones lineales en $V$). Luego, el dual del tensor intensidad de campo $F^{\mu\nu}$ es
\begin{equation*}
    \mathcal{F}^{\alpha\beta} = \frac{1}{2} \varepsilon^{\alpha\beta\mu\nu}
    F_{\mu\nu} 
    =
    \left(
        \begin{matrix}
            0       & -B_{x} & -B_{y} & -B_{z}\\
            & & & \\
            B_{x}   & 0      &  E_{z} &  -E_{y}\\
            & & & \\
            B_{y}   & -E_{z} & 0      &  E_{x}\\
            & & & \\
            B_{z}   &  E_{y} & -E_{x} &   0
        \end{matrix}
    \right)
\end{equation*}
notar que lo que hizo esta contracción fue intercambiar las componentes del campo magnético por las del campo eléctrico. Además notar que en $F^{\mu\nu}$, los signos de todas las componentes del campo eléctrico, de un lado y del otro de la diagonal son los mismos, mientras que para el campo magnético están mezclados. Esto se debe a que el campo eléctrico es un vector y el campo magnético es un pseudo vector. Al aplicar un pseudotensor al cuadritensor intensidad de campo, la parte vectorial se vuelve pseudo vectorial y la parte pseudovectorial se vuelve vectorial, haciendo que en el dual de $F^{\mu\nu}$ ahora las componentes del campo magnético tengan el mismo signo de un lado y del otro de la diagonal, mientras que para el campo eléctrico ahora están mezcladas.\\
\indent Veamos ahora como se escriben las ecuaciones de Maxwel generales en forma covariante
\begin{equation*}
    \partial_{\mu}F^{\mu\nu} = 
    \partial_{\mu}
    \left(
        \partial^{\mu}A^{\nu} - \partial^{\nu}A^{\mu}
    \right)
    = \partial_{\mu}\partial^{\mu}A^{\nu} 
    - \partial_{\nu}(\partial_{\mu}A^{\mu})
    = \partial_{\mu}F^{\mu\nu} = \frac{4\pi}{c}J^{\nu}
\end{equation*}
que son cuatro ecuaciones para $\nu = 0,1,2,3$\footnote{Ver apéndice 50} resulta
\begin{itemize}
    \item $\nu = 0$): equivale a aplicar el operador $\partial_{\mu}$ y sumar sobre la primera columna del tensor $F^{\mu\nu}$ y se obtiene la Ley de Gauss
    \begin{equation*}
        \nabla \cdot \textbf{E} = 4\pi \rho
    \end{equation*}
    
    \item $\nu = 1,2,3$): Equivale a hacer sumas individuales sobre las columnas $1$, $2$ y $3$ del tensor $F^{\mu\nu}$ y de cada una de estas sumas se obtienen las componentes $x$, $y$ y $z$ (respectivamente) de la Ley de Ampere (con el término de desplazamiento en el vacío)
    \begin{equation*}
        \nu = 1:
        \quad
        \quad
        - \frac{1}{c}\frac{\partial E_{x}}{\partial t} 
        + 
        \underbrace%
        {%
            \left(
                \frac{\partial B_{z}}{\partial y} 
                - \frac{\partial B_{y}}{\partial z}
            \right)
        }_{\left(
            \nabla \times \textbf{B}
        \right)_{x} }
        = \frac{4\pi}{c}j_{x}
    \end{equation*}
    \begin{equation*}
        \nu = 2:
        \quad
        \quad
        - \frac{1}{c}\frac{\partial E_{y}}{\partial t} 
        + 
        \underbrace%
        {%
            \left(
                \frac{\partial B_{x}}{\partial z} 
                - \frac{\partial B_{z}}{\partial x}
            \right)
        }_{\left(
            \nabla \times \textbf{B}
        \right)_{y}}
        = \frac{4\pi}{c}j_{y}
    \end{equation*}
    \begin{equation*}
        \nu = 3:
        \quad
        \quad
        - \frac{1}{c}\frac{\partial E_{z}}{\partial t} 
        + 
        \underbrace%
        {%
            \left(
                \frac{\partial B_{y}}{\partial x} 
                - \frac{\partial B_{x}}{\partial y}
            \right)
        }_{\left(
            \nabla \times \textbf{B}
        \right)_{z}}
        = \frac{4\pi}{c}j_{z}
    \end{equation*}
    que arregladas en un vector se escribe como
    \begin{equation*}
        \nabla \times \textbf{B} =
        \frac{4\pi}{c}\textbf{j} + 
        \frac{1}{c}\frac{\partial \textbf{E}}{\partial t}
    \end{equation*}
\end{itemize}
Son las ecuaciones de Maxwell inhomogéneas! Nos falta las ecuaciones homogéneas. Tomamos entonces el caso $\partial_{\mu}\mathcal{F}^{\mu\nu} = 0$
\begin{itemize}
    \item $\nu = 0$) Equivale a aplicar el operador $\partial_{\mu}$ y sumar sobre la primera columna del tensor $\mathcal{F}^{\mu\nu}$. Se termina llegando a la inexistencia de los monopolos magnéticos
    \begin{equation*}
        \nabla \cdot \textbf{B} = 0
    \end{equation*}
    
    \item $\nu = 1,2,3$) Equivale a hacer sumas individuales aplicando el operador para los columnas $1$, $2$ y $3$ del tensor dual. De cada una de estas se va a obtener las componentes $x$, $y$ y $z$ respectivamente de la Ley de Faraday (en el vacío)
    \begin{equation*}
        \nu = 1:
        \quad
        \quad
        \left(
            \nabla \times \textbf{E}
        \right)_{x} 
        + \frac{1}{c}\frac{\partial B_{x}}{\partial t}
        = 0
    \end{equation*}
    \begin{equation*}
        \nu = 2:
        \quad
        \quad
        \left(
            \nabla \times \textbf{E}
        \right)_{y} 
        + \frac{1}{c}\frac{\partial B_{y}}{\partial t}
        = 0
    \end{equation*}
    \begin{equation*}
        \nu = 3:
        \quad
        \quad
        \left(
            \nabla \times \textbf{E}
        \right)_{z} 
        + \frac{1}{c}\frac{\partial B_{z}}{\partial t}
        = 0
    \end{equation*}
    y se recupera
    \begin{equation*}
        \nabla \times \textbf{E} = 
        -\frac{1}{c}
        \frac{\partial \textbf{B}}{\partial t}
    \end{equation*}
    usando la definición de tensor dual, esta expresión puede escribirse en términos de $F^{\mu\nu}$ como
    \begin{equation*}
        \partial^{\alpha}F^{\beta\gamma}
        +\gamma^{\gamma}F^{\alpha\beta}
        +\partial^{\beta}F^{\gamma\alpha}
        = 0
    \end{equation*}
\end{itemize}


%%%%%%%%%%%%%%%%%%%%%%%%%%%%%%%%%%%%%%%%%%%
%%%%%%%%%%%%%%%%%%%%%%%%%%%%%%%%%%%%%%%%%%%


\subsection{Transformación del campo electromagnético}
Veamos primero que ciertas cantidades no cambian (son invariantes relativistas). Las podemos hallar contrayendo índices. Tomemos
\begin{itemize}
    \item $F^{\mu}_{~\mu} = 0$, esto significa que la traza es nula para todos sistema de referencia inercial (es invariante relativista)
    \item $F^{\mu\nu}F_{\mu\nu} = -2E^{2} + 2B^{2} = -2(E^{2}-B^{2})$, esto significa que si la resta del cuadrado de los campos es positiva en $S$ será positiva en todo sistema de referencia inercial $S'$. \textbf{pregunta de final:} Si estoy en una condición en que $\textbf{E}\cdot \textbf{E} = 0$, cuál es el campo que puede anularse según las restricción que impone esta segunda condición?
    \item $F_{\mu\nu}\mathcal{F}^{\mu\nu} = -4\textbf{E}\cdot \textbf{B}$, esto significa que si el campo eléctrico es perpendicular en un dado sistema de referencia $S$, será perpendicular para todos sistema de referencia inercial $S'$. 
\end{itemize}
Veamos ahora las transformaciones de $F^{\mu\nu}$. Este debe transformarse como
\begin{equation*}
    F'^{\mu\nu} = L^{\mu}_{~\alpha}L^{\nu}_{~\beta}F^{\alpha \beta}
\end{equation*}
donde 
\begin{equation*}
    L^{\mu}_{~\alpha}
    =
    \left(
        \begin{matrix}
            \gamma          & -\beta \gamma & 0 & 0\\
            -\beta \gamma   & \gamma        & 0 & 0\\
            0               & 0             & 1 & 0\\
            0               & 0             & 0 & 1\\
        \end{matrix}
    \right)
\end{equation*}
entonces se obtiene
\begin{equation*}
    \left\{
        \begin{array}{ll}
            \textbf{E}'_{\parallel} 
            = &\textbf{E}_{\parallel}\\
            &\\
            \textbf{E}'_{\perp} 
            = &\gamma(\textbf{E}_{\perp} 
            + \bar{\beta} \times \textbf{B})\\
            &\\
            \textbf{B}'_{\parallel} 
            = &\textbf{B}_{\parallel}\\
            \textbf{B}'_{\perp} 
            &\\
            = &\gamma(\textbf{B}_{\perp} 
            - \bar{\beta}\times\textbf{E})
        \end{array}
    \right.
\end{equation*}
donde paralelo y perpendicular es respecto al boost, o la dirección de desplazamiento del sistema de referencia $S'$. Notar que tanto el campo eléctrico paralelo como el campo magnético paralelo no cambian de un sistema de referencia a otro (al contrario del efecto de la contracción espacial en la mecánica), mientras que las componentes perpendiculares sí. Más aún, para el caso particular en el que el sistema $S$ no existe campo magnético, $\textbf{B} = \textbf{B}_{\parallel} + \textbf{B}_{\perp} = 0$, en el sistema $S'$ en movimiento, el observador verá un campo magnético $\textbf{B}'_{\perp}$, siempre que el campo eléctrico $\textbf{E}$ en $S$ no sea también nulo (en cuyo caso ya no habría nada). Entonces, si en un sistema de referencia alguno de los dos campos es nulo, en otro sistema de referencia podría no serlo.


%fin clase 21
%%%%%%%%%%%%%%%%%%%%%%%%%%%%%%%%%%%%%%%%%%%
%%%%%%%%%%%%%%%%%%%%%%%%%%%%%%%%%%%%%%%%%%%
%inicio clase 22 (no la vi)


\subsection{Efecto Doppler relativista}
Consideremos una onda plana en el sistema de referencia $S$, que se propaga con vector de onda $\textbf{k}$ con un ángulo $\theta$ relativo al eje $x$. Escribo la ecuación del campo y la onda plana de forma clásica y en notación covariante como
\begin{equation*}
    \textbf{E}
    = \textbf{E}_{0}\,e^{i(\omega t - \textbf{k}\cdot\textbf{x})} 
    = \textbf{E}_{0}\,e^{ik_{\mu}x^{\mu}}
\end{equation*}
con $k_{\mu} = \left( \frac{\omega}{c}, -\textbf{k}\right)$ y $k^{\mu} = \left( \frac{\omega}{c}, \textbf{k}\right)$ el cuadrivector covariante y contravariante (respectivamente) de onda. Ahora considero la onda vista desde el sistema $S'$, que se escribe con las magnitudes primadas
\begin{equation*}
    \textbf{E}' = \textbf{E}'_{0}\,e^{ik'_{\mu}x'^{\mu}}
\end{equation*}
donde hay que observar que la amplitud se transforma según cómo se transforman los campos y la fase $k'_{\mu}x'^{\mu}$ es una contracción, con lo cual es un invariante relativista. La fase es la misma desde todo sistema de referencia inercial: $k'_{\mu}x'^{\mu} = k_{\mu}x^{\mu}$. Sin embargo, las componentes del cuadrivector de onda sí se transforman, por ejemplo, para un boost en $\hat{x}$ se tiene
\begin{equation*}
    \left\{
        \begin{array}{ll}
            \frac{\omega'}{c} 
            = &\gamma
            \left(
                \frac{\omega}{c} - \beta k_{x}
            \right)\\
            k'_{x} = &\gamma
            \left(
                k_{x} - \beta\frac{\omega}{c}
            \right)\\
            k'_{y} = & k_{y}\\
            k'_{z} = & k_{z}
        \end{array}
    \right.
\end{equation*}
con esto se puede determinar completamente la frecuencia $\omega'$ en el sistema de referencia $S'$ usando que $|k'| = \frac{\omega'}{c}$ y que $k_{x} = |\textbf{k}|\cos{(\theta)}$, reemplazando en la primera ecuación de la transformación y despejando se llega a la frecuencia transformada
\begin{equation*}
    \omega' = \gamma \omega (1 - \beta \cos{(\theta)})
\end{equation*}
con $\beta = v/c$ y $\gamma = 1/\sqrt{1-\beta^{2}}$. Entonces, lo primero que se ve es que la frecuencia $\omega'$ depende del ángulo $\theta$ medido en el sistema de referencia $S$ y además, como $\gamma$ es función de $\beta$ y $\beta$ es función de la velocidad se ve que la frecuencia observada en el sistema $S'$ depende de la velocidad de desplazamiento del mismo. Veamos algunos casos particulares referentes al ángulo $\theta$, que es el ángulo relativo entre el vector de onda y el desplazamiento del sistema $S'$
\begin{itemize}
    \item $\theta = 0$) El observador se aleja de la fuente: En ese caso la frecuencia transformada queda expresada como
    \begin{equation*}
        \omega' = \omega 
        \left(
            \frac{1 - \beta}{\sqrt{1- \beta^{2}}}
        \right)
        = \omega
        \left(
            \frac{1 - \beta}{1 + \beta}
        \right)^{1/2}< \omega
    \end{equation*}
    o sea que la frecuencia $\omega'$ que ve el observador en el sistema $S'$ es menor que la frecuencia que ve un observador en el sistema $S$.
    \item $\theta = \pi$) El observador se acerca a la fuente. En este caso la frecuencia transformada queda como
    \begin{equation*}
        \omega' = \omega
        \left(
            \frac{1 + \beta}{1 - \beta}
        \right)^{1/2} > \omega
    \end{equation*}
    con lo cual, si el observador en $S'$ se acerca a la fuente, la frecuencia que ve es es mayor a la frecuencia que vería un observador parado en el sistema $S$. Tiene todo el sentido del mundo.
    \item $\theta = \pi/2$) El observador se mueve perpendicular a la fuente. La frecuencia queda
    \begin{equation*}
        \omega ' = 
        \gamma \omega = \frac{\omega}{\sqrt{1 - \frac{v^{2}}{c^{2}}}}
        \quad
        \left\{
            \begin{matrix}
                > \omega\ \mbox{Si}\ v\to c\\
                \\
                = \omega\ \mbox{Si}\ v = 0
            \end{matrix}
        \right.
        \end{equation*}
    a medida que la velocidad $v$ se acerca a $c$, la frecuencia $\omega'$ se vuelve cada vez más grande. Algo que no tiene sentido que pase clásicamente, la frecuencia debería mantenerse constate (considerando una onda plana, si me muevo transversalmente a la fuente, me muevo paralelamente a los planos de fase constante), sin embargo, en el caso relativista la frecuencia aumenta con la velocidad del sistema $S'$.
\end{itemize}
Además hay un cambio en la dirección de propagación, que se transforma\footnote{Ver apéndice 51}
\begin{equation*}
    \cos{(\theta')} = \frac{\cos{(\theta)}-\beta}{1 - \beta\cos{(\theta)}}
\end{equation*}
y este resultado explica la aberración estelar.



%%%%%%%%%%%%%%%%%%%%%%%%%%%%%%%%%%%%%%%%%%%
%%%%%%%%%%%%%%%%%%%%%%%%%%%%%%%%%%%%%%%%%%%


\subsection{Mecánica relativista}

Las ecuaciones de Newton no son covariantes. Para tener una mecánica relativista debemos definir velocidades, aceleraciones, impulsos, etc., que se transformen como cuadrivectores. Lo que queremos entonces es reescribir la ecuación de Newton en notación de cuadrivectores, y luego ver si se pueden recuperar los casos clásicos. Notar que las componentes de la velocidad $\frac{\Delta x}{\Delta t}$ no se transforman según Lorentz. Esto es así porque el tiempo $t$ cambia frente a transformaciones de Lorentz.\\
\indent Se define el cuadrivector velocidad como 
\begin{equation*}
    u^{\mu} = \frac{dx^{\mu}}{d\tau}
\end{equation*}
donde $x^{\mu}$ es el cuadrivector posición y $\tau$ es el tiempo propio, que es un invariante relativista, vale lo mismo en todos sistema de referencia. Es el tiempo del observador, medido desde su sistema de referencia. La necesidad de usar el tiempo propio viene de que si usara el tiempo $t$ en la definición de velocidad, tendría que transformar el tiempo. En cambio usando el tiempo propio, que es invariante, esto no es necesario y el cuadrivector velocidad queda explícitamente escrito como un cuadrivector contravariante, transforma como transforman las coordenadas.\\
\indent Entonces, el cuadrivector velocidad debe ser contravariante y usando regla de la cadena y la expresión para el tiempo propio \eqref{ec:TiempoPropio}, se puede escribir como\footnote{Ver apéndice 50.7}
\begin{equation*}
    u^{\mu} = \gamma_{\textbf{v}}(c, \textbf{v})
\end{equation*}
$\textbf{v}$ velocidad de la partícula respecto de $S$ (La velocidad del sistema $S'$). Se transforma según las transformaciones de Lorentz. La velocidad $\textbf{v}$ se transforma según la transformación de la velocidad. Notar que
\begin{equation*}
    u_{\mu}u^{\mu} = 
    \gamma^{2}(c,-\textbf{v})(c, \textbf{v}) =
    \frac{c^{2}-v^{2}}{1 - \frac{v^{2}}{c^{2}}} =
    \frac{c^{2}-v^{2}}{\frac{1}{c^{2}}(c^{2}-v^{2})} = 
    c^{2}
\end{equation*}
que es invariante, pero no es cualquier invariante, es la velocidad de la luz al cuadrado. Con lo cual $\frac{u^{\mu}}{c}$ es un vector unitario tangente a la linea del universo de la partícula.\\
\indent Volviendo a derivar el cuadrivector velocidad respecto a $\tau$, se obtiene el cuadrivector aceleración
\begin{equation*}
    a^{\mu} 
    = \frac{du^{\mu}}{d\tau} 
    = \frac{d^{2}x^{\mu}}{d\tau^{2}}
    = \gamma_{v}\frac{d\gamma_{v}}{dt}(c, \textbf{v}) 
    + \gamma_{v}^{2} \frac{d}{dt}(c, \textbf{v})
\end{equation*}
y es fácil demostrar cuánto vale la contracción $u^{\mu}a_{\mu}$ invariante derivando $u^{\mu}u_{\mu} = c^{2}$:
\begin{equation*}
    \frac{d}{d\tau} (u_{\mu}u^{\mu}) 
    = 2 u_{\mu}\frac{du^{\mu}}{d\tau} 
    = 2 u_{\mu}a^{\mu} = 0
\end{equation*}
con lo cual, los cuadrivectores velocidad y aceleración son ortogonales (¿y qué implica esto físicamente?).\\
\indent Ahora queremos hallar la generalización de la ecuación de Newton
\begin{equation*}
    \frac{d}{dt}(m\textbf{v}) = \textbf{F}
\end{equation*}
tal que para el límite en el que $c \to \infty$ (o $v << 1$) recuperamos la ecuación de Newton. Tomemos entonces la siguiente propuesta para la ecuación de Newton en esta notación y veamos a qué resultados nos lleva
\begin{equation*}
    \frac{d}{d\tau} (m u^{\mu}) = K^{\mu}
\end{equation*}
donde $m$ es la masa en reposo (invariante relativista), $K$ es la fuerza de Minkowski y debe satisfacer que $K^{i} \to F^{i}$ cuando $c \to \infty$. Esta expresión es por construcción covariante, que es lo que queríamos. Y para el caso de velocidades clásicas, el tiempo propio $\tau \to t$ y el cuadrivector velocidad $u^{\mu}\to \textbf{v}$. Habría que ver si la fuerza de Minkowski $K^{\mu}$ tiende a una fuerza clásica cuando $c\to \infty$.\\
\indent Defino el cuadrivector impulso y se pueden escribir las ecuaciones de Newton de la siguiente manera
\begin{equation*}
    p^{\mu} = m u^{\mu}
    \Longrightarrow \frac{dp^{\mu}}{d\tau} = K^{\mu}
\end{equation*}
todavía no sabemos como es $K^{\mu}$, no sabemos qué hay adentro o como entra la fuerza clásica dentro del cuadrivector. Sin embargo, sí sabemos que las propiedades de transformación de $K^{\mu}$ debe ser las mismas independientemente de la naturaleza de la fuerza. Entonces podemos considerar el caso particular de fuerzas electromagnéticas para ver qué pasa. Tomo la fuerza de Lorentz
\begin{equation*}
    \textbf{F} =
    q \textbf{E} + \frac{q}{c} \textbf{v}\times \textbf{B}
\end{equation*}
entonces, escribimos el cuadrivector de fuerzas como
\begin{equation*}
    K^{\mu} = \frac{q}{c} F^{\mu\nu}u_{\nu}
\end{equation*}
para convencerse de que esta forma de escribir el cuadrivector $K^{\mu}$ tiene sentido, basta con mirar alguna componente, digamos $x$ y ver que efectivamente se llega a la componente $x$ de la fuerza de Lorentz\footnote{Ver apéndice 51.7}. En efecto, se llega a 
\begin{equation*}
    K^{i} = \gamma_{\textbf{v}}F^{i}
\end{equation*}
con $i = 1,2,3$, las componentes espaciales de la fuerza. Entonces, ya sé como se escribe la parte espacial del cuadrivector fuerza de Minkowski, es $\gamma_{\textbf{v}}\textbf{F}$. Falta ver cómo se escribe la parte temporal (coordenada $x^{0}$ del cuadrivector).\\
\indent Pero primero veamos las componentes del cuadrivector impulso
\begin{equation*}
    p^{\mu} = (p^{0}, \textbf{p})
    \quad
    \quad
    \mbox{con}
    \quad
    \quad
    \left\{
        \begin{matrix}
            \textbf{p} = \gamma_{v} m \textbf{v}\\
            \\
            p^{0} = \gamma_{v} m c
        \end{matrix}
    \right.
\end{equation*}
de acá se observa que el impulso $\textbf{p}$ depende de $\gamma_{\textbf{v}}$ y por ende de la velocidad. Esto implica físicamente que mientras mayor velocidad tenga la partícula, mayor inercia tendrá. Y vemos que de derivar el impulso respecto al tiempo propio $\tau$, es decir $\frac{d\textbf{p}}{d\tau} = ... = \gamma_{\textbf{v}}\frac{d\textbf{p}}{dt} = \gamma_{\textbf{v}}\textbf{F}$, con lo cual, efectivamente se obtiene lo que se sabía de las ecuaciones de Newton. Es decir, la derivada temporal de la componente espacial del cuadrivector impulso, devuelve las componentes espaciales de la fuerza.
\begin{equation*}
    \frac{d\textbf{p}}{dt} = \textbf{F}
    \quad
    \quad
    \mbox{con}\
    \textbf{p} = \gamma_{v}m \textbf{v}
\end{equation*}
y en el límite de $c \to \infty$ recobramos Newton. Pero la componente temporal todavía no sabemos qué forma tiene. Para ver qué pasa con la componente temporal, tomo $\frac{dp^{0}}{d\tau} = K^{0}$ y $p^{0} = \gamma_{v} m c$, pero ¿Cuánto vale $K^{0}$ en general? si tomo el siguiente invariante $u_{\mu}K^{\mu}$, escribiendo la fuerza Minkowski como la derivada respecto de $\tau$ del cuadrivector impulso y distribuyendo la derivada se llega a que 
\begin{equation*}
    u_{\mu}K^{\mu} = 
    \frac{d}{d\tau}
    \left(
        \frac{1}{2}mc^{2}
    \right) = 0
\end{equation*}
y expandiendo la contracción de forma explícita se puede despejar $K^{0}$
\begin{equation*}
    K^{0} = \frac{\gamma_{\textbf{v}}}{c}\textbf{v}\cdot \textbf{F}
\end{equation*}
entonces, finalmente la expresión para el cuadrivector Fuerza de Minkowski se escribe como
\begin{equation*}
    K^{\mu} = \gamma_{v}
    \left(
        \frac{\textbf{v}\cdot \textbf{F}}{c}, \textbf{F}
    \right)
\end{equation*}
Observando la componente temporal $K^{0}$ y viendo que por definición vale lo mismo que la derivada respecto a $\tau$ de la componente temporal del cuadrivector impulso, es decir
\begin{equation*}
    \frac{dp^{0}}{dt}
    = \frac{\textbf{v}\cdot \textbf{F}}{c}
\end{equation*}
y como $p^{0} = \gamma_{\textbf{v}}mc$, resulta que puedo escribir $\frac{d}{dt}(\gamma_{\textbf{v}}mc^{2}) = \textbf{v}\cdot\textbf{F}$ y quiero determinar qué es lo que está entre paréntesis. Entonces tengo la derivada temporal de una magnitud a determinar que es igual a la velocidad por la fuerza, que también puede escribirse en función de la derivada de la posición respecto al tiempo, de forma que queda la fuerza por un diferencial de distancia se resume en la potencia, o sea
\begin{equation*}
    \frac{d}{dt}(\gamma_{\textbf{v}}m c^{2}) 
    = \textbf{v}\cdot \textbf{F} 
    = \frac{d\textbf{x}}{dt}\cdot \textbf{F}
    = \mbox{potencia}
\end{equation*}
y como tengo que la derivada temporal de una magnitud es la potencia, entonces esa magnitud tiene que ser la energía (y tiene unidades de energía)
\begin{equation*}
    E = \gamma_{v}mc^{2} = \frac{mc^{2}}{\sqrt{1 - \frac{v^{2}}{c^{2}}}}
\end{equation*}
y, finalmente el cuadrivector impulso se puede escribir con la energía dividido la velocidad de la luz en su parte temporal y el impulso mecánico en la parte espacial
\begin{equation*}
    p^{\mu} = 
    \left(
        \frac{E}{c}, \textbf{p} 
    \right)
\end{equation*}
notar que se puede escribir $\textbf{p} = \gamma_{v} m \textbf{v} = \frac{E}{c^{2}}\textbf{v}$. O sea que ahora la inercia está dada por $E/c^{2}$, depende de la energía de la partícula.\\
\indent De la definición de $p^{\mu}$ se sigue que en la relatividad especial, siempre que se conserva el momento espacial, se conserva la energía $E$. La energía se conserva aún en choques inelásticos! (pero se viola la conservación de la masa. La masa total no es la suma de las masas de cada partícula en el choque). Veamos el invariante longitud de $p^{\mu}$:
\begin{equation*}
    p^{\mu}p_{\mu} = m^{2}u_{\mu}u^{\mu} = m^{2}c^{2} = \frac{E^{2}}{c^{2}} - p^{2}
\end{equation*}
de donde se siguen las expresiones usuales para la energía en relatividad espacial, que viene dada por
\begin{equation}
    E = \sqrt{p^{2}c^{2} + m^{2}c^{4}}
        \label{ec:CinecticaRelativista}
\end{equation}
lo cual es una expresión para la energía de una partícula que puede llegar a prescindir de la necesidad de que la partícula tenga masa para que tenga energía cinética. De acá se deduce que si la partícula está en reposo, entonces $p = 0$ y se llega a la famosa
\begin{equation*}
    E = mc^{2}
\end{equation*}
y para el caso de una partícula sin masa (como un fotón) se tiene
\begin{equation*}
    E = pc
\end{equation*}
\textit{notar la relación entre el vector de Poynting y la densidad de momento electromagnético}. Si le hago Taylor a la expresión \eqref{ec:CinecticaRelativista} para $v/c << 1$ (límite de $c \to \infty$) que sería el límite clásico se se obtiene 
\begin{equation*}
    E \approx m c^{2} + \frac{1}{2}mv^{2}
\end{equation*}
y como la energía está siempre definida a diferencia de una constante arbitraria, puedo tomar que $mc^{2} = 0$ y recupero la expresión para la energía cinética clásica.\\
\indent \textbf{Ejemplo:} Carga libre en un capacitor con campo eléctrico uniforme: Sin hacer las cuentas, si la partícula está en reposo cuando prendo el campo eléctrico, lo que sucede es que esta se acelera hasta que asintóticamente su velocidad tiende a la velocidad de la luz.\\
\indent \textbf{Ejemplo:} Carga libre en un campo magnético uniforme:


%fin clase 22
%%%%%%%%%%%%%%%%%%%%%%%%%%%%%%%%%%%%%%%%%%%%%%%%
%%%%%%%%%%%%%%%%%%%%%%%%%%%%%%%%%%%%%%%%%%%%%%%%
%%%%%%%%%%%%%%%%%%%%%%%%%%%%%%%%%%%%%%%%%%%%%%%%
%inicio clase 23 

\newpage
\section{Campos de cargas en movimiento}
Hasta ahora, vimos la evolución de los campos electromagnéticos en el tiempo, o la mecánica de cargas en campos electromagnéticos prescritos. El primer caso lo estudiamos con las ecuaciones de Maxwell, el segundo con la extensión relativista de las ecuaciones de la mecánica clásica. Sin embargo, el movimiento de las cargas genera campos electromagnéticos \textit{propios} que no consideramos (la única excepción es el caso MHD (magnetohidrodinámico?), donde consideramos la ecuación de inducción acoplada con la de conservación de movimiento; pero allí $v/c << 1$ y despreciamos las corrientes de desplazamiento, luego no teníamos pérdidas por radiación).\\
\indent Vimos también que al \textit{prender} un dipolo teníamos campos que decaían como $1/r$ (campos de radiación), los cuales eran capaces de llevarse flujo de energía hacia el infinito. En ese ejemplo, se halló la solución completa de los campos, que incluían una parte de campo cercano o cuasiestacionario, y una parte de campo lejano o de radiación. Logramos desarrollar mecanismos en los que solo resolvimos los ecuaciones de forma de llegar a las soluciones de campo cercano para disminuir la complejidad de los cálculos al resolver la ecuación completa. Se quiere algo análogo a eso pero para los campos de radiación: Desarrollar mecanismos para el cálculo de los campos de radiación sin la necesidad de conocer como se comportan cerca de la fuente.\\
\indent Olvidemos por un momento el acoplamiento y estudiemos el campo generado por cargas en movimiento.

%%%%%%%%%%%%%%%%%%%%%%%%%%%%%%%%%%%%%%%%%%%
%%%%%%%%%%%%%%%%%%%%%%%%%%%%%%%%%%%%%%%%%%%

\subsection{Potenciales de Liénard-Wiechert}
Vamos a considerar una carga puntual $q$ que se mueve en una trayectoria prescrita, por ejemplo, un electrón clásico orbitando alrededor de un núcleo (átomo de hidrógeno). La trayectoria puede ser cualquier cosa, puede ser no cerrada, pero lo importante es que se pueda parametrizar y que sea dato. En el gauge de Lorenz $\nabla \cdot \textbf{A} + \frac{1}{c}\frac{\partial \varphi}{\partial t} = 0$ y las ecuaciones de ondas para los potenciales se escribían
\begin{equation*}
    \left.
        \begin{array}{ll}
             \nabla^{2}\varphi 
             - \frac{1}{c} \frac{\partial^{2}\varphi}{\partial t^{2}} 
             & = -\frac{4\pi}{c}c\rho  \\
             &\\
             \nabla^{2}\varphi
             - \frac{1}{c} \frac{\partial^{2}\textbf{A}}{\partial t^{2}} 
             & = - \frac{4\pi}{c}\textbf{j} 
        \end{array}
    \right\}
    \quad
    \mbox{o}
    \quad
    \square A^{\mu} = \frac{4\pi}{c}J^{\mu}
\end{equation*}
Si las condiciones de contorno son tales que $\varphi = \textbf{A} = 0$ antes que el sistema empiece a evolucionar, las soluciones se escriben en términos de los potenciales retardados (los avanzados describen, por ejemplo, el caso en se apagan los campos).\\
\indent Entonces, la función de Green retardada era
\begin{equation*}
    G^{+}(\textbf{r},t;\textbf{r}',t') =
    \frac%
    {%
        \delta
        \left[
            t' - 
            \left(
                t - \frac{|\textbf{r}-\textbf{r}'|}{c}
            \right)
        \right]
    }
    {%
        |\textbf{r}-\textbf{r}'|
    }
\end{equation*}
y las ecuaciones para los potenciales usando la función de Green se puede resumir en
\begin{equation}
    \left.
        \begin{matrix}
            \varphi(\textbf{r},t) = \int
            \frac%
            {%
                \rho(\textbf{r}',t')\delta
                \left[
                    t' - 
                    \left(
                        t - \frac{|\textbf{r}-\textbf{r}'|}{c}
                    \right)
                \right]
            }
            {%
                |\textbf{r}-\textbf{r}'|
            }dt'd^{3}r'\\
            \\
            \textbf{A}(\textbf{r},t) = \int 
            \frac%
            {%
                \textbf{j}(\textbf{r}',t')\delta
                \left[
                    t' - 
                    \left(
                        t - \frac{|\textbf{r}-\textbf{r}'|}{c}
                    \right)
                \right]
            }
            {%
                c|\textbf{r}-\textbf{r}'|
            }dt'd^{3}r'
        \end{matrix}
    \right\}
    \Longrightarrow
    A^{\mu}(x) = \frac{1}{c}\int J^{x'}G^{+}(x,x')d^{4}x'
        \label{ec:PotencialesRetardados}
\end{equation}
Consideramos una carga en movimiento $\textbf{r}' = \textbf{r}_{0}(t')$, entonces la densidad de carga puede escribirse como el valor de la carga por una delta de Dirac centrada $\textbf{r}_{0}(t)'$, o sea que se mueve:
\begin{equation*}
    \rho(\textbf{r}',t') = q \delta
    \left[
        \textbf{r}' - \textbf{r}_{0}(t')
    \right]
\end{equation*}
y como la densidad de corriente es generada por el desplazamiento de cargas, la puedo escribir como la densidad de cargas por la velocidad de desplazamiento, que en este caso sería $\dot{\textbf{r}}_{0}$, entonces
\begin{equation*}
    \textbf{j}(\textbf{r}', t') = q\,\dot{\textbf{r}}_{0}(t')\delta[\textbf{r}'-\textbf{r}_{0}(t')]
\end{equation*}
Notar que por ahora $\textbf{r}_{0}(t')$ es una función conocida. Reemplazando las fuentes en las expresiones para los potenciales e integrando respecto de $r'$, que gracias a la delta de Dirac no es más que reemplazar $\textbf{r}' \to \textbf{r}_{0}(t')$
\begin{equation*}
    \varphi(\textbf{r},t) = \int q 
    \frac
    {
        \delta 
        \left[
            t' - 
            \left(
                t - \frac{|\textbf{r}-\textbf{r}_{0}(t')|}{c}
            \right)
        \right]
    }
    {
        |\textbf{r}-\textbf{r}_{0}(t')|
    }dt'
\end{equation*}
pero ahora hay que integrar esta ecuación respecto al $t'$, y no es trivial porque tengo la trayectoria en función de $t'$: $\textbf{r}_{0}(t')$. Para eso hay que usar la siguiente propiedad
\begin{equation*}
    \delta(f(x)) = \frac{\delta(x - f^{-1}(0))}{|f'(f^{-1}(0))|}
\end{equation*}
donde $f^{-1}(0)$ son las raíces de $f(x)$. Para probarlo lo que quiero hacer es ver que frente a cualquier función de prueba, ambas expresiones de comportan de la misma manera\footnote{Ver apéndice 52}. No voy a hacer esas cuentas, lo voy a usar y se llega a
\begin{equation*}
    f(t') = t' - t + \frac{|\textbf{r}-\textbf{r}_{0}(t')|}{c} = 0
    \Longrightarrow c(t - t') = |\textbf{r} - \textbf{r}_{0}(t')|
\end{equation*}
La señal que sale en $t'$ de $\textbf{r}_{0}(t')$ llega a $\textbf{r}$ después de un $\Delta t = |\textbf{r}-\textbf{r}_{0}(t')|/c$. El valor de $t'$ que hace que $c(t-t') = |\textbf{r}-\textbf{r}_{0}(t')|$ es el $t_{ret}$ (tiempo retardado). Es el tiempo en el que se emite la onda electromagnética que será vista en la posición $\textbf{r}$ un tiempo posterior $t$. Además, $t_{ret}$ es único, porque no existe otro tiempo que cumpla que la distancia recorrida por la onda electromagnética sea $|\textbf{r}-\textbf{r}_{0}(t')|$. Calculando ahora la derivada de $f$ y reemplazando \footnote{Ver apéndice 52} se termina obteniendo la expresión para el potencial, tomando $R = |\textbf{r}-\textbf{r}_{0}(t')|$
\begin{equation*}
    \varphi(\textbf{r},t) = 
    \left.
        \frac{q}{R(1 - \hat{n}\cdot \bar{\beta})}
    \right|_{tret}
\end{equation*}
y lo mismo para el potencial vector $\textbf{A}$
\begin{equation*}
    \textbf{A}(\textbf{r},t) = 
    \left.
        \frac{q}{R(1 - \hat{n}\cdot \bar{\beta})}
        \bar{\beta}
    \right|_{tret}
\end{equation*}
Son los potenciales de Liénard-Wiechert. $\hat{n}$ es el versor en la dirección del punto fuente - punto campo. $\bar{\beta} \to \frac{v}{c}$ de la fuente en $t_{ret}$.\\
\indent Ahora que ya tengo los potenciales puedo calcular los campos, nuevamente usando las expresiones de las ecuaciones de Maxwell con forma de ecuación de ondas para el gauge de Lorenz
\begin{equation*}
    \left\{
        \begin{array}{ll}
             \textbf{E} = & 
             -\nabla \varphi 
             - \frac{1}{c}\frac{\partial \textbf{A}}{\partial t}  \\
             &\\
             \textbf{B} = & \nabla \times \textbf{A}
        \end{array}
    \right.
\end{equation*}
y finalmente los campos quedan \footnote{Ver apéndice HACER APÉNDICE DE ESTO POR DIOS}
\begin{equation*}
    \textbf{E} = 
    \left.
        \frac
        {
            q(\hat{n}-\bar{\beta})
        }
        {
            \gamma^{2}(1 - \bar{\beta}\cdot \hat{n})^{3}R^{2}
        }
    \right|_{t_{ret}}
    +
    \frac{q}{c}
    \left.
        \frac
        {
            \hat{n}\times
            \left[
                (\hat{n} - \bar{\beta})\times \dot{\bar{\beta}}
            \right]
        }
        {
            (1 - \bar{\beta}\cdot \hat{n})^{3}R
        }
    \right|_{t_{ret}},
    \quad
    \quad
    \quad
    \textbf{B} = \hat{n}\times \textbf{E}\big)_{t_{ret}}
\end{equation*}
tenemos dos términos en los campos: El primero depende de $\textbf{r}_{0}$ y $\bar{\beta}$ y decae como $1/R^{2}$ y es el \textbf{campo de velocidades}, es proporcional a la velocidad de las cargas. El segundo término depende de $\dot{\beta}$ y decae como $1/R$, es el \textbf{campo de aceleraciones o radiación}. Este término está presente sólo si el movimiento es acelerado y lleva energía al infinito.\\
\indent Aparecen dos tipos de efectos relativistas: el que tiene que ver con el ángulo formado por $\bar{\beta}$ y $\dot{\bar{\beta}}$ (numerador), y el que tiene que ver con la transformación del sistema de referencia de la carga al del observador y van como $1 - \bar{\beta} \cdot \hat{n}$.\\
\indent Por partes qué es cada cosa:
\begin{itemize}
    \item $\hat{n}$: Es la dirección en la que apunta el vector entre la posición de la carga en $\textbf{r}_{0}(t')$ a tiempo (retardado) $t'$ y la posición del observador $\textbf{r}$ en tiempo $t$.
    \item $\bar{\beta}$: Es la dirección en la que se desplaza la carga, es la tangente a la trayectoria $\textbf{r}_{0}(t')$ de la carga a un dado $t'$.
    \item $\dot{\bar{\beta}}$: Es la dirección en la que apunta la aceleración de carga, que puede o no ser coincidente con la dirección de $\bar{\beta}$ (por ejemplo, una trayectoria circular es un movimiento acelerado y en ese caso $\bar{\beta} \perp \dot{\bar{\beta}}$).
\end{itemize}
\indent \textbf{Obs:} Si bien quedan las expresiones con términos donde aparece $\beta$ que se definió en relatividad, es simplemente una comodidad de la notación. Lo que se hizo hasta acá no es relatividad, es resolver las ecuaciones de Maxwell. Sin embargo, no hay que perder de vista que estas ecuaciones son invariantes de Lorentz, porque las ecuaciones de Maxwell lo son, entonces pueden reescribirse en términos de cuadrivectores. Pero no hay ningún concepto de la relatividad aplicado en estas deducciones.\\
\indent Veamos algunos casos particulares.
\begin{itemize}
    \item Para una carga en reposo, el campo de velocidades es el único que sobrevive (ya que $\dot{\bar{\beta}} = 0$) y $\bar{\beta} = 0$, de forma que la expresión se simplifica a:
    \begin{equation*}
        \textbf{E} = \frac{q\hat{n}}{R^{2}}
    \end{equation*}
    es la ley de Coulomb! El campo eléctrico es radial a la carga.
    \item Para una carga con movimiento uniforme:
    \begin{equation*}
        \textbf{E} = 
        \left.
            \frac
            {
                q
                (\hat{n} - \bar{\beta})
            }
            {
                \gamma^{2}(1-\bar{\beta}\cdot\hat{n})^{3}R^{2}
            }
        \right|_{t_{ret}}
    \end{equation*}
    y este resultado así como está escrito no dice nada, pero físicamente implica algo para nada intuitivo. Primero que nada, el numerador va como $(\hat{n} - \bar{\beta})$, es de carácter vectorial, mientras que el término del denominador $1 - \bar{\beta}\cdot\hat{n}$ es un escalar. Entonces, la dirección del campo está definida por el numerador de la expresión, que es $\hat{n} - \bar{\beta}$.\\
    \indent Recordando que $\hat{n}$ es la dirección del vector que une la posición $(\textbf{r},t)$ del observador con la carga $\textbf{r}_{0}(t')$, y definiendo un nuevo vector $\textbf{p}$ como el vector que une a la posición de la carga en tiempo $t$, $\textbf{r}_{0}(t)$, con la posición del observador a tiempo $t$, $(\textbf{r}, t)$, se puede demostrar\footnote{Ver apéndice 53} que la dirección del campo eléctrico, dada por $\hat{n} - \bar{\beta}$ coincide con la dirección del vector $\textbf{p}$. Esto quiere decir que el campo en $(\textbf{r},t)$ es radial respecto al punto en el que está en ese instante la carga con $\textbf{v}$ uniforme, en $\textbf{r}_{0}(t)$, a pesar de que el campo es el generado por la carga en $\textbf{r}_{0}(t')$ (en un tiempo anterior).\\
    \indent Esto es un resultado relativista y a simple viste parecería que se viola algún postulado porque, de cierta forma, parecería que la velocidad de transmisión de la información es infinita: ¿Cómo puede ser posible que en tiempo $t$, vea el campo generado por una carga en en el mismo instante de tiempo $t$ que se encuentra a una distancia $|\textbf{r}(t) - \textbf{r}_{0}(t)|$? Esto solo sería posible si la información se transmitiera de forma instantánea. Resulta que no, lo que está pasando es que la partícula se mueve en movimiento rectilíneo, y justo pasa que en este ejemplo la posición aparente de la carga $\textbf{r}_{0}(t')$ (con $t'$ un tiempo anterior a $t$) genera un campo que, al alcanzar al observador, coincide con el generado por la posición actual de la carga $\textbf{r}_{0}(t)$, como si ahora estuviera en reposo (campo radial).\\
    \indent En general, si la carga se mueve en una trayectoria no lineal, el campo generado por la carga en $t'$ y visto en $t$ no coinciden con el campo que generaría la carga en su posición real.
    \item Para una carga que se mueve en movimiento acelerado tenemos
    \begin{equation*}
        \textbf{E}_{rad} = 
        \frac{q}{c}
        \left.
            \frac
            {
                \hat{n}\times
                \left[
                    (\hat{n} - \bar{\beta})\times \dot{\bar{\beta}}
                \right]
            }
            {
                (1 - \bar{\beta}\cdot \hat{n})^{3}R
            }
        \right|_{t_{ret}}
    \end{equation*}
    y 
    \begin{equation*}
        \textbf{B}_{rad} = \hat{n}\times\textbf{E}_{rad}
    \end{equation*}
    tanto $\textbf{E}_{rad}$ como $\textbf{B}_{rad}$ y $\hat{n}$ son perpendiculares. Voy a mirar el caso para velocidades bajas ($\beta \approx 0$) se tiene
    \begin{equation*}
        \textbf{E}_{rad} = \frac{q}{c}
        \left.
            \frac{\hat{n}\times (\hat{n}\times \dot{\bar{\beta}})}{R}
        \right|_{t_{ret}}
    \end{equation*}
    y en este caso el campo eléctrico va en la dirección $\hat{n}\times \hat{n}\times \dot{\bar{\beta}}$, que haciendo las cuentas\footnote{Ver apéndice 53.bis Pero igual no llegué a nada, pero bueno, está el dibujito} se ve que campo eléctrico está polarizado en el plano que contiene a $\dot{\bar{\beta}}$ y $\hat{n}$ y además, $\textbf{E}_{rad} \propto -\dot{\bar{\beta}}_{\perp}$, o sea que el campo magnético apunta en la dirección contraria a la componente perpendicular a $\hat{n}$ de la aceleración $\dot{\bar{\beta}}$.
\end{itemize}



%fin clase 23
%%%%%%%%%%%%%%%%%%%%%%%%%%%%%%%%%%%%%%%%%%%
%%%%%%%%%%%%%%%%%%%%%%%%%%%%%%%%%%%%%%%%%%%
%inicio clase 24


\subsection{Potencia Irradiada}
Se quiere calcular la potencia irradiada por una carga $q$. Para eso supongo una superficie imaginaria esférica que contiene a la carga y se quiere calcular el flujo de campo que energía que atraviesa la superficie. Por conservación de la carga, el flujo de energía que salga por la superficie deberá ser igual a la energía cinética que pierda la carga en el movimiento.\\
\indent Tenemos el flujo instantáneo de energía a partir de vector de Poynting\footnote{Con unidades de $\frac{E}{T A}$ Energía por unidad de Tiempo por unidad de Area}, usando solamente los campos de radiación sería
\begin{equation*}
    \textbf{S} = \frac{c}{4\pi} \textbf{E}\times \textbf{B} 
    = \frac{c}{4\pi}\textbf{E}_{rad} \times (\hat{n} \times \textbf{E}_{rad})
    = \frac{c}{4\pi} |\textbf{E}_{rad}|^{2}\hat{n}
\end{equation*}
la potencia irradiada por diferencial de área es $dP = \textbf{S}\cdot d\textbf{A}$, donde $d\textbf{A}$ es el diferencial de ángulo sólido por el radio de la esfera imaginaria por el versor $\hat{n}$: $d\textbf{A} = R^{2}d\Omega \hat{n}$ reemplazando y despejando se llega a la potencia irradiada por unidad de ángulo sólido por unidad de tiempo como
\begin{equation*}
    \frac{dP}{d\Omega} = \frac{q^{2}}{4\pi c}
    \left|
        \frac%
        {%
            \hat{n}\times (\hat{n} - \bar{\beta})\times \dot{\bar{\beta}}
        }%
        {%
            (1- \bar{\beta}\cdot \hat{n})
        }^{3}
    \right|^{2}
\end{equation*}
Ahora veamos el límite clásico y el límite relativista. Para el límite de velocidades bajas 
\begin{equation}
    \frac{dP}{d\Omega} = \frac{q^{2}}{4\pi c}|\dot{\textbf{v}}|^{2}
    \sin^{2}{(\theta)}
        \label{ec:PotenciaIrradiadaPuntual}
\end{equation}
supongo el caso particular en el que la partícula está acelerada en la misma dirección de movimiento, de forma que $\bar{\beta} \parallel \dot{\bar{\beta}}$, esto puede ser una carga en movimiento rectilíneo uniforme, un oscilador armónico, etc. $\theta$ es el ángulo formado entre la dirección de movimiento de la carga y el el vector en dirección $\hat{n}$ que apunta al observador. Se puede dibujar el patrón de radiación\footnote{Ver apéndice 53.8} (ver dibujito clase 24, pagina 2), pero lo importante es notar que la potencia irradiada depende del ángulo $\theta$: Si el observador se posiciona sobre la misma linea de movimiento de la carga, entonces $\theta = 0$ y la potencia que ve el observador es $0$. Si el observador se posiciona transversalmente al movimiento de la carga, $\theta = \pi/2$ y $\sin{(\pi/2)}$ es máximo, de forma que la potencia irradiada es máxima. Además, la potencia no depende del ángulo $\phi$, con lo cual hay simetría de rotación alrededor del eje de desplazamiento de la carga.\\
\indent Se puede integrar sobre el ángulo sólido y se tiene la potencial total irradiada
\begin{equation}
    P = \frac{2}{3}\frac{q^{2}}{c^{3}}|\dot{\textbf{v}}|^{2}
        \label{ec:Larmor}
\end{equation}
que es la fórmula de \textbf{Larmor}.\\
\indent En el caso relativista, debemos tener cudiado, pues $\frac{dP}{d\Omega}$ es energía irradiada por unidad de área y de tiempo en el sistema del observador, y queremos la potencia irradiada por la fuente, es decir, por unidad de tiempo en el sistema de la fuente (tiempo proprio).\\
\indent Tenemos 
\begin{equation*}
    \frac{dP}{d\Omega} 
    = \frac{dE}{dt\,d\Omega}
    = R^{2} \textbf{S}\cdot \hat{n}
\end{equation*}
donde $dE$ es diferencial de energía y queremos ver esto en el tiempo $t'$ \footnote{Ver apéndice 53.8}
\begin{equation*}
    \frac{dP(t')}{d\Omega} 
    = ... 
    = R^{2} \textbf{S} \cdot \hat{n} (1 - \bar{\beta}\cdot \hat{n})
\end{equation*}
finalmente, la potencia irradiada por la fuente en la caso relativista será
\begin{equation*}
    \frac{dP(t')}{d\Omega}
    = \frac{q^{2}}{4\pi c}
    \frac%
    {%
        |\hat{n} \times (\hat{n} - \bar{\beta})\times \dot{\bar{\beta}}|^{2}
    }%
    {%
        (1 - \bar{\beta}\cdot \hat{n})^{5}
    }
\end{equation*}
Pero volviendo a poner como ejemplo el caso en el que el desplazamiento es colineal con la aceleración: $\dot{\bar{\beta}}\parallel \bar{\beta}$ (movimiento rectilíneo), la expresión se simplifica a
\begin{equation*}
    \frac{dP(t')}{d\Omega} = 
    \frac{q^{2}}{4\pi c}|\dot{\bar{\beta}}|^{2}
    \frac{\sin^{2}{(\theta)}}{(1-\beta \cos{(\theta)})^{5}}
\end{equation*}
con lo cual, la dirección en la que la potencia irradiada es máxima depende de esa relación $\sin^{2}{(\theta)}/(1 - \beta \cos{(\theta)})$ y podemos adelantar que no tendrá la misma forma que en el caso clásico. Para verlo, hago un cambio de variables $x = \cos{(\theta)}$ y busco máximos de la función resultante. De esto se obtiene que
\begin{equation*}
    x_{max} 
    = \cos{(\theta_{max})} 
    = \frac{\sqrt{1 + 15 \beta^{2}} - 1}{3\beta}
\end{equation*}
y tiende a $1$ cuando $\beta \to 1$. Esto que a medida que la velocidad de desplazamiento de la carga tiende más a la velocidad de la luz, el ángulo $\theta_{max}$ que maximiza la potencia irradiada se vuelve más y más pequeño. Entonces la radiación queda confinada a un cono delgado en la dirección de movimiento\footnote{Ver video cláse 24 minuto 45} (cuando es por frenado por colisiones Coulomb se llama Bremsstrahlung).



%%%%%%%%%%%%%%%%%%%%%%%%%%%%%%%%%%%%%%%%%%%
%%%%%%%%%%%%%%%%%%%%%%%%%%%%%%%%%%%%%%%%%%%



\subsection{Radiación de fuentes localizadas}
Hasta ahora se pudo calcular la solución exacta a una carga puntual desplazándose en una trayectoria dada y se obtuvieron los potenciales de Lienard-Wiechert.\\
\indent Consideremos ahora una distribución de carga extensa, en el que ya no se tiene una única carga puntual. El camino natural para resolver esto sería usar las funciones de Green avanzadas y/o retardadas, e integrar en toda la distribución para obtener los potenciales. Sin embargo, esto tiene el grave problema de que cada diferencial de carga posicionado en diferentes puntos del espacio tendrá un tiempo propio diferente. La solución es considerar a la fuente como localizada y ver los campos de radiación lejos, es decir, haciendo aproximaciones (como caen como $1/R$ serán los dominantes). Haciendo análisis de Fourier se tiene
\begin{equation*}
    \left\{
        \begin{matrix}
            \rho(\textbf{r}', t') = \rho(\textbf{r}')\,e^{-i\omega t'}\\
            \\
            \textbf{j}(\textbf{r}', t') = \textbf{j}(\textbf{r}')\,e^{-i\omega t'}
        \end{matrix}
    \right.
\end{equation*}
dado el potencial vector $\textbf{A}$, podemos obtener todos todos los campos. El campo magnético viene dado por $\textbf{B} = \nabla \times \textbf{A}$ y luego, lejos de las fuentes puedo escribir $\nabla \times \textbf{B} = \frac{1}{c}\frac{\partial \textbf{E}}{\partial t} + \cancel{\frac{4\pi}{c}\textbf{j}}$, donde el término con corrientes lo cancelo porque no lejos de ellas no hay corriente. Entonces, usando que $\textbf{E} = \textbf{E}e^{-i\omega t}$ puedo escribir el campo eléctrico como sigue
\begin{equation*}
    -\frac{i\omega}{c}\textbf{E} = \nabla \times \textbf{B}
    \quad
    \quad
    \mbox{y}
    \quad
    \quad
    \textbf{E} = \frac{i}{k}\nabla \times \textbf{B}
\end{equation*}
O sea que con $\textbf{A}$ puedo escribir tanto $\textbf{E}$ como $\textbf{B}$.\\
\indent La solución de $\textbf{A}$ está dada por los potenciales retardados de \eqref{ec:PotencialesRetardados}, reemplazando por la expresión de $\textbf{j}(\textbf{r}',t')$ e integrando en $t'$ se llega
\begin{equation*}
    \textbf{A}(\textbf{r},t) = \frac{1}{c}
    \int d^{3}r'
    \frac%
    {%
        \textbf{j}(\textbf{r}')
        e^%
        {%
            -i\omega
            \left(
                t - \frac{|\textbf{r}-\textbf{r}'|}{c}
            \right)
        }
    }%
    {%
        |\textbf{r}- \textbf{r}'|
    }
    = \frac{e^{-i\omega t}}{c}
    \int d^{3}r'
    \frac%
    {%
        \textbf{j}(\textbf{r}')e^{ik|\textbf{r}-\textbf{r}'|}
    }%
    {%
        |\textbf{r}-\textbf{r}'|
    }
\end{equation*}
pero hay que tener en cuenta que para cada punto fuente va a haber un $t_{ret} = t - |\textbf{r}-\textbf{r}'0|/c$ diferente, porque cada punto fuente está a una distancia diferente del observador.\\% Y esta expresión está escrita en función de $\textbf{r}' = \textbf{r}(t')$, con lo cual no es trivial de integrar en $r'$.\\
\indent Entonces lo que hay que hacer es meter la primera aproximación: Si considero que la distribución de cargas tiene un tamaño característico $d$ que es mucho menor que la distancia $r$ a la que se encuentra el observador y que, a su vez, la distancia $r$ es mucho menor que la longitud de onda $\lambda$ de la fuente, entonces estamos en el caso donde vale cuasiestacionario: $d << r << \lambda$.\\
\indent En cambio el caso en el que tengo $d << \lambda << r$, estoy en la región de campo lejano: la longitud de onda es mucho menor que la distancia, con lo cual la variación de los campos es apreciable. Para resolver las ecuaciones en esta aproximación, hay dos formas de hacerlo: Una sería hacer el desarrollo multipolar en esféricas del potencial vector $\textbf{A}$ y se terminaría llegando a
\begin{equation*}
    \textbf{A}(\textbf{r},t) = 
    \frac{e^{-i\omega t}}{c}
    \int d^{3}r'\textbf{j}(\textbf{r}')
    \left\{
        4\pi i k
        \sum\limits_{l = 0}^{\infty}
        J_{l}(kr_{<})h_{l}^{(1)}(kr_{>})
        \sum\limits_{m = -l}^{l}
        Y_{lm}^{*}(\theta', \phi')Y_{lm}(\theta, \phi)
    \right\}
\end{equation*}
con $J_{l}$ las funciones esféricas de Bessel y $h_{l}^{(1)}$ las funciones esféricas de Hankel de tipo uno. Pero esto es un bardo así que Mininni decide hacerlo en cartesianas y manipular los coeficientes para llegar al resultado de forma más fácil.\\
\indent Aproximando $r' << r$ (porque $d << r$), de forma que $\frac{1}{|\textbf{r}-\textbf{r}'|} \approx \frac{1}{r}$ y los términos de órdenes más altos no me interesan porque solo quiero ver los términos de radiación que caen como $1/r$, y además la fase con Taylor se reduce a $e^{ik|\textbf{r}-\textbf{r}'|} \approx e^{ik(r - \hat{n}\cdot \textbf{r}')}$. Reemplazando en la expresión del potencial vector se llega a
\begin{equation}
    \textbf{A}(\textbf{r},t) = 
    \frac{e^{i(kr - \omega t)}}{cr}
    \int \textbf{j}(\textbf{r}') e^{-ik\hat{n}\cdot \textbf{r}'}\,
    d^{3}r'
        \label{ec:PotencialVectorRadiacion}
\end{equation}
hasta acá solo use la parte de la aproximación en que el observador en $\textbf{r}$ está muy lejos de la fuente, o sea que $r >> r'$.\\ \indent Todavía se puede usar la aproximación en la que $r >>\lambda >> d$, es decir, la longitud de onda es muy grande comparada con el tamaño de la fuente. En ese caso tenemos $|k\hat{n}\cdot \textbf{r}'|\sim |kd| << 1$, ya que $d$ es el tamaño característico de la fuente, con lo cual el punto fuente más grande es de tamaño $d$. Entonces, usando que $d << \lambda$ se puede aproximar la fase tomando $e^{-ik\hat{n}\cdot \textbf{r}'} \approx 1 - i k\hat{n} \cdot \textbf{r}' + ....$. Usando esta aproximación, para los diferentes órdenes de la fase voy a ir escribiendo los diferentes términos del desarrollo. Queda un desarrollo en función de potencias del número de onda.
\subsubsection{Término dipolar eléctrico}
Tomamos a orden cero, entonces $e^{-ik\hat{n}\cdot \textbf{r}'} \sim 1$, y queda
\begin{equation*}
    \textbf{A}(\textbf{r},t) \approx \frac{e^{i(kr - \omega t)}}{cr}
    \int \textbf{j}(\textbf{r}')d^{3}r'
\end{equation*}
y la integral de la corriente en el volumen se puede reescribir usando partes y llego a\footnote{Ver apéndice 55.1}
\begin{equation*}
    \textbf{A}(\textbf{r},t) \approx - 
    \frac{e^{i(kr -\omega t)}}{cr}i\omega 
    \underbrace{%
        \int \textbf{r}'\rho(\textbf{r}')d^{3}r'
        }_{\textbf{p}}
\end{equation*}
donde $\textbf{p}$ es el momento dipolar eléctrico en el caso independiente del tiempo! entonces 
\begin{equation}
    \textbf{A}^{(0)}(\textbf{r},t) = - i k\, \textbf{p}\,
    \frac{e^{i(kr - \omega t)}}{r}
        \label{ec:TerminoDipolarElectricoA}
\end{equation}
es el potencial vector asociado al campo de radiación al orden más bajo generado por una distribución de cargas. Además es una onda esférica viajera\footnote{Ver apéndice 55.1}
\begin{equation*}
    \textbf{B}^{(0)} 
    = \nabla \times \textbf{A}^{(0)} 
    = k^{2}(\hat{n} \times \textbf{p}) 
    \frac{e^{i(kr - \omega t)}}{r}
    \left(
        1 - \frac{1}{ikr}
    \right)
\end{equation*}
y como yo solo quiero el término del campo magnético (a orden $0$) que viene del campo de radiación, puedo tirar todo aquello que caiga más rápido que $1/r$, de forma que el segundo término entre paréntesis a la derecha, muere ($\lambda << r$ entonces $kr >> 1$) y entonces\footnote{Notar que hacerle rotor al potencial vector termina siendo hacer la siguiente cuenta: $\textbf{B} = \nabla \times \textbf{A} = ik\hat{n}\times \textbf{A}$}
\begin{equation}
    \textbf{B}^{(0)}(\textbf{r}, t) \approx
    k^{2} (\hat{n}\times \textbf{p})
    \frac{e^{i(kr - \omega t)}}{r}
        \label{ec:TerminoDipolarMagneticoB}
\end{equation}
y el campo eléctrico a orden cero será
\begin{equation*}
    \textbf{E}^{(0)}(\textbf{r},t)
    = \frac{i}{k}\nabla \times \textbf{B}
    = k^{2}(\hat{n}\times \textbf{p}) \times \hat{n}
    \frac{e^{i(kr - \omega t)}}{r}
    = \textbf{B} \times \hat{n}
\end{equation*}
%
%
%
%fin clase 24
%%%%%%%%%%%%%%%%%%%%%%%%%%%%%%%%%%%%%%%%%%%
%%%%%%%%%%%%%%%%%%%%%%%%%%%%%%%%%%%%%%%%%%%
%inicio clase 25 
%
calculemos la potencia irradiada usando $\frac{dP}{d\Omega} = r^{2} \textbf{S}\cdot \hat{n}$ y en el caso complejo nos interesa el valor medio sobre un periodo de la onda para el vector de Poynting
\begin{equation*}
    \mean{\textbf{S}} = ... = \frac{c}{8\pi} |\textbf{B}|^{2}\hat{n}
\end{equation*}
entonces, la potencia irradiada por unidad de ángulo sólido y unidad de tiempo(? será
\begin{equation}
    \frac{dP}{d\Omega}
    = \frac{ck^{4}}{8\pi}|\hat{n}\times \textbf{p}|^{2}
    = \frac{ck^{4}}{8\pi}|\textbf{p}|^{2}\sin^{2}{(\theta)}
        \label{ec:PotenciaIrradiadaDipolarElectrico}
\end{equation}
que es una expresión muy similar al caso de la carga puntual \eqref{ec:PotenciaIrradiadaPuntual}, ambas quedan en función de $\sin^{2}{(\theta)}$, pero esta ultima, a diferencia de la potencia irradiada por la carga puntual, depende del momento dipolar eléctrico $\textbf{p}$, mientras que la otra dependía de la aceleración de la carga. La forma de función es idéntica, es un lóbulo de simetría de rotación en el eje de movimiento de la distribución de carga y con potencial máxima en $\theta = 2\pi$.\\ \indent Integrando sobre el ángulo sólido para tener la potencia total
\begin{equation*}
    P = \frac{ck^{4}}{3}|\textbf{p}|^{2}
\end{equation*}
es la potencia irradiada y es proporcional a la frecuencia a la cuarta $\omega^{4}$. Esto quiere decir que frecuencias más altas irradian más potencia (son ondas más energéticas) que las frecuencias bajas. En el espectro visible, la radiación en el rojo es menor y en el violeta mayor, esto es importante para entender por qué el cielo es azul en Scattering de Rayleigh)\\
\indent La polarización del campo de radiación es (dibujito de la clase)\\
\textbf{Ejemplo: Antena}


%%%%%%%%%%%%%%%%%%%%%%%%%%%%%%%%%%%%%%%%%%%
%%%%%%%%%%%%%%%%%%%%%%%%%%%%%%%%%%%%%%%%%%%


\subsubsection{Términos dipolar magnético y cuadrupolar eléctrico}
Al orden siguiente en la expansión se tiene $e^{-ik\hat{n}\cdot \textbf{r}'} \approx 1 - k\hat{n}\cdot \textbf{r}' + ... $. Como ya reemplacé a orden $0$, toca reemplazar el orden $1$ que es $k\hat{n}\cdot \textbf{r}'$ en la expresión del potencial vector $\textbf{A}$, \eqref{ec:PotencialVectorRadiacion}. Sacando fuera de la integral todo lo que no depende de $\textbf{r}'$ se llega a
\begin{equation*}
    \textbf{A}(\textbf{r},t) 
    = -\frac{i k e^{i(kr - \omega t)}}{cr}
    \int \textbf{j}(\textbf{r}')(\hat{n}\cdot \textbf{r}')d^{3}r'
\end{equation*}
y la integral de la derecha se puede escribir en notación de índices como
\begin{equation*}
    \int j_{i}n_{l}r_{l}\,dV 
    = n_{l}
    \underbrace%
    {%
        \left(
            \int j_{i}r_{l}\,dV
        \right)
    }_{\mbox{tensor}}
    = n_{l}
    \Big[
        \underbrace%
        {%
            \frac{1}{2}
            \int
            \left(
                j_{i}r_{l} - j_{l}r_{i}
            \right)\,dV
        }_{\mbox{Antisimétrico}}
        +
        \underbrace%
        {%
            \frac{1}{2}
            \int
            \left(
                j_{i}r_{l} + j_{l}r_{i}
            \right)\,dV
        }_{\mbox{Simétrico}}
    \Big]
\end{equation*}
En definitiva, terminé escribiendo la integral en términos de un tensor simétrico y otro antisimétrico.\\
\indent Como las cuentas se hacen en cartesianas, para encontrar los términos de la expansión que queremos, hay que reordenarlos convenientemente. De hacer esto en esféricas los términos hubieses surgido naturalmente, pero según Mininni era una quilombo y no aportaba nada.\\
\indent Comenzando por el \textbf{antisimétrico}, va a resultar que este término es el correspondiente a la contribución dipolar magnética de la distribución de carga. Se puede ver que para la parte antisimétrica, definiendo el momento dipolar magnético como
\begin{equation*}
    \textbf{m} = \frac{1}{2c}
    \int (\textbf{r}'\times \textbf{j})d^{3}r'
\end{equation*}
el potencial vector se escribe como\footnote{Ver apéndice 55.1}
\begin{equation*}
    \textbf{A}^{(M)} (\textbf{r},t) 
    = ik\frac{e^{i(kr - \omega t)}}{r}
    \left( \hat{n} \times \textbf{m} \right)
\end{equation*}
que es el término dipolar magnético de la expansión a orden $1$ y mirando únicamente la parte antisimétrica. Faltan determinar $\textbf{B}^{(M)}$ y $\textbf{E}^{(M)}$ para el término de radiación dipolar magnética. Salen de hacer lo siguiente
\begin{equation*}
    \textbf{B} = \nabla \times \textbf{A},
    \quad
    \quad
    \textbf{E} = \frac{i}{k} \nabla \times \textbf{B}
\end{equation*}
entonces para hallar los campos hay que aplicar los rotores sobre el término dipolar magnético para el potencial vector $\textbf{A}^{(M)}$, para obtener $\textbf{B}^{(M)}$ y luego volver a aplicar rotor sobre el campo magnético para obtener el campo eléctrico. Un atajo es ver que esa cuenta se hizo para pasar de  \eqref{ec:TerminoDipolarElectricoA} a \eqref{ec:TerminoDipolarMagneticoB}, solo que donde dice $\textbf{p}$, ahora va a decir $\hat{n}\times \textbf{m}$. De cualquier forma, haciendo la cuenta y tirando los términos que decaen más rápido que $1/r$ no los considero, y queda\footnote{Ver apéndice 55.1}
\begin{equation*}
    \textbf{B}^{(M)}(\textbf{r},t) 
    = k^{2} (\hat{n} \times \textbf{m}) \times \hat{n}
    \frac{e^{i(kr - \omega t)}}{r}
\end{equation*}
y para el campo eléctrico obtenemos,
\begin{equation*}
    \textbf{E}^{(M)} 
    = -k^{2}(\hat{n}\times \textbf{m})
    \frac{e^{i(kr - \omega t)}}{r}
    \left(
        1 - \frac{1}{ikr}
    \right)
\end{equation*}
y la parte de radiación es (despreciando términos que decaen más rápido que $1/r$)
\begin{equation*}
    \textbf{E}^{(M)}(\textbf{r},t)
    = -k^{2}(\hat{n} \times \textbf{m})
    \frac{e^{i(kr - \omega t)}}{r}
\end{equation*}
y la potencia irradiada es la misma que antes en \eqref{ec:PotenciaIrradiadaDipolarElectrico}, pero donde dice $\textbf{p}$ ahora va $\textbf{m}$
\begin{equation*}
    \frac{dP}{d\Omega}
    = \frac{ck^{4}}{8\pi}|\hat{n}\times \textbf{m}|^{2}
    = \frac{ck^{4}}{8\pi}|\textbf{m}|^{2}\sin^{2}{(\theta)}
\end{equation*}
pero lo que cambia es la polarización de los campos, en un caso los campos apuntaban en un sentido y en el otro, apuntan en otro sentido (ver dibujito pdf clase 25 página 4)\\
\indent Veamos finalmente la parte \textbf{simétrica} del tensor que sale de reescribir la integral, este es el tensor asociado al término cuadrupolar eléctrico: Tenemos
\begin{equation*}
    A_{i}^{(E)}(\textbf{r},t)
    = -\frac{k^{2}}{6}
    \frac{e^{i(kr - \omega t)}}{r}
    \left(
        Q_{ij} + C_{ll}\delta_{ij}
    \right)n_{j}
\end{equation*}
o escrita en forma vectorial
\begin{equation*}
    \textbf{A}^{(E)}(\textbf{r},t)
    = -\frac{k^{2}}{6}
    \frac{e^{i(kr - \omega t)}}{r}
    \left(
        \bar{\bar{Q}} + \Tr(\bar{\bar{C}})\bar{\bar{I}}
    \right)\hat{n}
\end{equation*}
la parte solo de radiación de los campos $\textbf{E}$ y $\textbf{B}$ para el término cuadrupolar eléctrico las podemos calcular fácilmente haciendo
\begin{equation*}
    \textbf{B}^{(E)}(\textbf{r},t) 
    = -\frac{ik^{3}}{6}\hat{n}\times(\bar{\bar{Q}}\cdot \hat{n})
    \frac{e^{i(kr - \omega t)}}{r}
\end{equation*}
y
\begin{equation*}
    \textbf{E}^{(E)}(\textbf{r},t) 
    = -\frac{ik^{3}}{6}
    \left[
        \hat{n}\times
        \left(
            \bar{\bar{Q}}\cdot \hat{n}
        \right)
    \right]
    \frac{e^{i(kr - \omega t)}}{r}
    \times \hat{n}
    =
    \textbf{B}^{(E)}\times \hat{n}
\end{equation*}
La potencia irradiada es 
\begin{equation*}
    \frac{dP}{d\Omega} 
    = \frac{ck^{6}}{288\pi}
    \left|
        \hat{n}\times 
            \left(
                \bar{\bar{Q}}\cdot \hat{n}
            \right)
    \right|^{2}
\end{equation*}


%fin clase 25
%%%%%%%%%%%%%%%%%%%%%%%%%%%%%%%%%%%%%%%%%%%
%%%%%%%%%%%%%%%%%%%%%%%%%%%%%%%%%%%%%%%%%%%
%inicio clase 26 Scattering de Rayleigh
