\documentclass[12pt]{article}
\usepackage[utf8]{inputenc}
\usepackage{anysize}
\marginsize{2cm}{2cm}{2cm}{2cm}
\usepackage{amsmath}
\usepackage{amsfonts}
\usepackage{xcolor}
\usepackage{graphics}
\usepackage{caption}
\usepackage{subcaption}
\usepackage{float}
\usepackage{tikz}
\usepackage{cancel}
\usepackage{svg}
\usepackage[hidelinks]{hyperref} %para agregar hiperlinks a las ecuaciones
\usepackage{physics}
\usepackage{qrcode}
\usepackage{multicol}
\usepackage{enumitem}
\setlist{  
  listparindent=\parindent,
  parsep=0pt,
}

\newcommand{\mean}[1]{\left\langle #1 \right\rangle}
\newcommand{\LaplaceSpherics}[1]{%20-03-2021 para escribir el laplaciano de una función en esféricas de forma rápida
\frac{1}{r}\frac{\partial^{2}}{\partial r^{2}}
    \left(
        r#1
    \right)
    +
    \frac{1}{r^{2}\sin{\theta}}
    \frac{\partial}{\partial \theta}
    \left(
        \sin{(\theta)}
        \frac{\partial #1}{\partial \theta}
    \right)
    +
    \frac{1}{r^{2}\sin^{2}{(\theta)}}
    \frac{\partial^{2} #1}{\partial \phi^{2}}}
\usepackage{multicol}
\title{Final Teo 1: Conceptos}
\author{}
%modificar date en base a las actualizaciones
\date{2020}